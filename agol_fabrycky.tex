





<!DOCTYPE html>
<html lang="en">
  <head>
    <meta charset="utf-8">
  <link rel="dns-prefetch" href="https://assets-cdn.github.com">
  <link rel="dns-prefetch" href="https://avatars0.githubusercontent.com">
  <link rel="dns-prefetch" href="https://avatars1.githubusercontent.com">
  <link rel="dns-prefetch" href="https://avatars2.githubusercontent.com">
  <link rel="dns-prefetch" href="https://avatars3.githubusercontent.com">
  <link rel="dns-prefetch" href="https://github-cloud.s3.amazonaws.com">
  <link rel="dns-prefetch" href="https://user-images.githubusercontent.com/">



  <link crossorigin="anonymous" href="https://assets-cdn.github.com/assets/frameworks-2d2d4c150f7000385741c6b992b302689ecd172246c6428904e0813be9bceca6.css" integrity="sha256-LS1MFQ9wADhXQca5krMCaJ7NFyJGxkKJBOCBO+m87KY=" media="all" rel="stylesheet" />
  <link crossorigin="anonymous" href="https://assets-cdn.github.com/assets/github-0522ae8d3b3bdc841d2f91f90efd5f1fd9040d910905674cd134ced43a6dfea6.css" integrity="sha256-BSKujTs73IQdL5H5Dv1fH9kEDZEJBWdM0TTO1Dpt/qY=" media="all" rel="stylesheet" />
  
  
  
  

  <meta name="viewport" content="width=device-width">
  
  <title>TTV_review/agol_fabrycky.tex at Dan-on-JasonS · ericagol/TTV_review</title>
  <link rel="search" type="application/opensearchdescription+xml" href="/opensearch.xml" title="GitHub">
  <link rel="fluid-icon" href="https://github.com/fluidicon.png" title="GitHub">
  <meta property="fb:app_id" content="1401488693436528">

    
    <meta content="https://avatars4.githubusercontent.com/u/243664?v=4&amp;s=400" property="og:image" /><meta content="GitHub" property="og:site_name" /><meta content="object" property="og:type" /><meta content="ericagol/TTV_review" property="og:title" /><meta content="https://github.com/ericagol/TTV_review" property="og:url" /><meta content="TTV_review - Chapter for Springer&#39;s Exoplanet Handbook on TTV/TDV" property="og:description" />

  <link rel="assets" href="https://assets-cdn.github.com/">
  <link rel="web-socket" href="wss://live.github.com/_sockets/VjI6MTg4NDgwNzc5OmNlNzIwNDdjZDFkNWU2M2UwN2M3ODE2ZWE3MjU4ZGNmZTU1MzFjYWEyYjg3OTEzMzM0YmU1MjY0ZDFmNjgxMzI=--75112b7b966f610fe74e5dac52434a1eebd30040">
  <meta name="pjax-timeout" content="1000">
  <link rel="sudo-modal" href="/sessions/sudo_modal">
  <meta name="request-id" content="2A58:223D2:379CA9:608007:5968F92B" data-pjax-transient>
  

  <meta name="selected-link" value="repo_source" data-pjax-transient>

  <meta name="google-site-verification" content="KT5gs8h0wvaagLKAVWq8bbeNwnZZK1r1XQysX3xurLU">
<meta name="google-site-verification" content="ZzhVyEFwb7w3e0-uOTltm8Jsck2F5StVihD0exw2fsA">
    <meta name="google-analytics" content="UA-3769691-2">

<meta content="collector.githubapp.com" name="octolytics-host" /><meta content="github" name="octolytics-app-id" /><meta content="https://collector.githubapp.com/github-external/browser_event" name="octolytics-event-url" /><meta content="2A58:223D2:379CA9:608007:5968F92B" name="octolytics-dimension-request_id" /><meta content="iad" name="octolytics-dimension-region_edge" /><meta content="iad" name="octolytics-dimension-region_render" /><meta content="22482260" name="octolytics-actor-id" /><meta content="danfabrycky" name="octolytics-actor-login" /><meta content="c312f5fed5186562557a8e0b969d3af8f095127bb7d30e53236cd3232de0d525" name="octolytics-actor-hash" />
<meta content="/&lt;user-name&gt;/&lt;repo-name&gt;/blob/show" data-pjax-transient="true" name="analytics-location" />




  <meta class="js-ga-set" name="dimension1" content="Logged In">


  

      <meta name="hostname" content="github.com">
  <meta name="user-login" content="danfabrycky">

      <meta name="expected-hostname" content="github.com">
    <meta name="js-proxy-site-detection-payload" content="YmE3M2VlN2VlYTg1MTNlZWQ2OGM2MGM2ZmY5ODk5YzkwMTMxNWNhODg2Njc1ZTM5MDdiOTUwOTBkMGI1MmQ1OHx7InJlbW90ZV9hZGRyZXNzIjoiMTI4LjEzNS4xMDAuMTA5IiwicmVxdWVzdF9pZCI6IjJBNTg6MjIzRDI6Mzc5Q0E5OjYwODAwNzo1OTY4RjkyQiIsInRpbWVzdGFtcCI6MTUwMDA1MTc2NCwiaG9zdCI6ImdpdGh1Yi5jb20ifQ==">


  <meta name="html-safe-nonce" content="ade9a7c8ba287f81a235b9f8c97c3fae893de2b2">

  <meta http-equiv="x-pjax-version" content="175e789258843bb35d9544367c73edfe">
  

      <link href="https://github.com/ericagol/TTV_review/commits/Dan-on-JasonS.atom" rel="alternate" title="Recent Commits to TTV_review:Dan-on-JasonS" type="application/atom+xml">

  <meta name="description" content="TTV_review - Chapter for Springer&#39;s Exoplanet Handbook on TTV/TDV">
  <meta name="go-import" content="github.com/ericagol/TTV_review git https://github.com/ericagol/TTV_review.git">

  <meta content="243664" name="octolytics-dimension-user_id" /><meta content="ericagol" name="octolytics-dimension-user_login" /><meta content="68408270" name="octolytics-dimension-repository_id" /><meta content="ericagol/TTV_review" name="octolytics-dimension-repository_nwo" /><meta content="true" name="octolytics-dimension-repository_public" /><meta content="false" name="octolytics-dimension-repository_is_fork" /><meta content="68408270" name="octolytics-dimension-repository_network_root_id" /><meta content="ericagol/TTV_review" name="octolytics-dimension-repository_network_root_nwo" /><meta content="false" name="octolytics-dimension-repository_explore_github_marketplace_ci_cta_shown" />


    <link rel="canonical" href="https://github.com/ericagol/TTV_review/blob/Dan-on-JasonS/agol_fabrycky.tex" data-pjax-transient>


  <meta name="browser-stats-url" content="https://api.github.com/_private/browser/stats">

  <meta name="browser-errors-url" content="https://api.github.com/_private/browser/errors">

  <link rel="mask-icon" href="https://assets-cdn.github.com/pinned-octocat.svg" color="#000000">
  <link rel="icon" type="image/x-icon" href="https://assets-cdn.github.com/favicon.ico">

<meta name="theme-color" content="#1e2327">


  <meta name="u2f-support" content="true">

  </head>

  <body class="logged-in env-production emoji-size-boost page-blob">
    



  <div class="position-relative js-header-wrapper ">
    <a href="#start-of-content" tabindex="1" class="bg-black text-white p-3 show-on-focus js-skip-to-content">Skip to content</a>
    <div id="js-pjax-loader-bar" class="pjax-loader-bar"><div class="progress"></div></div>

    
    
    



        
<div class="header" role="banner">
  <div class="container clearfix">
    <a class="header-logo-invertocat" href="https://github.com/" data-hotkey="g d" aria-label="Homepage" data-ga-click="Header, go to dashboard, icon:logo">
  <svg aria-hidden="true" class="octicon octicon-mark-github" height="32" version="1.1" viewBox="0 0 16 16" width="32"><path fill-rule="evenodd" d="M8 0C3.58 0 0 3.58 0 8c0 3.54 2.29 6.53 5.47 7.59.4.07.55-.17.55-.38 0-.19-.01-.82-.01-1.49-2.01.37-2.53-.49-2.69-.94-.09-.23-.48-.94-.82-1.13-.28-.15-.68-.52-.01-.53.63-.01 1.08.58 1.23.82.72 1.21 1.87.87 2.33.66.07-.52.28-.87.51-1.07-1.78-.2-3.64-.89-3.64-3.95 0-.87.31-1.59.82-2.15-.08-.2-.36-1.02.08-2.12 0 0 .67-.21 2.2.82.64-.18 1.32-.27 2-.27.68 0 1.36.09 2 .27 1.53-1.04 2.2-.82 2.2-.82.44 1.1.16 1.92.08 2.12.51.56.82 1.27.82 2.15 0 3.07-1.87 3.75-3.65 3.95.29.25.54.73.54 1.48 0 1.07-.01 1.93-.01 2.2 0 .21.15.46.55.38A8.013 8.013 0 0 0 16 8c0-4.42-3.58-8-8-8z"/></svg>
</a>


        <div class="header-search scoped-search site-scoped-search js-site-search" role="search">
  <!-- '"` --><!-- </textarea></xmp> --></option></form><form accept-charset="UTF-8" action="/ericagol/TTV_review/search" class="js-site-search-form" data-scoped-search-url="/ericagol/TTV_review/search" data-unscoped-search-url="/search" method="get"><div style="margin:0;padding:0;display:inline"><input name="utf8" type="hidden" value="&#x2713;" /></div>
    <label class="form-control header-search-wrapper js-chromeless-input-container">
        <a href="/ericagol/TTV_review/blob/Dan-on-JasonS/agol_fabrycky.tex" class="header-search-scope no-underline">This repository</a>
      <input type="text"
        class="form-control header-search-input js-site-search-focus js-site-search-field is-clearable"
        data-hotkey="s"
        name="q"
        value=""
        placeholder="Search"
        aria-label="Search this repository"
        data-unscoped-placeholder="Search GitHub"
        data-scoped-placeholder="Search"
        autocapitalize="off">
        <input type="hidden" class="js-site-search-type-field" name="type" >
    </label>
</form></div>


      <ul class="header-nav float-left" role="navigation">
        <li class="header-nav-item">
          <a href="/pulls" aria-label="Pull requests you created" class="js-selected-navigation-item header-nav-link" data-ga-click="Header, click, Nav menu - item:pulls context:user" data-hotkey="g p" data-selected-links="/pulls /pulls/assigned /pulls/mentioned /pulls">
            Pull requests
</a>        </li>
        <li class="header-nav-item">
          <a href="/issues" aria-label="Issues you created" class="js-selected-navigation-item header-nav-link" data-ga-click="Header, click, Nav menu - item:issues context:user" data-hotkey="g i" data-selected-links="/issues /issues/assigned /issues/mentioned /issues">
            Issues
</a>        </li>
            <li class="header-nav-item">
              <a href="/marketplace" class="js-selected-navigation-item header-nav-link" data-ga-click="Header, click, Nav menu - item:marketplace context:user" data-selected-links=" /marketplace">
                Marketplace
</a>            </li>
          <li class="header-nav-item">
            <a class="header-nav-link" href="https://gist.github.com/" data-ga-click="Header, go to gist, text:gist">Gist</a>
          </li>
      </ul>

    
<ul class="header-nav user-nav float-right" id="user-links">
  <li class="header-nav-item">
    

  </li>

  <li class="header-nav-item dropdown js-menu-container">
    <a class="header-nav-link tooltipped tooltipped-s js-menu-target" href="/new"
       aria-label="Create new…"
       aria-expanded="false"
       aria-haspopup="true"
       data-ga-click="Header, create new, icon:add">
      <svg aria-hidden="true" class="octicon octicon-plus float-left" height="16" version="1.1" viewBox="0 0 12 16" width="12"><path fill-rule="evenodd" d="M12 9H7v5H5V9H0V7h5V2h2v5h5z"/></svg>
      <span class="dropdown-caret"></span>
    </a>

    <div class="dropdown-menu-content js-menu-content">
      <ul class="dropdown-menu dropdown-menu-sw">
        
<a class="dropdown-item" href="/new" data-ga-click="Header, create new repository">
  New repository
</a>

  <a class="dropdown-item" href="/new/import" data-ga-click="Header, import a repository">
    Import repository
  </a>

<a class="dropdown-item" href="https://gist.github.com/" data-ga-click="Header, create new gist">
  New gist
</a>

  <a class="dropdown-item" href="/organizations/new" data-ga-click="Header, create new organization">
    New organization
  </a>



  <div class="dropdown-divider"></div>
  <div class="dropdown-header">
    <span title="ericagol/TTV_review">This repository</span>
  </div>
    <a class="dropdown-item" href="/ericagol/TTV_review/issues/new" data-ga-click="Header, create new issue">
      New issue
    </a>

      </ul>
    </div>
  </li>

  <li class="header-nav-item dropdown js-menu-container">
    <a class="header-nav-link name tooltipped tooltipped-sw js-menu-target" href="/danfabrycky"
       aria-label="View profile and more"
       aria-expanded="false"
       aria-haspopup="true"
       data-ga-click="Header, show menu, icon:avatar">
      <img alt="@danfabrycky" class="avatar" src="https://avatars4.githubusercontent.com/u/22482260?v=4&amp;s=40" height="20" width="20">
      <span class="dropdown-caret"></span>
    </a>

    <div class="dropdown-menu-content js-menu-content">
      <div class="dropdown-menu dropdown-menu-sw">
        <div class="dropdown-header header-nav-current-user css-truncate">
          Signed in as <strong class="css-truncate-target">danfabrycky</strong>
        </div>

        <div class="dropdown-divider"></div>

        <a class="dropdown-item" href="/danfabrycky" data-ga-click="Header, go to profile, text:your profile">
          Your profile
        </a>
        <a class="dropdown-item" href="/danfabrycky?tab=stars" data-ga-click="Header, go to starred repos, text:your stars">
          Your stars
        </a>
        <a class="dropdown-item" href="/explore" data-ga-click="Header, go to explore, text:explore">
          Explore
        </a>
        <a class="dropdown-item" href="https://help.github.com" data-ga-click="Header, go to help, text:help">
          Help
        </a>

        <div class="dropdown-divider"></div>

        <a class="dropdown-item" href="/settings/profile" data-ga-click="Header, go to settings, icon:settings">
          Settings
        </a>

        <!-- '"` --><!-- </textarea></xmp> --></option></form><form accept-charset="UTF-8" action="/logout" class="logout-form" method="post"><div style="margin:0;padding:0;display:inline"><input name="utf8" type="hidden" value="&#x2713;" /><input name="authenticity_token" type="hidden" value="jL65mW/VGHBlM5lnfj3QUC05Da/QmPCOBn3tSY9g9RzaFbinyU+ZGVMddxvpVPtAJF3rdC7uGB7B9SuP6WmUAQ==" /></div>
          <button type="submit" class="dropdown-item dropdown-signout" data-ga-click="Header, sign out, icon:logout">
            Sign out
          </button>
</form>      </div>
    </div>
  </li>
</ul>


    <!-- '"` --><!-- </textarea></xmp> --></option></form><form accept-charset="UTF-8" action="/logout" class="sr-only right-0" method="post"><div style="margin:0;padding:0;display:inline"><input name="utf8" type="hidden" value="&#x2713;" /><input name="authenticity_token" type="hidden" value="qdoHgT1JuxMXsw1wFD0Q0sYTnMKgGFwycGXSmoaV0lj/cQa/m9M6eiGd4wyDVDvCz3d6GV5utKK37RRc4JyzRQ==" /></div>
      <button type="submit" class="dropdown-item dropdown-signout" data-ga-click="Header, sign out, icon:logout">
        Sign out
      </button>
</form>  </div>
</div>


      

  </div>

  <div id="start-of-content" class="show-on-focus"></div>

    <div id="js-flash-container">
</div>



  <div role="main">
        <div itemscope itemtype="http://schema.org/SoftwareSourceCode">
    <div id="js-repo-pjax-container" data-pjax-container>
      




    <div class="pagehead repohead instapaper_ignore readability-menu experiment-repo-nav">
      <div class="container repohead-details-container">

        <ul class="pagehead-actions">
  <li>
        <!-- '"` --><!-- </textarea></xmp> --></option></form><form accept-charset="UTF-8" action="/notifications/subscribe" class="js-social-container" data-autosubmit="true" data-remote="true" method="post"><div style="margin:0;padding:0;display:inline"><input name="utf8" type="hidden" value="&#x2713;" /><input name="authenticity_token" type="hidden" value="pu+J5Ek0304tUu4I8JlMAFLluhRBF6rTuRjG7+TyfDMGcf0V2+rZKmgwdxmOdRhqL14hfXl3iuHsrdjl5rrwfA==" /></div>      <input class="form-control" id="repository_id" name="repository_id" type="hidden" value="68408270" />

        <div class="select-menu js-menu-container js-select-menu">
          <a href="/ericagol/TTV_review/subscription"
            class="btn btn-sm btn-with-count select-menu-button js-menu-target"
            role="button"
            aria-haspopup="true"
            aria-expanded="false"
            aria-label="Toggle repository notifications menu"
            data-ga-click="Repository, click Watch settings, action:blob#show">
            <span class="js-select-button">
                <svg aria-hidden="true" class="octicon octicon-eye" height="16" version="1.1" viewBox="0 0 16 16" width="16"><path fill-rule="evenodd" d="M8.06 2C3 2 0 8 0 8s3 6 8.06 6C13 14 16 8 16 8s-3-6-7.94-6zM8 12c-2.2 0-4-1.78-4-4 0-2.2 1.8-4 4-4 2.22 0 4 1.8 4 4 0 2.22-1.78 4-4 4zm2-4c0 1.11-.89 2-2 2-1.11 0-2-.89-2-2 0-1.11.89-2 2-2 1.11 0 2 .89 2 2z"/></svg>
                Watch
            </span>
          </a>
            <a class="social-count js-social-count"
              href="/ericagol/TTV_review/watchers"
              aria-label="1 user is watching this repository">
              1
            </a>

        <div class="select-menu-modal-holder">
          <div class="select-menu-modal subscription-menu-modal js-menu-content">
            <div class="select-menu-header js-navigation-enable" tabindex="-1">
              <svg aria-label="Close" class="octicon octicon-x js-menu-close" height="16" role="img" version="1.1" viewBox="0 0 12 16" width="12"><path fill-rule="evenodd" d="M7.48 8l3.75 3.75-1.48 1.48L6 9.48l-3.75 3.75-1.48-1.48L4.52 8 .77 4.25l1.48-1.48L6 6.52l3.75-3.75 1.48 1.48z"/></svg>
              <span class="select-menu-title">Notifications</span>
            </div>

              <div class="select-menu-list js-navigation-container" role="menu">

                <div class="select-menu-item js-navigation-item selected" role="menuitem" tabindex="0">
                  <svg aria-hidden="true" class="octicon octicon-check select-menu-item-icon" height="16" version="1.1" viewBox="0 0 12 16" width="12"><path fill-rule="evenodd" d="M12 5l-8 8-4-4 1.5-1.5L4 10l6.5-6.5z"/></svg>
                  <div class="select-menu-item-text">
                    <input checked="checked" id="do_included" name="do" type="radio" value="included" />
                    <span class="select-menu-item-heading">Not watching</span>
                    <span class="description">Be notified when participating or @mentioned.</span>
                    <span class="js-select-button-text hidden-select-button-text">
                      <svg aria-hidden="true" class="octicon octicon-eye" height="16" version="1.1" viewBox="0 0 16 16" width="16"><path fill-rule="evenodd" d="M8.06 2C3 2 0 8 0 8s3 6 8.06 6C13 14 16 8 16 8s-3-6-7.94-6zM8 12c-2.2 0-4-1.78-4-4 0-2.2 1.8-4 4-4 2.22 0 4 1.8 4 4 0 2.22-1.78 4-4 4zm2-4c0 1.11-.89 2-2 2-1.11 0-2-.89-2-2 0-1.11.89-2 2-2 1.11 0 2 .89 2 2z"/></svg>
                      Watch
                    </span>
                  </div>
                </div>

                <div class="select-menu-item js-navigation-item " role="menuitem" tabindex="0">
                  <svg aria-hidden="true" class="octicon octicon-check select-menu-item-icon" height="16" version="1.1" viewBox="0 0 12 16" width="12"><path fill-rule="evenodd" d="M12 5l-8 8-4-4 1.5-1.5L4 10l6.5-6.5z"/></svg>
                  <div class="select-menu-item-text">
                    <input id="do_subscribed" name="do" type="radio" value="subscribed" />
                    <span class="select-menu-item-heading">Watching</span>
                    <span class="description">Be notified of all conversations.</span>
                    <span class="js-select-button-text hidden-select-button-text">
                      <svg aria-hidden="true" class="octicon octicon-eye" height="16" version="1.1" viewBox="0 0 16 16" width="16"><path fill-rule="evenodd" d="M8.06 2C3 2 0 8 0 8s3 6 8.06 6C13 14 16 8 16 8s-3-6-7.94-6zM8 12c-2.2 0-4-1.78-4-4 0-2.2 1.8-4 4-4 2.22 0 4 1.8 4 4 0 2.22-1.78 4-4 4zm2-4c0 1.11-.89 2-2 2-1.11 0-2-.89-2-2 0-1.11.89-2 2-2 1.11 0 2 .89 2 2z"/></svg>
                        Unwatch
                    </span>
                  </div>
                </div>

                <div class="select-menu-item js-navigation-item " role="menuitem" tabindex="0">
                  <svg aria-hidden="true" class="octicon octicon-check select-menu-item-icon" height="16" version="1.1" viewBox="0 0 12 16" width="12"><path fill-rule="evenodd" d="M12 5l-8 8-4-4 1.5-1.5L4 10l6.5-6.5z"/></svg>
                  <div class="select-menu-item-text">
                    <input id="do_ignore" name="do" type="radio" value="ignore" />
                    <span class="select-menu-item-heading">Ignoring</span>
                    <span class="description">Never be notified.</span>
                    <span class="js-select-button-text hidden-select-button-text">
                      <svg aria-hidden="true" class="octicon octicon-mute" height="16" version="1.1" viewBox="0 0 16 16" width="16"><path fill-rule="evenodd" d="M8 2.81v10.38c0 .67-.81 1-1.28.53L3 10H1c-.55 0-1-.45-1-1V7c0-.55.45-1 1-1h2l3.72-3.72C7.19 1.81 8 2.14 8 2.81zm7.53 3.22l-1.06-1.06-1.97 1.97-1.97-1.97-1.06 1.06L11.44 8 9.47 9.97l1.06 1.06 1.97-1.97 1.97 1.97 1.06-1.06L13.56 8l1.97-1.97z"/></svg>
                        Stop ignoring
                    </span>
                  </div>
                </div>

              </div>

            </div>
          </div>
        </div>
</form>
  </li>

  <li>
    
  <div class="js-toggler-container js-social-container starring-container ">
    <!-- '"` --><!-- </textarea></xmp> --></option></form><form accept-charset="UTF-8" action="/ericagol/TTV_review/unstar" class="starred" data-remote="true" method="post"><div style="margin:0;padding:0;display:inline"><input name="utf8" type="hidden" value="&#x2713;" /><input name="authenticity_token" type="hidden" value="8VXZTwkCRhjHgJDoHs3VNW/4UIQ9IvwXUZy0JqB5quE1xvi3eaDis5xxbf4f7XNb23lq02Q7n9x6GzcvKcnKrQ==" /></div>
      <button
        type="submit"
        class="btn btn-sm btn-with-count js-toggler-target"
        aria-label="Unstar this repository" title="Unstar ericagol/TTV_review"
        data-ga-click="Repository, click unstar button, action:blob#show; text:Unstar">
        <svg aria-hidden="true" class="octicon octicon-star" height="16" version="1.1" viewBox="0 0 14 16" width="14"><path fill-rule="evenodd" d="M14 6l-4.9-.64L7 1 4.9 5.36 0 6l3.6 3.26L2.67 14 7 11.67 11.33 14l-.93-4.74z"/></svg>
        Unstar
      </button>
        <a class="social-count js-social-count" href="/ericagol/TTV_review/stargazers"
           aria-label="1 user starred this repository">
          1
        </a>
</form>
    <!-- '"` --><!-- </textarea></xmp> --></option></form><form accept-charset="UTF-8" action="/ericagol/TTV_review/star" class="unstarred" data-remote="true" method="post"><div style="margin:0;padding:0;display:inline"><input name="utf8" type="hidden" value="&#x2713;" /><input name="authenticity_token" type="hidden" value="xNF8lPZ7+0Y4Qm86kNBmObuAXcowgDirg9iMpZvTeeL2SCxuvXhOyk9+6NusUvQMXRSTzSaMsBm59kmyg01/fQ==" /></div>
      <button
        type="submit"
        class="btn btn-sm btn-with-count js-toggler-target"
        aria-label="Star this repository" title="Star ericagol/TTV_review"
        data-ga-click="Repository, click star button, action:blob#show; text:Star">
        <svg aria-hidden="true" class="octicon octicon-star" height="16" version="1.1" viewBox="0 0 14 16" width="14"><path fill-rule="evenodd" d="M14 6l-4.9-.64L7 1 4.9 5.36 0 6l3.6 3.26L2.67 14 7 11.67 11.33 14l-.93-4.74z"/></svg>
        Star
      </button>
        <a class="social-count js-social-count" href="/ericagol/TTV_review/stargazers"
           aria-label="1 user starred this repository">
          1
        </a>
</form>  </div>

  </li>

  <li>
          <!-- '"` --><!-- </textarea></xmp> --></option></form><form accept-charset="UTF-8" action="/ericagol/TTV_review/fork" class="btn-with-count" method="post"><div style="margin:0;padding:0;display:inline"><input name="utf8" type="hidden" value="&#x2713;" /><input name="authenticity_token" type="hidden" value="jjRWYnshzXu4+meDcYroOntuUHPQfh4uz38upTEuaO67PMY/9qaTw+QQMFvkUPTGKIE7X3wiY0xzA76TUxV7Og==" /></div>
            <button
                type="submit"
                class="btn btn-sm btn-with-count"
                data-ga-click="Repository, show fork modal, action:blob#show; text:Fork"
                title="Fork your own copy of ericagol/TTV_review to your account"
                aria-label="Fork your own copy of ericagol/TTV_review to your account">
              <svg aria-hidden="true" class="octicon octicon-repo-forked" height="16" version="1.1" viewBox="0 0 10 16" width="10"><path fill-rule="evenodd" d="M8 1a1.993 1.993 0 0 0-1 3.72V6L5 8 3 6V4.72A1.993 1.993 0 0 0 2 1a1.993 1.993 0 0 0-1 3.72V6.5l3 3v1.78A1.993 1.993 0 0 0 5 15a1.993 1.993 0 0 0 1-3.72V9.5l3-3V4.72A1.993 1.993 0 0 0 8 1zM2 4.2C1.34 4.2.8 3.65.8 3c0-.65.55-1.2 1.2-1.2.65 0 1.2.55 1.2 1.2 0 .65-.55 1.2-1.2 1.2zm3 10c-.66 0-1.2-.55-1.2-1.2 0-.65.55-1.2 1.2-1.2.65 0 1.2.55 1.2 1.2 0 .65-.55 1.2-1.2 1.2zm3-10c-.66 0-1.2-.55-1.2-1.2 0-.65.55-1.2 1.2-1.2.65 0 1.2.55 1.2 1.2 0 .65-.55 1.2-1.2 1.2z"/></svg>
              Fork
            </button>
</form>
    <a href="/ericagol/TTV_review/network" class="social-count"
       aria-label="1 user forked this repository">
      1
    </a>
  </li>
</ul>

        <h1 class="public ">
  <svg aria-hidden="true" class="octicon octicon-repo" height="16" version="1.1" viewBox="0 0 12 16" width="12"><path fill-rule="evenodd" d="M4 9H3V8h1v1zm0-3H3v1h1V6zm0-2H3v1h1V4zm0-2H3v1h1V2zm8-1v12c0 .55-.45 1-1 1H6v2l-1.5-1.5L3 16v-2H1c-.55 0-1-.45-1-1V1c0-.55.45-1 1-1h10c.55 0 1 .45 1 1zm-1 10H1v2h2v-1h3v1h5v-2zm0-10H2v9h9V1z"/></svg>
  <span class="author" itemprop="author"><a href="/ericagol" class="url fn" rel="author">ericagol</a></span><!--
--><span class="path-divider">/</span><!--
--><strong itemprop="name"><a href="/ericagol/TTV_review" data-pjax="#js-repo-pjax-container">TTV_review</a></strong>

</h1>

      </div>
      <div class="container">
        
<nav class="reponav js-repo-nav js-sidenav-container-pjax"
     itemscope
     itemtype="http://schema.org/BreadcrumbList"
     role="navigation"
     data-pjax="#js-repo-pjax-container">

  <span itemscope itemtype="http://schema.org/ListItem" itemprop="itemListElement">
    <a href="/ericagol/TTV_review/tree/Dan-on-JasonS" class="js-selected-navigation-item selected reponav-item" data-hotkey="g c" data-selected-links="repo_source repo_downloads repo_commits repo_releases repo_tags repo_branches /ericagol/TTV_review/tree/Dan-on-JasonS" itemprop="url">
      <svg aria-hidden="true" class="octicon octicon-code" height="16" version="1.1" viewBox="0 0 14 16" width="14"><path fill-rule="evenodd" d="M9.5 3L8 4.5 11.5 8 8 11.5 9.5 13 14 8 9.5 3zm-5 0L0 8l4.5 5L6 11.5 2.5 8 6 4.5 4.5 3z"/></svg>
      <span itemprop="name">Code</span>
      <meta itemprop="position" content="1">
</a>  </span>

    <span itemscope itemtype="http://schema.org/ListItem" itemprop="itemListElement">
      <a href="/ericagol/TTV_review/issues" class="js-selected-navigation-item reponav-item" data-hotkey="g i" data-selected-links="repo_issues repo_labels repo_milestones /ericagol/TTV_review/issues" itemprop="url">
        <svg aria-hidden="true" class="octicon octicon-issue-opened" height="16" version="1.1" viewBox="0 0 14 16" width="14"><path fill-rule="evenodd" d="M7 2.3c3.14 0 5.7 2.56 5.7 5.7s-2.56 5.7-5.7 5.7A5.71 5.71 0 0 1 1.3 8c0-3.14 2.56-5.7 5.7-5.7zM7 1C3.14 1 0 4.14 0 8s3.14 7 7 7 7-3.14 7-7-3.14-7-7-7zm1 3H6v5h2V4zm0 6H6v2h2v-2z"/></svg>
        <span itemprop="name">Issues</span>
        <span class="Counter">1</span>
        <meta itemprop="position" content="2">
</a>    </span>

  <span itemscope itemtype="http://schema.org/ListItem" itemprop="itemListElement">
    <a href="/ericagol/TTV_review/pulls" class="js-selected-navigation-item reponav-item" data-hotkey="g p" data-selected-links="repo_pulls /ericagol/TTV_review/pulls" itemprop="url">
      <svg aria-hidden="true" class="octicon octicon-git-pull-request" height="16" version="1.1" viewBox="0 0 12 16" width="12"><path fill-rule="evenodd" d="M11 11.28V5c-.03-.78-.34-1.47-.94-2.06C9.46 2.35 8.78 2.03 8 2H7V0L4 3l3 3V4h1c.27.02.48.11.69.31.21.2.3.42.31.69v6.28A1.993 1.993 0 0 0 10 15a1.993 1.993 0 0 0 1-3.72zm-1 2.92c-.66 0-1.2-.55-1.2-1.2 0-.65.55-1.2 1.2-1.2.65 0 1.2.55 1.2 1.2 0 .65-.55 1.2-1.2 1.2zM4 3c0-1.11-.89-2-2-2a1.993 1.993 0 0 0-1 3.72v6.56A1.993 1.993 0 0 0 2 15a1.993 1.993 0 0 0 1-3.72V4.72c.59-.34 1-.98 1-1.72zm-.8 10c0 .66-.55 1.2-1.2 1.2-.65 0-1.2-.55-1.2-1.2 0-.65.55-1.2 1.2-1.2.65 0 1.2.55 1.2 1.2zM2 4.2C1.34 4.2.8 3.65.8 3c0-.65.55-1.2 1.2-1.2.65 0 1.2.55 1.2 1.2 0 .65-.55 1.2-1.2 1.2z"/></svg>
      <span itemprop="name">Pull requests</span>
      <span class="Counter">0</span>
      <meta itemprop="position" content="3">
</a>  </span>

    <a href="/ericagol/TTV_review/projects" class="js-selected-navigation-item reponav-item" data-selected-links="repo_projects new_repo_project repo_project /ericagol/TTV_review/projects">
      <svg aria-hidden="true" class="octicon octicon-project" height="16" version="1.1" viewBox="0 0 15 16" width="15"><path fill-rule="evenodd" d="M10 12h3V2h-3v10zm-4-2h3V2H6v8zm-4 4h3V2H2v12zm-1 1h13V1H1v14zM14 0H1a1 1 0 0 0-1 1v14a1 1 0 0 0 1 1h13a1 1 0 0 0 1-1V1a1 1 0 0 0-1-1z"/></svg>
      Projects
      <span class="Counter" >0</span>
</a>
    <a href="/ericagol/TTV_review/wiki" class="js-selected-navigation-item reponav-item" data-hotkey="g w" data-selected-links="repo_wiki /ericagol/TTV_review/wiki">
      <svg aria-hidden="true" class="octicon octicon-book" height="16" version="1.1" viewBox="0 0 16 16" width="16"><path fill-rule="evenodd" d="M3 5h4v1H3V5zm0 3h4V7H3v1zm0 2h4V9H3v1zm11-5h-4v1h4V5zm0 2h-4v1h4V7zm0 2h-4v1h4V9zm2-6v9c0 .55-.45 1-1 1H9.5l-1 1-1-1H2c-.55 0-1-.45-1-1V3c0-.55.45-1 1-1h5.5l1 1 1-1H15c.55 0 1 .45 1 1zm-8 .5L7.5 3H2v9h6V3.5zm7-.5H9.5l-.5.5V12h6V3z"/></svg>
      Wiki
</a>

    <div class="reponav-dropdown js-menu-container">
      <button type="button" class="btn-link reponav-item reponav-dropdown js-menu-target " data-no-toggle aria-expanded="false" aria-haspopup="true">
        Insights
        <svg aria-hidden="true" class="octicon octicon-triangle-down v-align-middle text-gray" height="11" version="1.1" viewBox="0 0 12 16" width="8"><path fill-rule="evenodd" d="M0 5l6 6 6-6z"/></svg>
      </button>
      <div class="dropdown-menu-content js-menu-content">
        <div class="dropdown-menu dropdown-menu-sw">
          <a class="dropdown-item" href="/ericagol/TTV_review/pulse" data-skip-pjax>
            <svg aria-hidden="true" class="octicon octicon-pulse" height="16" version="1.1" viewBox="0 0 14 16" width="14"><path fill-rule="evenodd" d="M11.5 8L8.8 5.4 6.6 8.5 5.5 1.6 2.38 8H0v2h3.6l.9-1.8.9 5.4L9 8.5l1.6 1.5H14V8z"/></svg>
            Pulse
          </a>
          <a class="dropdown-item" href="/ericagol/TTV_review/graphs" data-skip-pjax>
            <svg aria-hidden="true" class="octicon octicon-graph" height="16" version="1.1" viewBox="0 0 16 16" width="16"><path fill-rule="evenodd" d="M16 14v1H0V0h1v14h15zM5 13H3V8h2v5zm4 0H7V3h2v10zm4 0h-2V6h2v7z"/></svg>
            Graphs
          </a>
        </div>
      </div>
    </div>
</nav>

      </div>
    </div>

<div class="container new-discussion-timeline experiment-repo-nav">
  <div class="repository-content">

    
  <a href="/ericagol/TTV_review/blob/352fe85c4608ac31b40dcf76350c9049dc878974/agol_fabrycky.tex" class="d-none js-permalink-shortcut" data-hotkey="y">Permalink</a>

  <!-- blob contrib key: blob_contributors:v21:77e60b2f86286b24b4e76be4ad571001 -->

  <div class="file-navigation js-zeroclipboard-container">
    
<div class="select-menu branch-select-menu js-menu-container js-select-menu float-left">
  <button class=" btn btn-sm select-menu-button js-menu-target css-truncate" data-hotkey="w"
    
    type="button" aria-label="Switch branches or tags" aria-expanded="false" aria-haspopup="true">
      <i>Branch:</i>
      <span class="js-select-button css-truncate-target">Dan-on-JasonS</span>
  </button>

  <div class="select-menu-modal-holder js-menu-content js-navigation-container" data-pjax>

    <div class="select-menu-modal">
      <div class="select-menu-header">
        <svg aria-label="Close" class="octicon octicon-x js-menu-close" height="16" role="img" version="1.1" viewBox="0 0 12 16" width="12"><path fill-rule="evenodd" d="M7.48 8l3.75 3.75-1.48 1.48L6 9.48l-3.75 3.75-1.48-1.48L4.52 8 .77 4.25l1.48-1.48L6 6.52l3.75-3.75 1.48 1.48z"/></svg>
        <span class="select-menu-title">Switch branches/tags</span>
      </div>

      <div class="select-menu-filters">
        <div class="select-menu-text-filter">
          <input type="text" aria-label="Find or create a branch…" id="context-commitish-filter-field" class="form-control js-filterable-field js-navigation-enable" placeholder="Find or create a branch…">
        </div>
        <div class="select-menu-tabs">
          <ul>
            <li class="select-menu-tab">
              <a href="#" data-tab-filter="branches" data-filter-placeholder="Find or create a branch…" class="js-select-menu-tab" role="tab">Branches</a>
            </li>
            <li class="select-menu-tab">
              <a href="#" data-tab-filter="tags" data-filter-placeholder="Find a tag…" class="js-select-menu-tab" role="tab">Tags</a>
            </li>
          </ul>
        </div>
      </div>

      <div class="select-menu-list select-menu-tab-bucket js-select-menu-tab-bucket" data-tab-filter="branches" role="menu">

        <div data-filterable-for="context-commitish-filter-field" data-filterable-type="substring">


            <a class="select-menu-item js-navigation-item js-navigation-open "
               href="/ericagol/TTV_review/blob/Dan-addfrontback/agol_fabrycky.tex"
               data-name="Dan-addfrontback"
               data-skip-pjax="true"
               rel="nofollow">
              <svg aria-hidden="true" class="octicon octicon-check select-menu-item-icon" height="16" version="1.1" viewBox="0 0 12 16" width="12"><path fill-rule="evenodd" d="M12 5l-8 8-4-4 1.5-1.5L4 10l6.5-6.5z"/></svg>
              <span class="select-menu-item-text css-truncate-target js-select-menu-filter-text">
                Dan-addfrontback
              </span>
            </a>
            <a class="select-menu-item js-navigation-item js-navigation-open "
               href="/ericagol/TTV_review/blob/Dan-edits-due-Oct-5/agol_fabrycky.tex"
               data-name="Dan-edits-due-Oct-5"
               data-skip-pjax="true"
               rel="nofollow">
              <svg aria-hidden="true" class="octicon octicon-check select-menu-item-icon" height="16" version="1.1" viewBox="0 0 12 16" width="12"><path fill-rule="evenodd" d="M12 5l-8 8-4-4 1.5-1.5L4 10l6.5-6.5z"/></svg>
              <span class="select-menu-item-text css-truncate-target js-select-menu-filter-text">
                Dan-edits-due-Oct-5
              </span>
            </a>
            <a class="select-menu-item js-navigation-item js-navigation-open "
               href="/ericagol/TTV_review/blob/Dan-edits-lastlap/agol_fabrycky.tex"
               data-name="Dan-edits-lastlap"
               data-skip-pjax="true"
               rel="nofollow">
              <svg aria-hidden="true" class="octicon octicon-check select-menu-item-icon" height="16" version="1.1" viewBox="0 0 12 16" width="12"><path fill-rule="evenodd" d="M12 5l-8 8-4-4 1.5-1.5L4 10l6.5-6.5z"/></svg>
              <span class="select-menu-item-text css-truncate-target js-select-menu-filter-text">
                Dan-edits-lastlap
              </span>
            </a>
            <a class="select-menu-item js-navigation-item js-navigation-open "
               href="/ericagol/TTV_review/blob/Dan-edits-smoothing/agol_fabrycky.tex"
               data-name="Dan-edits-smoothing"
               data-skip-pjax="true"
               rel="nofollow">
              <svg aria-hidden="true" class="octicon octicon-check select-menu-item-icon" height="16" version="1.1" viewBox="0 0 12 16" width="12"><path fill-rule="evenodd" d="M12 5l-8 8-4-4 1.5-1.5L4 10l6.5-6.5z"/></svg>
              <span class="select-menu-item-text css-truncate-target js-select-menu-filter-text">
                Dan-edits-smoothing
              </span>
            </a>
            <a class="select-menu-item js-navigation-item js-navigation-open selected"
               href="/ericagol/TTV_review/blob/Dan-on-JasonS/agol_fabrycky.tex"
               data-name="Dan-on-JasonS"
               data-skip-pjax="true"
               rel="nofollow">
              <svg aria-hidden="true" class="octicon octicon-check select-menu-item-icon" height="16" version="1.1" viewBox="0 0 12 16" width="12"><path fill-rule="evenodd" d="M12 5l-8 8-4-4 1.5-1.5L4 10l6.5-6.5z"/></svg>
              <span class="select-menu-item-text css-truncate-target js-select-menu-filter-text">
                Dan-on-JasonS
              </span>
            </a>
            <a class="select-menu-item js-navigation-item js-navigation-open "
               href="/ericagol/TTV_review/blob/master/agol_fabrycky.tex"
               data-name="master"
               data-skip-pjax="true"
               rel="nofollow">
              <svg aria-hidden="true" class="octicon octicon-check select-menu-item-icon" height="16" version="1.1" viewBox="0 0 12 16" width="12"><path fill-rule="evenodd" d="M12 5l-8 8-4-4 1.5-1.5L4 10l6.5-6.5z"/></svg>
              <span class="select-menu-item-text css-truncate-target js-select-menu-filter-text">
                master
              </span>
            </a>
        </div>

          <!-- '"` --><!-- </textarea></xmp> --></option></form><form accept-charset="UTF-8" action="/ericagol/TTV_review/branches" class="js-create-branch select-menu-item select-menu-new-item-form js-navigation-item js-new-item-form" method="post"><div style="margin:0;padding:0;display:inline"><input name="utf8" type="hidden" value="&#x2713;" /><input name="authenticity_token" type="hidden" value="DzQc/G8EwFxllrANJDr2tSHDFOfSVT5YGjxhVCNyrceFuVd5XZzbNSUOBgCYmIayv10aku5GyUxGCxcwecDtyQ==" /></div>
          <svg aria-hidden="true" class="octicon octicon-git-branch select-menu-item-icon" height="16" version="1.1" viewBox="0 0 10 16" width="10"><path fill-rule="evenodd" d="M10 5c0-1.11-.89-2-2-2a1.993 1.993 0 0 0-1 3.72v.3c-.02.52-.23.98-.63 1.38-.4.4-.86.61-1.38.63-.83.02-1.48.16-2 .45V4.72a1.993 1.993 0 0 0-1-3.72C.88 1 0 1.89 0 3a2 2 0 0 0 1 1.72v6.56c-.59.35-1 .99-1 1.72 0 1.11.89 2 2 2 1.11 0 2-.89 2-2 0-.53-.2-1-.53-1.36.09-.06.48-.41.59-.47.25-.11.56-.17.94-.17 1.05-.05 1.95-.45 2.75-1.25S8.95 7.77 9 6.73h-.02C9.59 6.37 10 5.73 10 5zM2 1.8c.66 0 1.2.55 1.2 1.2 0 .65-.55 1.2-1.2 1.2C1.35 4.2.8 3.65.8 3c0-.65.55-1.2 1.2-1.2zm0 12.41c-.66 0-1.2-.55-1.2-1.2 0-.65.55-1.2 1.2-1.2.65 0 1.2.55 1.2 1.2 0 .65-.55 1.2-1.2 1.2zm6-8c-.66 0-1.2-.55-1.2-1.2 0-.65.55-1.2 1.2-1.2.65 0 1.2.55 1.2 1.2 0 .65-.55 1.2-1.2 1.2z"/></svg>
            <div class="select-menu-item-text">
              <span class="select-menu-item-heading">Create branch: <span class="js-new-item-name"></span></span>
              <span class="description">from ‘Dan-on-JasonS’</span>
            </div>
            <input type="hidden" name="name" id="name" class="js-new-item-value">
            <input type="hidden" name="branch" id="branch" value="Dan-on-JasonS">
            <input type="hidden" name="path" id="path" value="agol_fabrycky.tex">
</form>
      </div>

      <div class="select-menu-list select-menu-tab-bucket js-select-menu-tab-bucket" data-tab-filter="tags">
        <div data-filterable-for="context-commitish-filter-field" data-filterable-type="substring">


        </div>

        <div class="select-menu-no-results">Nothing to show</div>
      </div>

    </div>
  </div>
</div>

    <div class="BtnGroup float-right">
      <a href="/ericagol/TTV_review/find/Dan-on-JasonS"
            class="js-pjax-capture-input btn btn-sm BtnGroup-item"
            data-pjax
            data-hotkey="t">
        Find file
      </a>
      <button aria-label="Copy file path to clipboard" class="js-zeroclipboard btn btn-sm BtnGroup-item tooltipped tooltipped-s" data-copied-hint="Copied!" type="button">Copy path</button>
    </div>
    <div class="breadcrumb js-zeroclipboard-target">
      <span class="repo-root js-repo-root"><span class="js-path-segment"><a href="/ericagol/TTV_review/tree/Dan-on-JasonS"><span>TTV_review</span></a></span></span><span class="separator">/</span><strong class="final-path">agol_fabrycky.tex</strong>
    </div>
  </div>


  
  <div class="commit-tease">
      <span class="float-right">
        <a class="commit-tease-sha" href="/ericagol/TTV_review/commit/352fe85c4608ac31b40dcf76350c9049dc878974" data-pjax>
          352fe85
        </a>
        <relative-time datetime="2017-07-01T06:18:33Z">Jul 1, 2017</relative-time>
      </span>
      <div>
        <img alt="@ericagol" class="avatar" height="20" src="https://avatars5.githubusercontent.com/u/243664?v=3&amp;s=40" width="20" />
        <a href="/ericagol" class="user-mention" rel="author">ericagol</a>
          <a href="/ericagol/TTV_review/commit/352fe85c4608ac31b40dcf76350c9049dc878974" class="message" data-pjax="true" title="Corrections thanks to Jack L.; added acks">Corrections thanks to Jack L.; added acks</a>
      </div>

    <div class="commit-tease-contributors">
      <button type="button" class="btn-link muted-link contributors-toggle" data-facebox="#blob_contributors_box">
        <strong>2</strong>
         contributors
      </button>
          <a class="avatar-link tooltipped tooltipped-s" aria-label="ericagol" href="/ericagol/TTV_review/commits/master/agol_fabrycky.tex?author=ericagol"><img alt="@ericagol" class="avatar" height="20" src="https://avatars5.githubusercontent.com/u/243664?v=3&amp;s=40" width="20" /> </a>
    <a class="avatar-link tooltipped tooltipped-s" aria-label="danfabrycky" href="/ericagol/TTV_review/commits/master/agol_fabrycky.tex?author=danfabrycky"><img alt="@danfabrycky" class="avatar" height="20" src="https://avatars4.githubusercontent.com/u/22482260?v=4&amp;s=40" width="20" /> </a>


    </div>

    <div id="blob_contributors_box" style="display:none">
      <h2 class="facebox-header" data-facebox-id="facebox-header">Users who have contributed to this file</h2>
      <ul class="facebox-user-list" data-facebox-id="facebox-description">
          <li class="facebox-user-list-item">
            <img alt="@ericagol" height="24" src="https://avatars7.githubusercontent.com/u/243664?v=3&amp;s=48" width="24" />
            <a href="/ericagol">ericagol</a>
          </li>
          <li class="facebox-user-list-item">
            <img alt="@danfabrycky" height="24" src="https://avatars6.githubusercontent.com/u/22482260?v=4&amp;s=48" width="24" />
            <a href="/danfabrycky">danfabrycky</a>
          </li>
      </ul>
    </div>
  </div>

  <div class="file">
    <div class="file-header">
  <div class="file-actions">

    <div class="BtnGroup">
      <a href="/ericagol/TTV_review/raw/Dan-on-JasonS/agol_fabrycky.tex" class="btn btn-sm BtnGroup-item" id="raw-url">Raw</a>
        <a href="/ericagol/TTV_review/blame/Dan-on-JasonS/agol_fabrycky.tex" class="btn btn-sm js-update-url-with-hash BtnGroup-item" data-hotkey="b">Blame</a>
      <a href="/ericagol/TTV_review/commits/Dan-on-JasonS/agol_fabrycky.tex" class="btn btn-sm BtnGroup-item" rel="nofollow">History</a>
    </div>

        <a class="btn-octicon tooltipped tooltipped-nw"
           href="https://desktop.github.com"
           aria-label="Open this file in GitHub Desktop"
           data-ga-click="Repository, open with desktop, type:mac">
            <svg aria-hidden="true" class="octicon octicon-device-desktop" height="16" version="1.1" viewBox="0 0 16 16" width="16"><path fill-rule="evenodd" d="M15 2H1c-.55 0-1 .45-1 1v9c0 .55.45 1 1 1h5.34c-.25.61-.86 1.39-2.34 2h8c-1.48-.61-2.09-1.39-2.34-2H15c.55 0 1-.45 1-1V3c0-.55-.45-1-1-1zm0 9H1V3h14v8z"/></svg>
        </a>

        <!-- '"` --><!-- </textarea></xmp> --></option></form><form accept-charset="UTF-8" action="/ericagol/TTV_review/edit/Dan-on-JasonS/agol_fabrycky.tex" class="inline-form js-update-url-with-hash" method="post"><div style="margin:0;padding:0;display:inline"><input name="utf8" type="hidden" value="&#x2713;" /><input name="authenticity_token" type="hidden" value="wDqddxfO6kukdcCmpHnNfTD5BSdKAIqjxZg0IlXW9celmmandRW9JSc40PNUneFYwDczm2ox4CWCB/+iq8gX4w==" /></div>
          <button class="btn-octicon tooltipped tooltipped-nw" type="submit"
            aria-label="Edit this file" data-hotkey="e" data-disable-with>
            <svg aria-hidden="true" class="octicon octicon-pencil" height="16" version="1.1" viewBox="0 0 14 16" width="14"><path fill-rule="evenodd" d="M0 12v3h3l8-8-3-3-8 8zm3 2H1v-2h1v1h1v1zm10.3-9.3L12 6 9 3l1.3-1.3a.996.996 0 0 1 1.41 0l1.59 1.59c.39.39.39 1.02 0 1.41z"/></svg>
          </button>
</form>        <!-- '"` --><!-- </textarea></xmp> --></option></form><form accept-charset="UTF-8" action="/ericagol/TTV_review/delete/Dan-on-JasonS/agol_fabrycky.tex" class="inline-form" method="post"><div style="margin:0;padding:0;display:inline"><input name="utf8" type="hidden" value="&#x2713;" /><input name="authenticity_token" type="hidden" value="uWtpX4oYVzHKCUCmSSPkUhjrePwrAbsu+Cp8+oxyXscOv+R6AoWaJ2XG8cJU80ItfqNR7jubq8+QSOLgZsyYHg==" /></div>
          <button class="btn-octicon btn-octicon-danger tooltipped tooltipped-nw" type="submit"
            aria-label="Delete this file" data-disable-with>
            <svg aria-hidden="true" class="octicon octicon-trashcan" height="16" version="1.1" viewBox="0 0 12 16" width="12"><path fill-rule="evenodd" d="M11 2H9c0-.55-.45-1-1-1H5c-.55 0-1 .45-1 1H2c-.55 0-1 .45-1 1v1c0 .55.45 1 1 1v9c0 .55.45 1 1 1h7c.55 0 1-.45 1-1V5c.55 0 1-.45 1-1V3c0-.55-.45-1-1-1zm-1 12H3V5h1v8h1V5h1v8h1V5h1v8h1V5h1v9zm1-10H2V3h9v1z"/></svg>
          </button>
</form>  </div>

  <div class="file-info">
      455 lines (374 sloc)
      <span class="file-info-divider"></span>
    45.1 KB
  </div>
</div>

    

  <div itemprop="text" class="blob-wrapper data type-tex">
      <table class="highlight tab-size js-file-line-container" data-tab-size="8">
      <tr>
        <td id="L1" class="blob-num js-line-number" data-line-number="1"></td>
        <td id="LC1" class="blob-code blob-code-inner js-file-line"><span class="pl-c"><span class="pl-c">%</span>%%%%%%%%%%%%%%%%%%% author.tex %%%%%%%%%%%%%%%%%%%%%%%%%%%%%%%%%%%</span></td>
      </tr>
      <tr>
        <td id="L2" class="blob-num js-line-number" data-line-number="2"></td>
        <td id="LC2" class="blob-code blob-code-inner js-file-line"><span class="pl-c"><span class="pl-c">%</span></span></td>
      </tr>
      <tr>
        <td id="L3" class="blob-num js-line-number" data-line-number="3"></td>
        <td id="LC3" class="blob-code blob-code-inner js-file-line"><span class="pl-c"><span class="pl-c">%</span> template for chapters to the Handbook of Exoplanets</span></td>
      </tr>
      <tr>
        <td id="L4" class="blob-num js-line-number" data-line-number="4"></td>
        <td id="LC4" class="blob-code blob-code-inner js-file-line"><span class="pl-c"><span class="pl-c">%</span> modified by H. Deeg from the &#39;template.tex&#39; provided by Springer for the svmult.cls class</span></td>
      </tr>
      <tr>
        <td id="L5" class="blob-num js-line-number" data-line-number="5"></td>
        <td id="LC5" class="blob-code blob-code-inner js-file-line"><span class="pl-c"><span class="pl-c">%</span> 20Mar 2016</span></td>
      </tr>
      <tr>
        <td id="L6" class="blob-num js-line-number" data-line-number="6"></td>
        <td id="LC6" class="blob-code blob-code-inner js-file-line"><span class="pl-c"><span class="pl-c">%</span></span></td>
      </tr>
      <tr>
        <td id="L7" class="blob-num js-line-number" data-line-number="7"></td>
        <td id="LC7" class="blob-code blob-code-inner js-file-line"><span class="pl-c"><span class="pl-c">%</span>%%%%%%%%%%%%%%% Springer %%%%%%%%%%%%%%%%%%%%%%%%%%%%%%%%%%</span></td>
      </tr>
      <tr>
        <td id="L8" class="blob-num js-line-number" data-line-number="8"></td>
        <td id="LC8" class="blob-code blob-code-inner js-file-line">
</td>
      </tr>
      <tr>
        <td id="L9" class="blob-num js-line-number" data-line-number="9"></td>
        <td id="LC9" class="blob-code blob-code-inner js-file-line">
</td>
      </tr>
      <tr>
        <td id="L10" class="blob-num js-line-number" data-line-number="10"></td>
        <td id="LC10" class="blob-code blob-code-inner js-file-line"><span class="pl-c"><span class="pl-c">%</span> RECOMMENDED %%%%%%%%%%%%%%%%%%%%%%%%%%%%%%%%%%%%%%%%%%%%%%%%%%%</span></td>
      </tr>
      <tr>
        <td id="L11" class="blob-num js-line-number" data-line-number="11"></td>
        <td id="LC11" class="blob-code blob-code-inner js-file-line"><span class="pl-c1">\documentclass</span>[graybox,natbib,nosecnum]{svmult}</td>
      </tr>
      <tr>
        <td id="L12" class="blob-num js-line-number" data-line-number="12"></td>
        <td id="LC12" class="blob-code blob-code-inner js-file-line"><span class="pl-c1">\bibpunct</span>{(}{)}{;}{a}{}{,} <span class="pl-c"><span class="pl-c">%</span> suppress commas between author-names and year</span></td>
      </tr>
      <tr>
        <td id="L13" class="blob-num js-line-number" data-line-number="13"></td>
        <td id="LC13" class="blob-code blob-code-inner js-file-line">
</td>
      </tr>
      <tr>
        <td id="L14" class="blob-num js-line-number" data-line-number="14"></td>
        <td id="LC14" class="blob-code blob-code-inner js-file-line"><span class="pl-c1">\pdfoutput</span>=1   <span class="pl-c"><span class="pl-c">%</span>forces use of pdflatex. Disable if you prefer to use .eps or .ps figures.</span></td>
      </tr>
      <tr>
        <td id="L15" class="blob-num js-line-number" data-line-number="15"></td>
        <td id="LC15" class="blob-code blob-code-inner js-file-line"><span class="pl-c"><span class="pl-c">%</span> choose options for [] as required from the list</span></td>
      </tr>
      <tr>
        <td id="L16" class="blob-num js-line-number" data-line-number="16"></td>
        <td id="LC16" class="blob-code blob-code-inner js-file-line"><span class="pl-c"><span class="pl-c">%</span> in the Reference Guide</span></td>
      </tr>
      <tr>
        <td id="L17" class="blob-num js-line-number" data-line-number="17"></td>
        <td id="LC17" class="blob-code blob-code-inner js-file-line">
</td>
      </tr>
      <tr>
        <td id="L18" class="blob-num js-line-number" data-line-number="18"></td>
        <td id="LC18" class="blob-code blob-code-inner js-file-line"><span class="pl-c1">\usepackage</span>{mathptmx}       <span class="pl-c"><span class="pl-c">%</span> selects Times Roman as basic font</span></td>
      </tr>
      <tr>
        <td id="L19" class="blob-num js-line-number" data-line-number="19"></td>
        <td id="LC19" class="blob-code blob-code-inner js-file-line"><span class="pl-c1">\usepackage</span>{helvet}         <span class="pl-c"><span class="pl-c">%</span> selects Helvetica as sans-serif font</span></td>
      </tr>
      <tr>
        <td id="L20" class="blob-num js-line-number" data-line-number="20"></td>
        <td id="LC20" class="blob-code blob-code-inner js-file-line"><span class="pl-c1">\usepackage</span>{courier}        <span class="pl-c"><span class="pl-c">%</span> selects Courier as typewriter font</span></td>
      </tr>
      <tr>
        <td id="L21" class="blob-num js-line-number" data-line-number="21"></td>
        <td id="LC21" class="blob-code blob-code-inner js-file-line"><span class="pl-c1">\usepackage</span>{type1cm}        <span class="pl-c"><span class="pl-c">%</span> activate if the above 3 fonts are</span></td>
      </tr>
      <tr>
        <td id="L22" class="blob-num js-line-number" data-line-number="22"></td>
        <td id="LC22" class="blob-code blob-code-inner js-file-line">                            <span class="pl-c"><span class="pl-c">%</span> not available on your system</span></td>
      </tr>
      <tr>
        <td id="L23" class="blob-num js-line-number" data-line-number="23"></td>
        <td id="LC23" class="blob-code blob-code-inner js-file-line">
</td>
      </tr>
      <tr>
        <td id="L24" class="blob-num js-line-number" data-line-number="24"></td>
        <td id="LC24" class="blob-code blob-code-inner js-file-line"><span class="pl-c1">\usepackage</span>{makeidx}         <span class="pl-c"><span class="pl-c">%</span> allows index generation</span></td>
      </tr>
      <tr>
        <td id="L25" class="blob-num js-line-number" data-line-number="25"></td>
        <td id="LC25" class="blob-code blob-code-inner js-file-line"><span class="pl-c1">\usepackage</span>{graphicx}        <span class="pl-c"><span class="pl-c">%</span> standard LaTeX graphics tool</span></td>
      </tr>
      <tr>
        <td id="L26" class="blob-num js-line-number" data-line-number="26"></td>
        <td id="LC26" class="blob-code blob-code-inner js-file-line"><span class="pl-c1">\usepackage</span>{units}        <span class="pl-c"><span class="pl-c">%</span></span></td>
      </tr>
      <tr>
        <td id="L27" class="blob-num js-line-number" data-line-number="27"></td>
        <td id="LC27" class="blob-code blob-code-inner js-file-line"><span class="pl-c1">\usepackage</span>{amssymb}        <span class="pl-c"><span class="pl-c">%</span></span></td>
      </tr>
      <tr>
        <td id="L28" class="blob-num js-line-number" data-line-number="28"></td>
        <td id="LC28" class="blob-code blob-code-inner js-file-line">                             <span class="pl-c"><span class="pl-c">%</span> when including figure files</span></td>
      </tr>
      <tr>
        <td id="L29" class="blob-num js-line-number" data-line-number="29"></td>
        <td id="LC29" class="blob-code blob-code-inner js-file-line"><span class="pl-c1">\usepackage</span>{multicol}        <span class="pl-c"><span class="pl-c">%</span> used for the two-column index</span></td>
      </tr>
      <tr>
        <td id="L30" class="blob-num js-line-number" data-line-number="30"></td>
        <td id="LC30" class="blob-code blob-code-inner js-file-line"><span class="pl-c1">\usepackage</span>[bottom]{footmisc}<span class="pl-c"><span class="pl-c">%</span> places footnotes at page bottom</span></td>
      </tr>
      <tr>
        <td id="L31" class="blob-num js-line-number" data-line-number="31"></td>
        <td id="LC31" class="blob-code blob-code-inner js-file-line"><span class="pl-c1">\usepackage</span>[normalem]{ulem}	<span class="pl-c"><span class="pl-c">%</span> for strike-through of text with \sout{}  </span></td>
      </tr>
      <tr>
        <td id="L32" class="blob-num js-line-number" data-line-number="32"></td>
        <td id="LC32" class="blob-code blob-code-inner js-file-line"><span class="pl-c1">\usepackage</span>{hyperref}  <span class="pl-c"><span class="pl-c">%</span>for hyperlinks</span></td>
      </tr>
      <tr>
        <td id="L33" class="blob-num js-line-number" data-line-number="33"></td>
        <td id="LC33" class="blob-code blob-code-inner js-file-line">
</td>
      </tr>
      <tr>
        <td id="L34" class="blob-num js-line-number" data-line-number="34"></td>
        <td id="LC34" class="blob-code blob-code-inner js-file-line">
</td>
      </tr>
      <tr>
        <td id="L35" class="blob-num js-line-number" data-line-number="35"></td>
        <td id="LC35" class="blob-code blob-code-inner js-file-line"><span class="pl-c1">\usepackage</span>{soul}   <span class="pl-c"><span class="pl-c">%</span> for high-lighting of text</span></td>
      </tr>
      <tr>
        <td id="L36" class="blob-num js-line-number" data-line-number="36"></td>
        <td id="LC36" class="blob-code blob-code-inner js-file-line"><span class="pl-c"><span class="pl-c">%</span> see the list of further useful packages</span></td>
      </tr>
      <tr>
        <td id="L37" class="blob-num js-line-number" data-line-number="37"></td>
        <td id="LC37" class="blob-code blob-code-inner js-file-line"><span class="pl-c"><span class="pl-c">%</span> in the Reference Guide</span></td>
      </tr>
      <tr>
        <td id="L38" class="blob-num js-line-number" data-line-number="38"></td>
        <td id="LC38" class="blob-code blob-code-inner js-file-line">
</td>
      </tr>
      <tr>
        <td id="L39" class="blob-num js-line-number" data-line-number="39"></td>
        <td id="LC39" class="blob-code blob-code-inner js-file-line"><span class="pl-c"><span class="pl-c">%</span> expansions of  journal abbreviations from bibtex entries by ADS</span></td>
      </tr>
      <tr>
        <td id="L40" class="blob-num js-line-number" data-line-number="40"></td>
        <td id="LC40" class="blob-code blob-code-inner js-file-line"><span class="pl-c"><span class="pl-c">%</span> adapted to Springer Basic style (no periods in abbreviations)</span></td>
      </tr>
      <tr>
        <td id="L41" class="blob-num js-line-number" data-line-number="41"></td>
        <td id="LC41" class="blob-code blob-code-inner js-file-line"><span class="pl-c1">\newcommand</span>*<span class="pl-c1">\aap</span>{A<span class="pl-cce">\&amp;</span>A}</td>
      </tr>
      <tr>
        <td id="L42" class="blob-num js-line-number" data-line-number="42"></td>
        <td id="LC42" class="blob-code blob-code-inner js-file-line"><span class="pl-c1">\let\astap</span>=<span class="pl-c1">\aap</span></td>
      </tr>
      <tr>
        <td id="L43" class="blob-num js-line-number" data-line-number="43"></td>
        <td id="LC43" class="blob-code blob-code-inner js-file-line"><span class="pl-c1">\newcommand</span>*<span class="pl-c1">\aapr</span>{A<span class="pl-cce">\&amp;</span>A Rev}</td>
      </tr>
      <tr>
        <td id="L44" class="blob-num js-line-number" data-line-number="44"></td>
        <td id="LC44" class="blob-code blob-code-inner js-file-line"><span class="pl-c1">\newcommand</span>*<span class="pl-c1">\aaps</span>{A<span class="pl-cce">\&amp;</span>AS}</td>
      </tr>
      <tr>
        <td id="L45" class="blob-num js-line-number" data-line-number="45"></td>
        <td id="LC45" class="blob-code blob-code-inner js-file-line"><span class="pl-c1">\newcommand</span>*<span class="pl-c1">\actaa</span>{Acta Astron}</td>
      </tr>
      <tr>
        <td id="L46" class="blob-num js-line-number" data-line-number="46"></td>
        <td id="LC46" class="blob-code blob-code-inner js-file-line"><span class="pl-c1">\newcommand</span>*<span class="pl-c1">\aj</span>{AJ}</td>
      </tr>
      <tr>
        <td id="L47" class="blob-num js-line-number" data-line-number="47"></td>
        <td id="LC47" class="blob-code blob-code-inner js-file-line"><span class="pl-c1">\newcommand</span>*<span class="pl-c1">\ao</span>{Appl Opt}</td>
      </tr>
      <tr>
        <td id="L48" class="blob-num js-line-number" data-line-number="48"></td>
        <td id="LC48" class="blob-code blob-code-inner js-file-line"><span class="pl-c1">\let\applopt\ao</span></td>
      </tr>
      <tr>
        <td id="L49" class="blob-num js-line-number" data-line-number="49"></td>
        <td id="LC49" class="blob-code blob-code-inner js-file-line"><span class="pl-c1">\newcommand</span>*<span class="pl-c1">\apj</span>{ApJ}</td>
      </tr>
      <tr>
        <td id="L50" class="blob-num js-line-number" data-line-number="50"></td>
        <td id="LC50" class="blob-code blob-code-inner js-file-line"><span class="pl-c1">\newcommand</span>*<span class="pl-c1">\apjl</span>{ApJ}</td>
      </tr>
      <tr>
        <td id="L51" class="blob-num js-line-number" data-line-number="51"></td>
        <td id="LC51" class="blob-code blob-code-inner js-file-line"><span class="pl-c1">\let\apjlett\apjl</span></td>
      </tr>
      <tr>
        <td id="L52" class="blob-num js-line-number" data-line-number="52"></td>
        <td id="LC52" class="blob-code blob-code-inner js-file-line"><span class="pl-c1">\newcommand</span>*<span class="pl-c1">\apjs</span>{ApJS}</td>
      </tr>
      <tr>
        <td id="L53" class="blob-num js-line-number" data-line-number="53"></td>
        <td id="LC53" class="blob-code blob-code-inner js-file-line"><span class="pl-c1">\let\apjsupp\apjs</span></td>
      </tr>
      <tr>
        <td id="L54" class="blob-num js-line-number" data-line-number="54"></td>
        <td id="LC54" class="blob-code blob-code-inner js-file-line"><span class="pl-c1">\newcommand</span>*<span class="pl-c1">\aplett</span>{Astrophys Lett}</td>
      </tr>
      <tr>
        <td id="L55" class="blob-num js-line-number" data-line-number="55"></td>
        <td id="LC55" class="blob-code blob-code-inner js-file-line"><span class="pl-c1">\newcommand</span>*<span class="pl-c1">\apspr</span>{Astrophys Space Phys Res}</td>
      </tr>
      <tr>
        <td id="L56" class="blob-num js-line-number" data-line-number="56"></td>
        <td id="LC56" class="blob-code blob-code-inner js-file-line"><span class="pl-c1">\newcommand</span>*<span class="pl-c1">\apss</span>{Ap<span class="pl-cce">\&amp;</span>SS}</td>
      </tr>
      <tr>
        <td id="L57" class="blob-num js-line-number" data-line-number="57"></td>
        <td id="LC57" class="blob-code blob-code-inner js-file-line"><span class="pl-c1">\newcommand</span>*<span class="pl-c1">\araa</span>{ARA<span class="pl-cce">\&amp;</span>A}</td>
      </tr>
      <tr>
        <td id="L58" class="blob-num js-line-number" data-line-number="58"></td>
        <td id="LC58" class="blob-code blob-code-inner js-file-line"><span class="pl-c1">\newcommand</span>*<span class="pl-c1">\azh</span>{AZh}</td>
      </tr>
      <tr>
        <td id="L59" class="blob-num js-line-number" data-line-number="59"></td>
        <td id="LC59" class="blob-code blob-code-inner js-file-line"><span class="pl-c1">\newcommand</span>*<span class="pl-c1">\baas</span>{BAAS}</td>
      </tr>
      <tr>
        <td id="L60" class="blob-num js-line-number" data-line-number="60"></td>
        <td id="LC60" class="blob-code blob-code-inner js-file-line"><span class="pl-c1">\newcommand</span>*<span class="pl-c1">\bac</span>{Bull astr Inst Czechosl}</td>
      </tr>
      <tr>
        <td id="L61" class="blob-num js-line-number" data-line-number="61"></td>
        <td id="LC61" class="blob-code blob-code-inner js-file-line"><span class="pl-c1">\newcommand</span>*<span class="pl-c1">\bain</span>{Bull Astron Inst Netherlands}</td>
      </tr>
      <tr>
        <td id="L62" class="blob-num js-line-number" data-line-number="62"></td>
        <td id="LC62" class="blob-code blob-code-inner js-file-line"><span class="pl-c1">\newcommand</span>*<span class="pl-c1">\caa</span>{Chinese Astron Astrophys}</td>
      </tr>
      <tr>
        <td id="L63" class="blob-num js-line-number" data-line-number="63"></td>
        <td id="LC63" class="blob-code blob-code-inner js-file-line"><span class="pl-c1">\newcommand</span>*<span class="pl-c1">\cjaa</span>{Chinese J Astron Astrophys}</td>
      </tr>
      <tr>
        <td id="L64" class="blob-num js-line-number" data-line-number="64"></td>
        <td id="LC64" class="blob-code blob-code-inner js-file-line"><span class="pl-c1">\newcommand</span>*<span class="pl-c1">\fcp</span>{Fund Cosmic Phys}</td>
      </tr>
      <tr>
        <td id="L65" class="blob-num js-line-number" data-line-number="65"></td>
        <td id="LC65" class="blob-code blob-code-inner js-file-line"><span class="pl-c1">\newcommand</span>*<span class="pl-c1">\gca</span>{Geochim Cosmochim Acta}</td>
      </tr>
      <tr>
        <td id="L66" class="blob-num js-line-number" data-line-number="66"></td>
        <td id="LC66" class="blob-code blob-code-inner js-file-line"><span class="pl-c1">\newcommand</span>*<span class="pl-c1">\grl</span>{Geophys Res Lett}</td>
      </tr>
      <tr>
        <td id="L67" class="blob-num js-line-number" data-line-number="67"></td>
        <td id="LC67" class="blob-code blob-code-inner js-file-line"><span class="pl-c1">\newcommand</span>*<span class="pl-c1">\iaucirc</span>{IAU Circ}</td>
      </tr>
      <tr>
        <td id="L68" class="blob-num js-line-number" data-line-number="68"></td>
        <td id="LC68" class="blob-code blob-code-inner js-file-line"><span class="pl-c1">\newcommand</span>*<span class="pl-c1">\icarus</span>{Icarus}</td>
      </tr>
      <tr>
        <td id="L69" class="blob-num js-line-number" data-line-number="69"></td>
        <td id="LC69" class="blob-code blob-code-inner js-file-line"><span class="pl-c1">\newcommand</span>*<span class="pl-c1">\jcap</span>{J Cosmology Astropart Phys}</td>
      </tr>
      <tr>
        <td id="L70" class="blob-num js-line-number" data-line-number="70"></td>
        <td id="LC70" class="blob-code blob-code-inner js-file-line"><span class="pl-c1">\newcommand</span>*<span class="pl-c1">\jcp</span>{J Chem Phys}</td>
      </tr>
      <tr>
        <td id="L71" class="blob-num js-line-number" data-line-number="71"></td>
        <td id="LC71" class="blob-code blob-code-inner js-file-line"><span class="pl-c1">\newcommand</span>*<span class="pl-c1">\jgr</span>{J Geophys Res}</td>
      </tr>
      <tr>
        <td id="L72" class="blob-num js-line-number" data-line-number="72"></td>
        <td id="LC72" class="blob-code blob-code-inner js-file-line"><span class="pl-c1">\newcommand</span>*<span class="pl-c1">\jqsrt</span>{J Quant Spectr Rad Transf}</td>
      </tr>
      <tr>
        <td id="L73" class="blob-num js-line-number" data-line-number="73"></td>
        <td id="LC73" class="blob-code blob-code-inner js-file-line"><span class="pl-c1">\newcommand</span>*<span class="pl-c1">\jrasc</span>{JRASC}</td>
      </tr>
      <tr>
        <td id="L74" class="blob-num js-line-number" data-line-number="74"></td>
        <td id="LC74" class="blob-code blob-code-inner js-file-line"><span class="pl-c1">\newcommand</span>*<span class="pl-c1">\memras</span>{MmRAS}</td>
      </tr>
      <tr>
        <td id="L75" class="blob-num js-line-number" data-line-number="75"></td>
        <td id="LC75" class="blob-code blob-code-inner js-file-line"><span class="pl-c1">\newcommand</span>*<span class="pl-c1">\memsai</span>{Mem Soc Astron Italiana}</td>
      </tr>
      <tr>
        <td id="L76" class="blob-num js-line-number" data-line-number="76"></td>
        <td id="LC76" class="blob-code blob-code-inner js-file-line"><span class="pl-c1">\newcommand</span>*<span class="pl-c1">\mnras</span>{MNRAS}</td>
      </tr>
      <tr>
        <td id="L77" class="blob-num js-line-number" data-line-number="77"></td>
        <td id="LC77" class="blob-code blob-code-inner js-file-line"><span class="pl-c1">\newcommand</span>*<span class="pl-c1">\na</span>{New A}</td>
      </tr>
      <tr>
        <td id="L78" class="blob-num js-line-number" data-line-number="78"></td>
        <td id="LC78" class="blob-code blob-code-inner js-file-line"><span class="pl-c1">\newcommand</span>*<span class="pl-c1">\nar</span>{New A Rev}</td>
      </tr>
      <tr>
        <td id="L79" class="blob-num js-line-number" data-line-number="79"></td>
        <td id="LC79" class="blob-code blob-code-inner js-file-line"><span class="pl-c1">\newcommand</span>*<span class="pl-c1">\nat</span>{Nature}</td>
      </tr>
      <tr>
        <td id="L80" class="blob-num js-line-number" data-line-number="80"></td>
        <td id="LC80" class="blob-code blob-code-inner js-file-line"><span class="pl-c1">\newcommand</span>*<span class="pl-c1">\nphysa</span>{Nucl Phys A}</td>
      </tr>
      <tr>
        <td id="L81" class="blob-num js-line-number" data-line-number="81"></td>
        <td id="LC81" class="blob-code blob-code-inner js-file-line"><span class="pl-c1">\newcommand</span>*<span class="pl-c1">\pasa</span>{PASA}</td>
      </tr>
      <tr>
        <td id="L82" class="blob-num js-line-number" data-line-number="82"></td>
        <td id="LC82" class="blob-code blob-code-inner js-file-line"><span class="pl-c1">\newcommand</span>*<span class="pl-c1">\pasj</span>{PASJ}</td>
      </tr>
      <tr>
        <td id="L83" class="blob-num js-line-number" data-line-number="83"></td>
        <td id="LC83" class="blob-code blob-code-inner js-file-line"><span class="pl-c1">\newcommand</span>*<span class="pl-c1">\pasp</span>{PASP}</td>
      </tr>
      <tr>
        <td id="L84" class="blob-num js-line-number" data-line-number="84"></td>
        <td id="LC84" class="blob-code blob-code-inner js-file-line"><span class="pl-c1">\newcommand</span>*<span class="pl-c1">\physrep</span>{Phys Rep}</td>
      </tr>
      <tr>
        <td id="L85" class="blob-num js-line-number" data-line-number="85"></td>
        <td id="LC85" class="blob-code blob-code-inner js-file-line"><span class="pl-c1">\newcommand</span>*<span class="pl-c1">\physscr</span>{Phys Scr}</td>
      </tr>
      <tr>
        <td id="L86" class="blob-num js-line-number" data-line-number="86"></td>
        <td id="LC86" class="blob-code blob-code-inner js-file-line"><span class="pl-c1">\newcommand</span>*<span class="pl-c1">\planss</span>{Planet Space Sci}</td>
      </tr>
      <tr>
        <td id="L87" class="blob-num js-line-number" data-line-number="87"></td>
        <td id="LC87" class="blob-code blob-code-inner js-file-line"><span class="pl-c1">\newcommand</span>*<span class="pl-c1">\pra</span>{Phys Rev A}</td>
      </tr>
      <tr>
        <td id="L88" class="blob-num js-line-number" data-line-number="88"></td>
        <td id="LC88" class="blob-code blob-code-inner js-file-line"><span class="pl-c1">\newcommand</span>*<span class="pl-c1">\prb</span>{Phys Rev B}</td>
      </tr>
      <tr>
        <td id="L89" class="blob-num js-line-number" data-line-number="89"></td>
        <td id="LC89" class="blob-code blob-code-inner js-file-line"><span class="pl-c1">\newcommand</span>*<span class="pl-c1">\prc</span>{Phys Rev C}</td>
      </tr>
      <tr>
        <td id="L90" class="blob-num js-line-number" data-line-number="90"></td>
        <td id="LC90" class="blob-code blob-code-inner js-file-line"><span class="pl-c1">\newcommand</span>*<span class="pl-c1">\prd</span>{Phys Rev D}</td>
      </tr>
      <tr>
        <td id="L91" class="blob-num js-line-number" data-line-number="91"></td>
        <td id="LC91" class="blob-code blob-code-inner js-file-line"><span class="pl-c1">\newcommand</span>*<span class="pl-c1">\pre</span>{Phys Rev E}</td>
      </tr>
      <tr>
        <td id="L92" class="blob-num js-line-number" data-line-number="92"></td>
        <td id="LC92" class="blob-code blob-code-inner js-file-line"><span class="pl-c1">\newcommand</span>*<span class="pl-c1">\prl</span>{Phys Rev Lett}</td>
      </tr>
      <tr>
        <td id="L93" class="blob-num js-line-number" data-line-number="93"></td>
        <td id="LC93" class="blob-code blob-code-inner js-file-line"><span class="pl-c1">\newcommand</span>*<span class="pl-c1">\procspie</span>{Proc SPIE}</td>
      </tr>
      <tr>
        <td id="L94" class="blob-num js-line-number" data-line-number="94"></td>
        <td id="LC94" class="blob-code blob-code-inner js-file-line"><span class="pl-c1">\newcommand</span>*<span class="pl-c1">\qjras</span>{QJRAS}</td>
      </tr>
      <tr>
        <td id="L95" class="blob-num js-line-number" data-line-number="95"></td>
        <td id="LC95" class="blob-code blob-code-inner js-file-line"><span class="pl-c1">\newcommand</span>*<span class="pl-c1">\rmxaa</span>{Rev Mexicana Astron Astrofis}</td>
      </tr>
      <tr>
        <td id="L96" class="blob-num js-line-number" data-line-number="96"></td>
        <td id="LC96" class="blob-code blob-code-inner js-file-line"><span class="pl-c1">\newcommand</span>*<span class="pl-c1">\skytel</span>{S<span class="pl-cce">\&amp;</span>T}</td>
      </tr>
      <tr>
        <td id="L97" class="blob-num js-line-number" data-line-number="97"></td>
        <td id="LC97" class="blob-code blob-code-inner js-file-line"><span class="pl-c1">\newcommand</span>*<span class="pl-c1">\solphys</span>{Sol Phys}</td>
      </tr>
      <tr>
        <td id="L98" class="blob-num js-line-number" data-line-number="98"></td>
        <td id="LC98" class="blob-code blob-code-inner js-file-line"><span class="pl-c1">\newcommand</span>*<span class="pl-c1">\sovast</span>{Soviet Ast}</td>
      </tr>
      <tr>
        <td id="L99" class="blob-num js-line-number" data-line-number="99"></td>
        <td id="LC99" class="blob-code blob-code-inner js-file-line"><span class="pl-c1">\newcommand</span>*<span class="pl-c1">\ssr</span>{Space Sci Rev}</td>
      </tr>
      <tr>
        <td id="L100" class="blob-num js-line-number" data-line-number="100"></td>
        <td id="LC100" class="blob-code blob-code-inner js-file-line"><span class="pl-c1">\newcommand</span>*<span class="pl-c1">\zap</span>{ZAp}</td>
      </tr>
      <tr>
        <td id="L101" class="blob-num js-line-number" data-line-number="101"></td>
        <td id="LC101" class="blob-code blob-code-inner js-file-line">
</td>
      </tr>
      <tr>
        <td id="L102" class="blob-num js-line-number" data-line-number="102"></td>
        <td id="LC102" class="blob-code blob-code-inner js-file-line">
</td>
      </tr>
      <tr>
        <td id="L103" class="blob-num js-line-number" data-line-number="103"></td>
        <td id="LC103" class="blob-code blob-code-inner js-file-line"><span class="pl-c1">\newcommand</span>{<span class="pl-c1">\hbindex</span>}[1]{<span class="pl-c1">\hl</span>{#1}<span class="pl-c1">\index</span>{#1}}  <span class="pl-c"><span class="pl-c">%</span>highlights index entries</span></td>
      </tr>
      <tr>
        <td id="L104" class="blob-num js-line-number" data-line-number="104"></td>
        <td id="LC104" class="blob-code blob-code-inner js-file-line">
</td>
      </tr>
      <tr>
        <td id="L105" class="blob-num js-line-number" data-line-number="105"></td>
        <td id="LC105" class="blob-code blob-code-inner js-file-line"><span class="pl-c1">\makeindex</span>             <span class="pl-c"><span class="pl-c">%</span> used for the subject index</span></td>
      </tr>
      <tr>
        <td id="L106" class="blob-num js-line-number" data-line-number="106"></td>
        <td id="LC106" class="blob-code blob-code-inner js-file-line">                       <span class="pl-c"><span class="pl-c">%</span> please use the style svind.ist with</span></td>
      </tr>
      <tr>
        <td id="L107" class="blob-num js-line-number" data-line-number="107"></td>
        <td id="LC107" class="blob-code blob-code-inner js-file-line">                       <span class="pl-c"><span class="pl-c">%</span> your makeindex program</span></td>
      </tr>
      <tr>
        <td id="L108" class="blob-num js-line-number" data-line-number="108"></td>
        <td id="LC108" class="blob-code blob-code-inner js-file-line">
</td>
      </tr>
      <tr>
        <td id="L109" class="blob-num js-line-number" data-line-number="109"></td>
        <td id="LC109" class="blob-code blob-code-inner js-file-line"><span class="pl-c"><span class="pl-c">%</span>%%%%%%%%%%%%%%%%%%%%%%%%%%%%%%%%%%%%%%%%%%%%%%%%%%%%%%%%%%%%%%%%%%%%%%%%%%%%%%%%%%%%%%%%</span></td>
      </tr>
      <tr>
        <td id="L110" class="blob-num js-line-number" data-line-number="110"></td>
        <td id="LC110" class="blob-code blob-code-inner js-file-line">
</td>
      </tr>
      <tr>
        <td id="L111" class="blob-num js-line-number" data-line-number="111"></td>
        <td id="LC111" class="blob-code blob-code-inner js-file-line"><span class="pl-c1">\begin</span>{document}</td>
      </tr>
      <tr>
        <td id="L112" class="blob-num js-line-number" data-line-number="112"></td>
        <td id="LC112" class="blob-code blob-code-inner js-file-line">
</td>
      </tr>
      <tr>
        <td id="L113" class="blob-num js-line-number" data-line-number="113"></td>
        <td id="LC113" class="blob-code blob-code-inner js-file-line"><span class="pl-c1">\title</span>*{Transit Timing and Duration Variations for the Discovery and Characterization of Exoplanets}</td>
      </tr>
      <tr>
        <td id="L114" class="blob-num js-line-number" data-line-number="114"></td>
        <td id="LC114" class="blob-code blob-code-inner js-file-line"><span class="pl-c"><span class="pl-c">%</span> Use \titlerunning{Short Title} for an abbreviated version of</span></td>
      </tr>
      <tr>
        <td id="L115" class="blob-num js-line-number" data-line-number="115"></td>
        <td id="LC115" class="blob-code blob-code-inner js-file-line"><span class="pl-c"><span class="pl-c">%</span> your contribution title if the original one is too long</span></td>
      </tr>
      <tr>
        <td id="L116" class="blob-num js-line-number" data-line-number="116"></td>
        <td id="LC116" class="blob-code blob-code-inner js-file-line"><span class="pl-c1">\titlerunning</span>{Transit Timing for Characterization and Discovery } </td>
      </tr>
      <tr>
        <td id="L117" class="blob-num js-line-number" data-line-number="117"></td>
        <td id="LC117" class="blob-code blob-code-inner js-file-line"><span class="pl-c1">\author</span>{Eric Agol and Daniel C.<span class="pl-cce">\ </span>Fabrycky}</td>
      </tr>
      <tr>
        <td id="L118" class="blob-num js-line-number" data-line-number="118"></td>
        <td id="LC118" class="blob-code blob-code-inner js-file-line"><span class="pl-c"><span class="pl-c">%</span> Use </span></td>
      </tr>
      <tr>
        <td id="L119" class="blob-num js-line-number" data-line-number="119"></td>
        <td id="LC119" class="blob-code blob-code-inner js-file-line"><span class="pl-c1">\authorrunning</span>{Agol <span class="pl-cce">\&amp;</span> Fabrycky} </td>
      </tr>
      <tr>
        <td id="L120" class="blob-num js-line-number" data-line-number="120"></td>
        <td id="LC120" class="blob-code blob-code-inner js-file-line"><span class="pl-c1">\institute</span>{Eric Agol <span class="pl-c1">\at</span> Department of Astronomy, Box 351580, University of Washington, Seattle, WA 98195-1580, USA <span class="pl-c1">\email</span>{agol@uw.edu}</td>
      </tr>
      <tr>
        <td id="L121" class="blob-num js-line-number" data-line-number="121"></td>
        <td id="LC121" class="blob-code blob-code-inner js-file-line"><span class="pl-c1">\and</span> Daniel C.<span class="pl-cce">\ </span>Fabrycky <span class="pl-c1">\at</span> Dept.<span class="pl-cce">\ </span>of Astronomy <span class="pl-cce">\&amp;</span> Astrophysics, University of Chicago, Chicago, IL 60637, USA <span class="pl-c1">\email</span>{fabrycky@uchicago.edu}}</td>
      </tr>
      <tr>
        <td id="L122" class="blob-num js-line-number" data-line-number="122"></td>
        <td id="LC122" class="blob-code blob-code-inner js-file-line"><span class="pl-c"><span class="pl-c">%</span></span></td>
      </tr>
      <tr>
        <td id="L123" class="blob-num js-line-number" data-line-number="123"></td>
        <td id="LC123" class="blob-code blob-code-inner js-file-line"><span class="pl-c"><span class="pl-c">%</span> Use the package &quot;url.sty&quot; to avoid</span></td>
      </tr>
      <tr>
        <td id="L124" class="blob-num js-line-number" data-line-number="124"></td>
        <td id="LC124" class="blob-code blob-code-inner js-file-line"><span class="pl-c"><span class="pl-c">%</span> problems with special characters</span></td>
      </tr>
      <tr>
        <td id="L125" class="blob-num js-line-number" data-line-number="125"></td>
        <td id="LC125" class="blob-code blob-code-inner js-file-line"><span class="pl-c"><span class="pl-c">%</span> used in your e-mail or web address</span></td>
      </tr>
      <tr>
        <td id="L126" class="blob-num js-line-number" data-line-number="126"></td>
        <td id="LC126" class="blob-code blob-code-inner js-file-line"><span class="pl-c"><span class="pl-c">%</span></span></td>
      </tr>
      <tr>
        <td id="L127" class="blob-num js-line-number" data-line-number="127"></td>
        <td id="LC127" class="blob-code blob-code-inner js-file-line"><span class="pl-c1">\maketitle</span></td>
      </tr>
      <tr>
        <td id="L128" class="blob-num js-line-number" data-line-number="128"></td>
        <td id="LC128" class="blob-code blob-code-inner js-file-line">
</td>
      </tr>
      <tr>
        <td id="L129" class="blob-num js-line-number" data-line-number="129"></td>
        <td id="LC129" class="blob-code blob-code-inner js-file-line">
</td>
      </tr>
      <tr>
        <td id="L130" class="blob-num js-line-number" data-line-number="130"></td>
        <td id="LC130" class="blob-code blob-code-inner js-file-line">\abstract{Transiting exoplanets in multi-planet systems have non-Keplerian orbits which can cause the times and durations of transits to vary.  The theory and observations of transit timing variations (TTV) and transit duration variations (TDV) are reviewed.  Since the last review, the \emph{Kepler} spacecraft has detected several hundred perturbed planets.  In a few cases, these data have been used to discover additional planets, similar to the historical discovery of Neptune in our own Solar System.  However, the more impactful aspect of TTV and TDV studies has been characterization of planetary systems in which multiple planets transit.  After addressing the equations of motion and parameter scalings, the main dynamical mechanisms for TTV and TDV are described, with citations to the observational literature for real examples.  We describe parameter constraints, particularly the origin of the mass/eccentricity degeneracy and how it is overcome by the high-frequency component of the signal.  On the observational side, derivation of timing precision and introduction to the timing diagram are given.  Science results are reviewed, with an emphasis on mass measurements of transiting sub-Neptunes and super-Earths, which allows access to the mass-radius diagram and hence inference of bulk compositions.  }</td>
      </tr>
      <tr>
        <td id="L131" class="blob-num js-line-number" data-line-number="131"></td>
        <td id="LC131" class="blob-code blob-code-inner js-file-line">
</td>
      </tr>
      <tr>
        <td id="L132" class="blob-num js-line-number" data-line-number="132"></td>
        <td id="LC132" class="blob-code blob-code-inner js-file-line"><span class="pl-c1">\section</span>{Introduction}</td>
      </tr>
      <tr>
        <td id="L133" class="blob-num js-line-number" data-line-number="133"></td>
        <td id="LC133" class="blob-code blob-code-inner js-file-line">
</td>
      </tr>
      <tr>
        <td id="L134" class="blob-num js-line-number" data-line-number="134"></td>
        <td id="LC134" class="blob-code blob-code-inner js-file-line">Transit Timing Variations (TTV) and Transit Duration Variations (TDV) are two of the newest tools in the exoplanetary observer&#39;s toolbox for discovering and characterizing planetary systems. Like most such tools, they rely on indirect inferences, rather than detecting light from the planet directly.  However, the amount of dynamical information they encode is extremely rich. </td>
      </tr>
      <tr>
        <td id="L135" class="blob-num js-line-number" data-line-number="135"></td>
        <td id="LC135" class="blob-code blob-code-inner js-file-line">
</td>
      </tr>
      <tr>
        <td id="L136" class="blob-num js-line-number" data-line-number="136"></td>
        <td id="LC136" class="blob-code blob-code-inner js-file-line">To decode this information, let us start with the dynamical concepts.  Consider the vector stretching from the star of mass $m_0$ to the planet of mass $m$ to be $\mathbf{r}=(x,y,z)$, with a distance $r$ and direction $\mathbf{\hat r}$.  The Keplerian potential per reduced mass, $\phi=-GM/r$ (where $M \equiv m_0 + m$ and the planet is replaced with a body of reduced mass $\mu \equiv m_0 m /M$), gives rise to closed orbits.  This means that, in the absense of perturbations, the trajectory is strictly periodic, $\mathbf{r}(t+P) = \mathbf{r}(t)$.  Moreover, Kepler showed that Tycho Brahe&#39;s excellent data for planetary positions were consistent with Copernicus&#39; idea of a heliocentric system only if the planets (including the Earth) followed elliptical paths of semi-major axis $a$, and one focus on the Sun. Newton was successful at finding the principle underlying such orbits, a force law $\mathbf{F} = \mu \mathbf{\ddot r} =-G \mu m_0 r^{-2} \mathbf{\hat r}$, which results in a period $P = 2 \pi a^{3/2} (GM)^{-1/2}$ (i.e. with the $a$-scaling Kepler found the planets actually obeyed).</td>
      </tr>
      <tr>
        <td id="L137" class="blob-num js-line-number" data-line-number="137"></td>
        <td id="LC137" class="blob-code blob-code-inner js-file-line">
</td>
      </tr>
      <tr>
        <td id="L138" class="blob-num js-line-number" data-line-number="138"></td>
        <td id="LC138" class="blob-code blob-code-inner js-file-line">This research program was thrown into some doubt by the ``Great Inequality,&#39;&#39; the fact that the orbits of Jupiter and Saturn did not fit the fixed Keplerian ellipse model.  This obstacle was overcome by the perturbation theory of Laplace, who used the masses derived via their satellite orbits to explain their deviations of their heliocentric orbits <span class="pl-c1">\citep</span>{1985Wilson}.  We can recreate the main effect of this insight by writing an additional force to that of gravity of the Sun: </td>
      </tr>
      <tr>
        <td id="L139" class="blob-num js-line-number" data-line-number="139"></td>
        <td id="LC139" class="blob-code blob-code-inner js-file-line"><span class="pl-c1">\begin</span>{equation}</td>
      </tr>
      <tr>
        <td id="L140" class="blob-num js-line-number" data-line-number="140"></td>
        <td id="LC140" class="blob-code blob-code-inner js-file-line"><span class="pl-c1">\mathbf</span>{F_{1}} = -G <span class="pl-c1">\mu</span>_1 M r_{1}^{-2} <span class="pl-c1">\mathbf</span>{<span class="pl-c1">\hat</span> r_{1}} + <span class="pl-c1">\mathbf</span>{F_{12}},</td>
      </tr>
      <tr>
        <td id="L141" class="blob-num js-line-number" data-line-number="141"></td>
        <td id="LC141" class="blob-code blob-code-inner js-file-line"><span class="pl-c1">\end</span>{equation}</td>
      </tr>
      <tr>
        <td id="L142" class="blob-num js-line-number" data-line-number="142"></td>
        <td id="LC142" class="blob-code blob-code-inner js-file-line">where we now specify forces and distances explicitly to planet 1, and add a force of planet 2 on planet 1.  This latter force consists of two terms: </td>
      </tr>
      <tr>
        <td id="L143" class="blob-num js-line-number" data-line-number="143"></td>
        <td id="LC143" class="blob-code blob-code-inner js-file-line"><span class="pl-c1">\begin</span>{equation}</td>
      </tr>
      <tr>
        <td id="L144" class="blob-num js-line-number" data-line-number="144"></td>
        <td id="LC144" class="blob-code blob-code-inner js-file-line"><span class="pl-c1">\mathbf</span>{F_{12}} = <span class="pl-c1">\mu</span>_1 <span class="pl-c1">\mathbf</span>{<span class="pl-c1">\ddot</span> r_1} = G <span class="pl-c1">\mu</span>_1 m_2 <span class="pl-c1">\vert</span> r_{2}-r_{1}<span class="pl-c1">\vert</span>^{-3} (<span class="pl-c1">\mathbf</span>{r_{2}} - <span class="pl-c1">\mathbf</span>{r_{1}}) - G <span class="pl-c1">\mu</span>_1 m_2 r_{2}^{-2} <span class="pl-c1">\mathbf</span>{<span class="pl-c1">\hat</span> r_{2}}.</td>
      </tr>
      <tr>
        <td id="L145" class="blob-num js-line-number" data-line-number="145"></td>
        <td id="LC145" class="blob-code blob-code-inner js-file-line"><span class="pl-c1">\end</span>{equation}</td>
      </tr>
      <tr>
        <td id="L146" class="blob-num js-line-number" data-line-number="146"></td>
        <td id="LC146" class="blob-code blob-code-inner js-file-line">The first term on the right-hand-side is the direct gravitational acceleration of planet 1 due to planet 2.  The second is an indirect frame-acceleration effect, due to the acceleration the star feels due to the second planet.  Since the Sun is fixed at the zero of the frame, this acceleration is modelled by acceleration of planet 1 in the opposite direction.</td>
      </tr>
      <tr>
        <td id="L147" class="blob-num js-line-number" data-line-number="147"></td>
        <td id="LC147" class="blob-code blob-code-inner js-file-line">
</td>
      </tr>
      <tr>
        <td id="L148" class="blob-num js-line-number" data-line-number="148"></td>
        <td id="LC148" class="blob-code blob-code-inner js-file-line">Likewise, Leverrier and Adams used planet-planet perturbations in the first discovery of a planet by gravitational means \citep{Adams1847,LeVerrier1877}. In this case, they did not know the zeroth order solution (i.e. the Keplerian ellipse) for the perturber, Neptune.  In its place, they assumed the Titius-Bode rule held, and sought only the phase of the orbit.  This technique worked because they only wanted to see how the acceleration, then deceleration, of Uranus as it passed Neptune, would betray Neptune's position on the sky to optical observers. The task of discovering planets by TTV is more demanding.  We do not have any hints as to what the planet&#39;s orbit might be, i.e. we cannot assume it is on a circular orbit or obeys some spacing law. A single orbit is insufficient for a detection: times of least three transits are needed to determine a period change. However, due to measurement error, in only a small fraction of cases is the high-frequency ``chopping&#39;&#39; signal (see Chopping section below) statistically significant after just three transits.  Moreover, the sampling of the orbit only at transit phase causes aliasing of the dynamical signals.</td>
      </tr>
      <tr>
        <td id="L149" class="blob-num js-line-number" data-line-number="149"></td>
        <td id="LC149" class="blob-code blob-code-inner js-file-line">
</td>
      </tr>
      <tr>
        <td id="L150" class="blob-num js-line-number" data-line-number="150"></td>
        <td id="LC150" class="blob-code blob-code-inner js-file-line">The times of transit are primarily constrained by the decline of stellar flux during transit ingress, and the rise over egress, which occur on a timescale </td>
      </tr>
      <tr>
        <td id="L151" class="blob-num js-line-number" data-line-number="151"></td>
        <td id="LC151" class="blob-code blob-code-inner js-file-line"><span class="pl-c1">\begin</span>{equation} <span class="pl-c1">\label</span>{ingress}</td>
      </tr>
      <tr>
        <td id="L152" class="blob-num js-line-number" data-line-number="152"></td>
        <td id="LC152" class="blob-code blob-code-inner js-file-line"><span class="pl-c1">\tau</span> <span class="pl-c1">\approx</span> <span class="pl-c1">\pi</span>^{-1} P(R_p/a) <span class="pl-c1">\approx</span> 2.2 min <span class="pl-c1">\left</span>(<span class="pl-c1">\frac</span>{R_p}{R_<span class="pl-c1">\oplus</span>}<span class="pl-c1">\right</span>) <span class="pl-c1">\left</span>(<span class="pl-c1">\frac</span>{M_{\star}}{<span class="pl-c1">\rm</span>{M_{sun}}}<span class="pl-c1">\right</span>)^{-1/3} <span class="pl-c1">\left</span>(<span class="pl-c1">\frac</span>{P}{10 <span class="pl-c1">\rm</span>{d}}<span class="pl-c1">\right</span>)^{1/3},</td>
      </tr>
      <tr>
        <td id="L153" class="blob-num js-line-number" data-line-number="153"></td>
        <td id="LC153" class="blob-code blob-code-inner js-file-line"><span class="pl-c1">\end</span>{equation}  </td>
      </tr>
      <tr>
        <td id="L154" class="blob-num js-line-number" data-line-number="154"></td>
        <td id="LC154" class="blob-code blob-code-inner js-file-line">assuming an orbit edge-on to the line of sight (impact parameter of <span class="pl-s"><span class="pl-pds">$</span>b=<span class="pl-c1">0</span><span class="pl-pds">$</span></span>) around a star of mass $M_{\star}$; usually timing precision can be measured to better than this timescale.</td>
      </tr>
      <tr>
        <td id="L155" class="blob-num js-line-number" data-line-number="155"></td>
        <td id="LC155" class="blob-code blob-code-inner js-file-line">This timing precision gives a sensitive measure of the variation of the angular position of a planet relative to a Keplerian orbit.  In contrast, the other dynamical techniques rely on a signal spread through the orbital timescale <span class="pl-s"><span class="pl-pds">$</span>P<span class="pl-pds">$</span></span>, and thus the precision of the orbital phase is poorly constrained unless the measurements are of high precision or long duration (although these conditions have been achieved by pulsar timing in PSR 1257 +12 which detected a Great Inequality <span class="pl-c1">\citep</span>{1994Sci...264..538W} and by radial velocity in GJ 876 which detected resonant orbital precession <span class="pl-c1">\citep</span>{2001Laughlin}).</td>
      </tr>
      <tr>
        <td id="L156" class="blob-num js-line-number" data-line-number="156"></td>
        <td id="LC156" class="blob-code blob-code-inner js-file-line">
</td>
      </tr>
      <tr>
        <td id="L157" class="blob-num js-line-number" data-line-number="157"></td>
        <td id="LC157" class="blob-code blob-code-inner js-file-line">Orbital positions or transit times are expressed in a table called an ephemeris. Perturbations cause motions or timing deviations from a Keplerian reference model, especially changes to its instantaneous semimajor axis $a$, eccentricity $e$, and longitude of periastron $\omega$, the angle between the position of closest approach and a plane perpendicular to the line of sight that contains either the primary body or the center of mass.  In the case of transit timing variations, the Keplerian alternative is simply an ephemeris with a constant transit period, <span class="pl-s"><span class="pl-pds">$</span>P<span class="pl-pds">$</span></span>:</td>
      </tr>
      <tr>
        <td id="L158" class="blob-num js-line-number" data-line-number="158"></td>
        <td id="LC158" class="blob-code blob-code-inner js-file-line"><span class="pl-c1">\begin</span>{equation}</td>
      </tr>
      <tr>
        <td id="L159" class="blob-num js-line-number" data-line-number="159"></td>
        <td id="LC159" class="blob-code blob-code-inner js-file-line">C = T_0 + P <span class="pl-c1">\times</span> E, </td>
      </tr>
      <tr>
        <td id="L160" class="blob-num js-line-number" data-line-number="160"></td>
        <td id="LC160" class="blob-code blob-code-inner js-file-line"><span class="pl-c1">\end</span>{equation}</td>
      </tr>
      <tr>
        <td id="L161" class="blob-num js-line-number" data-line-number="161"></td>
        <td id="LC161" class="blob-code blob-code-inner js-file-line">where <span class="pl-s"><span class="pl-pds">$</span>E<span class="pl-pds">$</span></span> is the epoch -- an integer transit number -- and <span class="pl-s"><span class="pl-pds">$</span>T_<span class="pl-c1">0</span><span class="pl-pds">$</span></span> is the time of the transit numbered <span class="pl-s"><span class="pl-pds">$</span>E=<span class="pl-c1">0</span><span class="pl-pds">$</span></span>; <span class="pl-s"><span class="pl-pds">$</span>C<span class="pl-pds">$</span></span> stands for ``calculated&#39;&#39; based on a constant-period model.  Meanwhile, the Observed times of transit are denoted <span class="pl-s"><span class="pl-pds">$</span>O<span class="pl-pds">$</span></span>.  This notation leads to an <span class="pl-s"><span class="pl-pds">$</span>O-C<span class="pl-pds">$</span></span> (pronounced ``O minus C&#39;&#39;; <span class="pl-c1">\citealt</span>{2005Sterken}) diagram, in which only the perturbation part is plotted.  An instructive version, modelled after the timing of WASP-47 <span class="pl-c1">\citep</span>{2015Becker} but with a greatly exaggerated perturbation, is shown in figure~<span class="pl-c1">\ref</span>{omc}.  The transit times come earlier than the linear model for transit numbers 0-3 and 10-14, and later than the linear model for transit numbers 4-9.  These deviations from a constant transit period are what we call TTVs.</td>
      </tr>
      <tr>
        <td id="L162" class="blob-num js-line-number" data-line-number="162"></td>
        <td id="LC162" class="blob-code blob-code-inner js-file-line">
</td>
      </tr>
      <tr>
        <td id="L163" class="blob-num js-line-number" data-line-number="163"></td>
        <td id="LC163" class="blob-code blob-code-inner js-file-line"><span class="pl-c"><span class="pl-c">%</span> For figures use</span></td>
      </tr>
      <tr>
        <td id="L164" class="blob-num js-line-number" data-line-number="164"></td>
        <td id="LC164" class="blob-code blob-code-inner js-file-line"><span class="pl-c1">\begin</span>{figure}</td>
      </tr>
      <tr>
        <td id="L165" class="blob-num js-line-number" data-line-number="165"></td>
        <td id="LC165" class="blob-code blob-code-inner js-file-line"><span class="pl-c1">\centerline</span>{</td>
      </tr>
      <tr>
        <td id="L166" class="blob-num js-line-number" data-line-number="166"></td>
        <td id="LC166" class="blob-code blob-code-inner js-file-line"><span class="pl-c1">\includegraphics</span>[width=0.9<span class="pl-c1">\textwidth</span>]{omc.pdf}}</td>
      </tr>
      <tr>
        <td id="L167" class="blob-num js-line-number" data-line-number="167"></td>
        <td id="LC167" class="blob-code blob-code-inner js-file-line"><span class="pl-c"><span class="pl-c">%</span></span></td>
      </tr>
      <tr>
        <td id="L168" class="blob-num js-line-number" data-line-number="168"></td>
        <td id="LC168" class="blob-code blob-code-inner js-file-line"><span class="pl-c1">\caption</span>{An example of timing data.  <span class="pl-c1">\emph</span>{Top panel}: the measured midtimes of exoplanet transits, to which a line is fit by least-squares.  <span class="pl-c1">\emph</span>{Bottom panel}: the residuals of that fit, which is the conventional observed minus calculated (<span class="pl-s"><span class="pl-pds">$</span>O-C<span class="pl-pds">$</span></span>) diagram; the original sinusoidal function, to which Gaussian noise was added, is also plotted as a line. }</td>
      </tr>
      <tr>
        <td id="L169" class="blob-num js-line-number" data-line-number="169"></td>
        <td id="LC169" class="blob-code blob-code-inner js-file-line"><span class="pl-c1">\label</span>{omc}       <span class="pl-c"><span class="pl-c">%</span> Give a unique label</span></td>
      </tr>
      <tr>
        <td id="L170" class="blob-num js-line-number" data-line-number="170"></td>
        <td id="LC170" class="blob-code blob-code-inner js-file-line"><span class="pl-c1">\end</span>{figure}</td>
      </tr>
      <tr>
        <td id="L171" class="blob-num js-line-number" data-line-number="171"></td>
        <td id="LC171" class="blob-code blob-code-inner js-file-line">
</td>
      </tr>
      <tr>
        <td id="L172" class="blob-num js-line-number" data-line-number="172"></td>
        <td id="LC172" class="blob-code blob-code-inner js-file-line">The other dynamical effect addressed by this review is TDVs.  Like TTVs, the cause can be changes in $a$, $e$, or $\omega$.  The most dramatic effect, however, is due to orbital plane reorientation.  The angle the orbital plane&#39;s normal vector makes to the observer&#39;s line of sight --- the inclination --- determines the length of the transit chord.  Changes in the inclination will change the length of that chord, which in turn changes the amount of time the planet remains in transit: duration variations. </td>
      </tr>
      <tr>
        <td id="L173" class="blob-num js-line-number" data-line-number="173"></td>
        <td id="LC173" class="blob-code blob-code-inner js-file-line">
</td>
      </tr>
      <tr>
        <td id="L174" class="blob-num js-line-number" data-line-number="174"></td>
        <td id="LC174" class="blob-code blob-code-inner js-file-line">The literature on exoplanets has a history of rediscovering effects that had been well studied in the field of binary and multiple stars.  In the current focus, it has long been known to eclipsing-binary observers that long-term depth changes can result from the torque of a third star orbiting the pair <span class="pl-c1">\citep</span>{1971Mayer}.  This effect owes to the secular and tidal dynamics which dominate triple star systems <span class="pl-c1">\citep</span>{2003A&amp;A...398.1091B}, dictated by their hierarchical configuration which allows them to remain stable. TDV due to perturbing planets is simply its exoplanetary analogue<span class="pl-c1">\citep</span>{2002ApJ...564.1019M}.</td>
      </tr>
      <tr>
        <td id="L175" class="blob-num js-line-number" data-line-number="175"></td>
        <td id="LC175" class="blob-code blob-code-inner js-file-line">
</td>
      </tr>
      <tr>
        <td id="L176" class="blob-num js-line-number" data-line-number="176"></td>
        <td id="LC176" class="blob-code blob-code-inner js-file-line">The first recognition of the importance of transit timing and duration variations was at the DPS and AAS meetings two decades ago by <span class="pl-c1">\citet</span>{1996DPS....28.1208D,1996BAAS...28.1112D}, followed a few years later by <span class="pl-c1">\citet</span>{2002ApJ...564.1019M} and <span class="pl-c1">\citet</span>{Schneider2003,Schneider2004}.  More detailed studies that included the important effect of mean-motion resonance were independently investigated by <span class="pl-c1">\citet</span>{2005Sci...307.1288H} and <span class="pl-c1">\citet</span>{2005MNRAS.359..567A}.  The former paper showed that Solar-system like perturbations might be used to find Earth-like planets, should transit times be measured with sufficient accuracy.  The latter paper coined the term `transit-timing variations,&#39; with acronym TTV, and defined TTVs as the observable accumulation of transit period changes (i.e.<span class="pl-cce">\ </span><span class="pl-s"><span class="pl-pds">$</span>O-C<span class="pl-pds">$</span></span>).</td>
      </tr>
      <tr>
        <td id="L177" class="blob-num js-line-number" data-line-number="177"></td>
        <td id="LC177" class="blob-code blob-code-inner js-file-line">
</td>
      </tr>
      <tr>
        <td id="L178" class="blob-num js-line-number" data-line-number="178"></td>
        <td id="LC178" class="blob-code blob-code-inner js-file-line">Initial studies of TTVs of hot Jupiters were able to place limits on the presence of Earth-mass planets near mean-motion resonance <span class="pl-c1">\citep</span>{2005MNRAS.364L..96S}.  Some further studies claimed detection of perturbing planets causing TTVs or TDVs, but each of these were quickly disputed or refuted by additional measurements.  The first convincing detection awaited the launch of the \emph{Kepler} spacecraft, and the discovery of Kepler-9 which showed large-amplitude TTVs of two Saturn-sized planets with strong significance <span class="pl-c1">\citep</span>{2010Sci...330...51H}; this discovery was remarkably similar to predictions that had been made based upon the GJ 876 system <span class="pl-c1">\citep</span>{2005MNRAS.359..567A}.  The Kepler-9 paper kicked off a series of discoveries of TTVs with the \emph{Kepler} spacecraft, with now more than 100 systems displaying TTVs, and a handful showing TDVs <span class="pl-c1">\citep</span>{2016ApJS..225....9H}.</td>
      </tr>
      <tr>
        <td id="L179" class="blob-num js-line-number" data-line-number="179"></td>
        <td id="LC179" class="blob-code blob-code-inner js-file-line">
</td>
      </tr>
      <tr>
        <td id="L180" class="blob-num js-line-number" data-line-number="180"></td>
        <td id="LC180" class="blob-code blob-code-inner js-file-line"><span class="pl-c1">\section</span>{Preliminaries}</td>
      </tr>
      <tr>
        <td id="L181" class="blob-num js-line-number" data-line-number="181"></td>
        <td id="LC181" class="blob-code blob-code-inner js-file-line">
</td>
      </tr>
      <tr>
        <td id="L182" class="blob-num js-line-number" data-line-number="182"></td>
        <td id="LC182" class="blob-code blob-code-inner js-file-line">Since the gravitational interactions between planets occur on the orbital timescale, the</td>
      </tr>
      <tr>
        <td id="L183" class="blob-num js-line-number" data-line-number="183"></td>
        <td id="LC183" class="blob-code blob-code-inner js-file-line">amplitude of TTVs is proportional to the orbital period of each planet,</td>
      </tr>
      <tr>
        <td id="L184" class="blob-num js-line-number" data-line-number="184"></td>
        <td id="LC184" class="blob-code blob-code-inner js-file-line">times a function of other dimensionless quantities.  Thanks to Newton&#39;s second law</td>
      </tr>
      <tr>
        <td id="L185" class="blob-num js-line-number" data-line-number="185"></td>
        <td id="LC185" class="blob-code blob-code-inner js-file-line">and Newton&#39;s law of gravity, the acceleration of a body does not depend on its own mass.</td>
      </tr>
      <tr>
        <td id="L186" class="blob-num js-line-number" data-line-number="186"></td>
        <td id="LC186" class="blob-code blob-code-inner js-file-line">Thus, the TTVs of each planet scale with the masses of the {<span class="pl-c1">\it</span> other} bodies</td>
      </tr>
      <tr>
        <td id="L187" class="blob-num js-line-number" data-line-number="187"></td>
        <td id="LC187" class="blob-code blob-code-inner js-file-line">in the system.</td>
      </tr>
      <tr>
        <td id="L188" class="blob-num js-line-number" data-line-number="188"></td>
        <td id="LC188" class="blob-code blob-code-inner js-file-line">In a two-planet system, then, to lowest order in mass ratio, the <span class="pl-s"><span class="pl-pds">$</span>O-C<span class="pl-pds">$</span></span> amplitudes are: </td>
      </tr>
      <tr>
        <td id="L189" class="blob-num js-line-number" data-line-number="189"></td>
        <td id="LC189" class="blob-code blob-code-inner js-file-line"><span class="pl-c1">\begin</span>{eqnarray}</td>
      </tr>
      <tr>
        <td id="L190" class="blob-num js-line-number" data-line-number="190"></td>
        <td id="LC190" class="blob-code blob-code-inner js-file-line"><span class="pl-c1">\vert</span> <span class="pl-c1">\delta</span> t_1<span class="pl-c1">\vert</span> &amp;=&amp; P_1 <span class="pl-c1">\frac</span>{m_2}{m_0} f_{12}(<span class="pl-c1">\alpha</span>_{12},<span class="pl-c1">\mathbf</span>{<span class="pl-c1">\theta</span>}_{12}),<span class="pl-c1">\cr</span></td>
      </tr>
      <tr>
        <td id="L191" class="blob-num js-line-number" data-line-number="191"></td>
        <td id="LC191" class="blob-code blob-code-inner js-file-line"><span class="pl-c1">\vert</span> <span class="pl-c1">\delta</span> t_2<span class="pl-c1">\vert</span> &amp;=&amp; P_2 <span class="pl-c1">\frac</span>{m_1}{m_0} f_{21}(<span class="pl-c1">\alpha</span>_{12},<span class="pl-c1">\mathbf</span>{<span class="pl-c1">\theta</span>}_{21}),</td>
      </tr>
      <tr>
        <td id="L192" class="blob-num js-line-number" data-line-number="192"></td>
        <td id="LC192" class="blob-code blob-code-inner js-file-line"><span class="pl-c1">\end</span>{eqnarray}</td>
      </tr>
      <tr>
        <td id="L193" class="blob-num js-line-number" data-line-number="193"></td>
        <td id="LC193" class="blob-code blob-code-inner js-file-line">where the masses of the star and planets are <span class="pl-s"><span class="pl-pds">$</span>m_<span class="pl-c1">0</span>, m_<span class="pl-c1">1</span>,<span class="pl-pds">$</span></span> and <span class="pl-s"><span class="pl-pds">$</span>m_<span class="pl-c1">2</span><span class="pl-pds">$</span></span>, and <span class="pl-s"><span class="pl-pds">$</span>f_{ij}<span class="pl-pds">$</span></span> describes the perturbations of planet <span class="pl-s"><span class="pl-pds">$</span>j<span class="pl-pds">$</span></span> on planet <span class="pl-s"><span class="pl-pds">$</span>i<span class="pl-pds">$</span></span>,</td>
      </tr>
      <tr>
        <td id="L194" class="blob-num js-line-number" data-line-number="194"></td>
        <td id="LC194" class="blob-code blob-code-inner js-file-line">which is a function of the semi-major axis ratio, <span class="pl-s"><span class="pl-pds">$</span><span class="pl-c1">\alpha</span>_{ij} = {<span class="pl-c1">\rm</span> min}(a_i/a_j,a_j/a_i)<span class="pl-pds">$</span></span>, and the angular orbital </td>
      </tr>
      <tr>
        <td id="L195" class="blob-num js-line-number" data-line-number="195"></td>
        <td id="LC195" class="blob-code blob-code-inner js-file-line">elements of the planets, <span class="pl-s"><span class="pl-pds">$</span><span class="pl-c1">\mathbf</span>{<span class="pl-c1">\theta</span>}_{ij} = (<span class="pl-c1">\lambda</span>_i,e_i,<span class="pl-c1">\omega</span>_i,I_i,<span class="pl-c1">\Omega</span>_i,<span class="pl-c1">\lambda</span>_j,e_j,<span class="pl-c1">\omega</span>_j,I_j,<span class="pl-c1">\Omega</span>_j)<span class="pl-pds">$</span></span>.  The evaluation of these functions can be found in a series of papers on perturbation theory: <span class="pl-c1">\cite</span>{2008ApJ...688..636N, 2009ApJ...701.1116N,2010ApJ...709L..44N,2016ApJ...818..177A,2016ApJ...821...96D}.</td>
      </tr>
      <tr>
        <td id="L196" class="blob-num js-line-number" data-line-number="196"></td>
        <td id="LC196" class="blob-code blob-code-inner js-file-line">
</td>
      </tr>
      <tr>
        <td id="L197" class="blob-num js-line-number" data-line-number="197"></td>
        <td id="LC197" class="blob-code blob-code-inner js-file-line"><span class="pl-c"><span class="pl-c">%</span>  - Linear TTV (independently adds from different planets, off resonance)  - </span></td>
      </tr>
      <tr>
        <td id="L198" class="blob-num js-line-number" data-line-number="198"></td>
        <td id="LC198" class="blob-code blob-code-inner js-file-line">With the addition of multiple perturbing planets, if the mass-ratios of the planets to the star are</td>
      </tr>
      <tr>
        <td id="L199" class="blob-num js-line-number" data-line-number="199"></td>
        <td id="LC199" class="blob-code blob-code-inner js-file-line">sufficiently small and if none of the planets exist in a resonant configuration, then the TTVs may be approximately expressed as linear combinations of the perturbations due to each companion.</td>
      </tr>
      <tr>
        <td id="L201" class="blob-num js-line-number" data-line-number="201"></td>
        <td id="LC201" class="blob-code blob-code-inner js-file-line">For <span class="pl-s"><span class="pl-pds">$</span>N<span class="pl-pds">$</span></span> planets, the TTVs become</td>
      </tr>
      <tr>
        <td id="L202" class="blob-num js-line-number" data-line-number="202"></td>
        <td id="LC202" class="blob-code blob-code-inner js-file-line"><span class="pl-c1">\begin</span>{equation}</td>
      </tr>
      <tr>
        <td id="L203" class="blob-num js-line-number" data-line-number="203"></td>
        <td id="LC203" class="blob-code blob-code-inner js-file-line"><span class="pl-c1">\delta</span> t_i = P_i <span class="pl-c1">\sum</span>_{j <span class="pl-c1">\ne</span> i}  <span class="pl-c1">\frac</span>{m_j}{m_0} f_{ij}(<span class="pl-c1">\alpha</span>_{ij},<span class="pl-c1">\mathbf</span>{<span class="pl-c1">\theta</span>}_{ij}),</td>
      </tr>
      <tr>
        <td id="L204" class="blob-num js-line-number" data-line-number="204"></td>
        <td id="LC204" class="blob-code blob-code-inner js-file-line"><span class="pl-c1">\end</span>{equation}</td>
      </tr>
      <tr>
        <td id="L205" class="blob-num js-line-number" data-line-number="205"></td>
        <td id="LC205" class="blob-code blob-code-inner js-file-line">for <span class="pl-s"><span class="pl-pds">$</span>i=<span class="pl-c1">1</span>,...,N<span class="pl-pds">$</span></span>.</td>
      </tr>
      <tr>
        <td id="L207" class="blob-num js-line-number" data-line-number="207"></td>
        <td id="LC207" class="blob-code blob-code-inner js-file-line">
</td>
      </tr>
      <tr>
        <td id="L208" class="blob-num js-line-number" data-line-number="208"></td>
        <td id="LC208" class="blob-code blob-code-inner js-file-line">The largest TTVs are caused by orbital period changes associated with librations of the system about a mean-motion resonance, in which the ratio of two planets' orbital periods is close to the ratio of small integers.  We may appeal to energy trades to compute the amplitude of the TTV in each planet (see <span class="pl-c1">\citealt</span>{2005MNRAS.359..567A,2010Sci...330...51H}).  Because of Kepler&#39;s relation <span class="pl-s"><span class="pl-pds">$</span>a <span class="pl-c1">\propto</span> P^{3/2}<span class="pl-pds">$</span></span>, a period lengthening of <span class="pl-s"><span class="pl-pds">$</span><span class="pl-c1">\delta</span> P_<span class="pl-c1">1</span> <span class="pl-c1">\ll</span> P_<span class="pl-c1">1</span><span class="pl-pds">$</span></span> is associated with a semi-major axis change of <span class="pl-s"><span class="pl-pds">$</span><span class="pl-c1">\delta</span> a_<span class="pl-c1">1</span> = (<span class="pl-c1">3</span>/<span class="pl-c1">2</span>) a_<span class="pl-c1">1</span> <span class="pl-c1">\delta</span> P_<span class="pl-c1">1</span> / P_<span class="pl-c1">1</span><span class="pl-pds">$</span></span>.  Differentiating the orbital energy equation <span class="pl-s"><span class="pl-pds">$</span>E_<span class="pl-c1">1</span>=-G M m_<span class="pl-c1">1</span> /(<span class="pl-c1">2</span>a_<span class="pl-c1">1</span>)<span class="pl-pds">$</span></span> shows that such a change results in an energy change of <span class="pl-s"><span class="pl-pds">$</span><span class="pl-c1">\delta</span> E_<span class="pl-c1">1</span>=(GMm_<span class="pl-c1">1</span> a_<span class="pl-c1">1</span>^{-2}/<span class="pl-c1">2</span>) <span class="pl-c1">\delta</span> a_<span class="pl-c1">1</span><span class="pl-pds">$</span></span>.  To conserve total energy, the other planet will have an energy change of <span class="pl-s"><span class="pl-pds">$</span>E_<span class="pl-c1">2</span>=-(GMm_<span class="pl-c1">1</span> a_<span class="pl-c1">1</span>^{-2}/<span class="pl-c1">2</span>) <span class="pl-c1">\delta</span> a_<span class="pl-c1">1</span><span class="pl-pds">$</span></span>, which can also be expressed as <span class="pl-s"><span class="pl-pds">$</span>+(GMm_<span class="pl-c1">2</span> a_<span class="pl-c1">2</span>^{-2}/<span class="pl-c1">2</span>)<span class="pl-c1">\delta</span> a_<span class="pl-c1">2</span><span class="pl-pds">$</span></span>.  Using the relation <span class="pl-s"><span class="pl-pds">$</span><span class="pl-c1">\delta</span> a_<span class="pl-c1">2</span> = (<span class="pl-c1">3</span>/<span class="pl-c1">2</span>) a_<span class="pl-c1">2</span> <span class="pl-c1">\delta</span> P_<span class="pl-c1">2</span> / P_<span class="pl-c1">2</span><span class="pl-pds">$</span></span>, and the Keplerian relation <span class="pl-s"><span class="pl-pds">$</span>a_<span class="pl-c1">2</span>/a_<span class="pl-c1">1</span>=(P_<span class="pl-c1">2</span>/P_<span class="pl-c1">1</span>)^{2/3}<span class="pl-pds">$</span></span>, we obtain: </td>
      </tr>
      <tr>
        <td id="L209" class="blob-num js-line-number" data-line-number="209"></td>
        <td id="LC209" class="blob-code blob-code-inner js-file-line"><span class="pl-c1">\begin</span>{equation}</td>
      </tr>
      <tr>
        <td id="L210" class="blob-num js-line-number" data-line-number="210"></td>
        <td id="LC210" class="blob-code blob-code-inner js-file-line"><span class="pl-c1">\delta</span> P_2 = -<span class="pl-c1">\delta</span> P_1 (m_1/m_2) (P_2/P_1)^{5/3}. <span class="pl-c1">\label</span>{eqn:deltaP}</td>
      </tr>
      <tr>
        <td id="L211" class="blob-num js-line-number" data-line-number="211"></td>
        <td id="LC211" class="blob-code blob-code-inner js-file-line"><span class="pl-c1">\end</span>{equation}</td>
      </tr>
      <tr>
        <td id="L212" class="blob-num js-line-number" data-line-number="212"></td>
        <td id="LC212" class="blob-code blob-code-inner js-file-line">When considering the <span class="pl-s"><span class="pl-pds">$</span>O-C<span class="pl-pds">$</span></span> shapes that each planet makes over a fixed time interval (e.g. from a survey that measures transits for both planets), we will have a factor of <span class="pl-s"><span class="pl-pds">$</span>P_<span class="pl-c1">2</span>/P_<span class="pl-c1">1</span><span class="pl-pds">$</span></span> more orbital periods for the inner planet than the outer planet.  Thus the accumulated time shift of the signal, <span class="pl-s"><span class="pl-pds">$</span><span class="pl-c1">\delta</span> t<span class="pl-pds">$</span></span>, builds up more for the inner planet, by one factor of the period ratio. In consideration of equation~<span class="pl-c1">\ref</span>{eqn:deltaP}, we are left with: </td>
      </tr>
      <tr>
        <td id="L213" class="blob-num js-line-number" data-line-number="213"></td>
        <td id="LC213" class="blob-code blob-code-inner js-file-line"><span class="pl-c1">\begin</span>{equation}</td>
      </tr>
      <tr>
        <td id="L214" class="blob-num js-line-number" data-line-number="214"></td>
        <td id="LC214" class="blob-code blob-code-inner js-file-line"><span class="pl-c1">\delta</span> t_2 = -<span class="pl-c1">\delta</span> t_1 (m_1/m_2) (P_2/P_1)^{2/3}. <span class="pl-c1">\label</span>{eqn:deltat}</td>
      </tr>
      <tr>
        <td id="L215" class="blob-num js-line-number" data-line-number="215"></td>
        <td id="LC215" class="blob-code blob-code-inner js-file-line"><span class="pl-c1">\end</span>{equation}</td>
      </tr>
      <tr>
        <td id="L216" class="blob-num js-line-number" data-line-number="216"></td>
        <td id="LC216" class="blob-code blob-code-inner js-file-line">This scaling agrees with analytic work performed in the resonant <span class="pl-c1">\citep</span>{2016ApJ...823...72N} and near-resonant <span class="pl-c1">\citep</span>{2012ApJ...761..122L,2016ApJ...828...44H} regimes. Hence the TTV curves of the two planets are anti-correlated, and in the case that the masses are equal, the amplitude of the outer planet&#39;s TTV is larger because its orbital size needs to change more for its Keplerian orbital energy to equal the change in the inner planet's Keplerian orbital energy. </td>
      </tr>
      <tr>
        <td id="L217" class="blob-num js-line-number" data-line-number="217"></td>
        <td id="LC217" class="blob-code blob-code-inner js-file-line">
</td>
      </tr>
      <tr>
        <td id="L221" class="blob-num js-line-number" data-line-number="221"></td>
        <td id="LC221" class="blob-code blob-code-inner js-file-line">The two observables associated with a light curve are the time stamp of each photometric</td>
      </tr>
      <tr>
        <td id="L222" class="blob-num js-line-number" data-line-number="222"></td>
        <td id="LC222" class="blob-code blob-code-inner js-file-line">measurement and the number of photons measured.  The number of photons is a dimensionless</td>
      </tr>
      <tr>
        <td id="L223" class="blob-num js-line-number" data-line-number="223"></td>
        <td id="LC223" class="blob-code blob-code-inner js-file-line">number, and thus may only constrain dimensionless quantities, such as radius ratio, impact </td>
      </tr>
      <tr>
        <td id="L224" class="blob-num js-line-number" data-line-number="224"></td>
        <td id="LC224" class="blob-code blob-code-inner js-file-line">parameter, or the ratio of the stellar size to the semi-major axis.  The quantities that </td>
      </tr>
      <tr>
        <td id="L225" class="blob-num js-line-number" data-line-number="225"></td>
        <td id="LC225" class="blob-code blob-code-inner js-file-line">have units of time --- the period, transit duration, ingress duration ---  can </td>
      </tr>
      <tr>
        <td id="L226" class="blob-num js-line-number" data-line-number="226"></td>
        <td id="LC226" class="blob-code blob-code-inner js-file-line">constrain the density of the system since the dynamical time relates to stellar density, <span class="pl-s"><span class="pl-pds">$</span><span class="pl-c1">\rho</span><span class="pl-pds">$</span></span>, as</td>
      </tr>
      <tr>
        <td id="L227" class="blob-num js-line-number" data-line-number="227"></td>
        <td id="LC227" class="blob-code blob-code-inner js-file-line"><span class="pl-s"><span class="pl-pds">$</span>t_{dyn} <span class="pl-c1">\approx</span> (G<span class="pl-c1">\rho</span>)^{-1/2}<span class="pl-pds">$</span></span>.  <span class="pl-c1">\citet</span>{2003ApJ...585.1038S} showed that a single transiting planet</td>
      </tr>
      <tr>
        <td id="L228" class="blob-num js-line-number" data-line-number="228"></td>
        <td id="LC228" class="blob-code blob-code-inner js-file-line">on a well-measured circular orbit may be used to gauge the density of the star;</td>
      </tr>
      <tr>
        <td id="L229" class="blob-num js-line-number" data-line-number="229"></td>
        <td id="LC229" class="blob-code blob-code-inner js-file-line">in the case of multiple transiting planets, the circular assumption may be relaxed</td>
      </tr>
      <tr>
        <td id="L230" class="blob-num js-line-number" data-line-number="230"></td>
        <td id="LC230" class="blob-code blob-code-inner js-file-line"><span class="pl-c1">\citep</span>{2014MNRAS.440.2164K}.</td>
      </tr>
      <tr>
        <td id="L231" class="blob-num js-line-number" data-line-number="231"></td>
        <td id="LC231" class="blob-code blob-code-inner js-file-line">
</td>
      </tr>
      <tr>
        <td id="L232" class="blob-num js-line-number" data-line-number="232"></td>
        <td id="LC232" class="blob-code blob-code-inner js-file-line">The transit depth, then, gives the radius-ratio of the planet to the star, while if two planets</td>
      </tr>
      <tr>
        <td id="L233" class="blob-num js-line-number" data-line-number="233"></td>
        <td id="LC233" class="blob-code blob-code-inner js-file-line">transit and show TTVs, their TTVs give an estimate of the mass ratio of the perturbing planet</td>
      </tr>
      <tr>
        <td id="L234" class="blob-num js-line-number" data-line-number="234"></td>
        <td id="LC234" class="blob-code blob-code-inner js-file-line">to the star.  Thus, two transiting, interacting planets yield an estimate of the density ratio of</td>
      </tr>
      <tr>
        <td id="L235" class="blob-num js-line-number" data-line-number="235"></td>
        <td id="LC235" class="blob-code blob-code-inner js-file-line">the planets to the star, and consequently we can obtain the density of the planets.</td>
      </tr>
      <tr>
        <td id="L236" class="blob-num js-line-number" data-line-number="236"></td>
        <td id="LC236" class="blob-code blob-code-inner js-file-line">Note that this is true even if the absolute mass and radius of the star are poorly</td>
      </tr>
      <tr>
        <td id="L237" class="blob-num js-line-number" data-line-number="237"></td>
        <td id="LC237" class="blob-code blob-code-inner js-file-line">constrained.  A caveat to this technique is that there is an eccentricity dependence that </td>
      </tr>
      <tr>
        <td id="L238" class="blob-num js-line-number" data-line-number="238"></td>
        <td id="LC238" class="blob-code blob-code-inner js-file-line">is present in stellar density estimate.  However, typically multi-transiting planet systems require low eccentricities to be stable,</td>
      </tr>
      <tr>
        <td id="L239" class="blob-num js-line-number" data-line-number="239"></td>
        <td id="LC239" class="blob-code blob-code-inner js-file-line">and in some cases the eccentricities can be constrained sufficiently from TTV analysis, from</td>
      </tr>
      <tr>
        <td id="L240" class="blob-num js-line-number" data-line-number="240"></td>
        <td id="LC240" class="blob-code blob-code-inner js-file-line">analyzing multiple planets <span class="pl-c1">\citep</span>{2014MNRAS.440.2164K}, or</td>
      </tr>
      <tr>
        <td id="L241" class="blob-num js-line-number" data-line-number="241"></td>
        <td id="LC241" class="blob-code blob-code-inner js-file-line">from statistical analysis of an ensemble of planets <span class="pl-c1">\citep</span>{Hadden2017}.  So this caveat ends up not impacting the stellar density </td>
      </tr>
      <tr>
        <td id="L242" class="blob-num js-line-number" data-line-number="242"></td>
        <td id="LC242" class="blob-code blob-code-inner js-file-line">estimate significantly (the mass-eccentricity degeneracy, however, reduces precision on planet-star mass ratios, and hence inflates the planet density uncertainty). </td>
      </tr>
      <tr>
        <td id="L243" class="blob-num js-line-number" data-line-number="243"></td>
        <td id="LC243" class="blob-code blob-code-inner js-file-line">Another way to obtain an estimate of stellar density is from asteroseismology:</td>
      </tr>
      <tr>
        <td id="L244" class="blob-num js-line-number" data-line-number="244"></td>
        <td id="LC244" class="blob-code blob-code-inner js-file-line">in fact, the time-dependence of asteroseismic measurements is what enables density</td>
      </tr>
      <tr>
        <td id="L245" class="blob-num js-line-number" data-line-number="245"></td>
        <td id="LC245" class="blob-code blob-code-inner js-file-line">to be constrained in that case as well <span class="pl-c1">\citep</span>{1986ApJ...306L..37U}.</td>
      </tr>
      <tr>
        <td id="L246" class="blob-num js-line-number" data-line-number="246"></td>
        <td id="LC246" class="blob-code blob-code-inner js-file-line">
</td>
      </tr>
      <tr>
        <td id="L247" class="blob-num js-line-number" data-line-number="247"></td>
        <td id="LC247" class="blob-code blob-code-inner js-file-line">If a pair of transiting exoplanets can be detected with {<span class="pl-c1">\it</span> both} TTVs and RVs, then the</td>
      </tr>
      <tr>
        <td id="L248" class="blob-num js-line-number" data-line-number="248"></td>
        <td id="LC248" class="blob-code blob-code-inner js-file-line">absolute dimensions of the system may be obtained <span class="pl-c1">\citep</span>{2005MNRAS.359..567A,</td>
      </tr>
      <tr>
        <td id="L249" class="blob-num js-line-number" data-line-number="249"></td>
        <td id="LC249" class="blob-code blob-code-inner js-file-line">2013ApJ...762..112M} as RVs have a dimensions of velocity, which </td>
      </tr>
      <tr>
        <td id="L250" class="blob-num js-line-number" data-line-number="250"></td>
        <td id="LC250" class="blob-code blob-code-inner js-file-line">when combined with time measurements from TTVs gives dimensions of distance.</td>
      </tr>
      <tr>
        <td id="L251" class="blob-num js-line-number" data-line-number="251"></td>
        <td id="LC251" class="blob-code blob-code-inner js-file-line">In practice this technique has yet to yield useful constraints upon the properties</td>
      </tr>
      <tr>
        <td id="L252" class="blob-num js-line-number" data-line-number="252"></td>
        <td id="LC252" class="blob-code blob-code-inner js-file-line">of planetary systems <span class="pl-c1">\citep</span>{2015MNRAS.453.2644A}, but it may prove fruitful</td>
      </tr>
      <tr>
        <td id="L253" class="blob-num js-line-number" data-line-number="253"></td>
        <td id="LC253" class="blob-code blob-code-inner js-file-line">in the future much as double-lined spectroscopic binaries have used to measuring </td>
      </tr>
      <tr>
        <td id="L254" class="blob-num js-line-number" data-line-number="254"></td>
        <td id="LC254" class="blob-code blob-code-inner js-file-line">the properties of binary stars, as hinted at by <span class="pl-c1">\cite</span>{2016A&amp;A...595L...5A}.  Circumbinary planets </td>
      </tr>
      <tr>
        <td id="L255" class="blob-num js-line-number" data-line-number="255"></td>
        <td id="LC255" class="blob-code blob-code-inner js-file-line">(CBP) are an extreme example of this technique: the timing offsets of the transits, combined with the eclipses</td>
      </tr>
      <tr>
        <td id="L256" class="blob-num js-line-number" data-line-number="256"></td>
        <td id="LC256" class="blob-code blob-code-inner js-file-line">and radial-velocity of the binary give very precise constraints on the absolute parameters</td>
      </tr>
      <tr>
        <td id="L257" class="blob-num js-line-number" data-line-number="257"></td>
        <td id="LC257" class="blob-code blob-code-inner js-file-line">of the Kepler-16 system <span class="pl-c1">\citep</span>{2011Sci...333.1602D}.</td>
      </tr>
      <tr>
        <td id="L258" class="blob-num js-line-number" data-line-number="258"></td>
        <td id="LC258" class="blob-code blob-code-inner js-file-line">
</td>
      </tr>
      <tr>
        <td id="L259" class="blob-num js-line-number" data-line-number="259"></td>
        <td id="LC259" class="blob-code blob-code-inner js-file-line"><span class="pl-c1">\section</span>{Theory and Paradigmatic Examples} </td>
      </tr>
      <tr>
        <td id="L260" class="blob-num js-line-number" data-line-number="260"></td>
        <td id="LC260" class="blob-code blob-code-inner js-file-line">
</td>
      </tr>
      <tr>
        <td id="L261" class="blob-num js-line-number" data-line-number="261"></td>
        <td id="LC261" class="blob-code blob-code-inner js-file-line">Here we discuss the physical models for different types of TTV interactions, and point the reader to real systems that exhibit that kind of interaction. </td>
      </tr>
      <tr>
        <td id="L262" class="blob-num js-line-number" data-line-number="262"></td>
        <td id="LC262" class="blob-code blob-code-inner js-file-line">
</td>
      </tr>
      <tr>
        <td id="L263" class="blob-num js-line-number" data-line-number="263"></td>
        <td id="LC263" class="blob-code blob-code-inner js-file-line">Close to resonances, a combination of changes in semi-major axis and eccentricity lead to TTV cycles whose period depends on the separation from the resonance <span class="pl-c1">\citep</span>{2012ApJ...761..122L}.  The main TTV variation comes from only one resonance, the one the system is closest to, which allows its critical angles to move slowly and thus its effect to build up.  If the period ratio <span class="pl-s"><span class="pl-pds">$</span>P_<span class="pl-c1">2</span>/P_<span class="pl-c1">1</span><span class="pl-pds">$</span></span> is within a few percent of the ratio <span class="pl-s"><span class="pl-pds">$</span>j/k<span class="pl-pds">$</span></span>, with <span class="pl-s"><span class="pl-pds">$</span>j<span class="pl-pds">$</span></span> and <span class="pl-s"><span class="pl-pds">$</span>k<span class="pl-pds">$</span></span> being integers, then the expected TTV period is </td>
      </tr>
      <tr>
        <td id="L264" class="blob-num js-line-number" data-line-number="264"></td>
        <td id="LC264" class="blob-code blob-code-inner js-file-line"><span class="pl-c1">\begin</span>{equation}</td>
      </tr>
      <tr>
        <td id="L265" class="blob-num js-line-number" data-line-number="265"></td>
        <td id="LC265" class="blob-code blob-code-inner js-file-line">P_{<span class="pl-c1">\rm</span> TTV} = 1/|j/P_2-k/P_1|. <span class="pl-c1">\label</span>{eqn:pttv}</td>
      </tr>
      <tr>
        <td id="L266" class="blob-num js-line-number" data-line-number="266"></td>
        <td id="LC266" class="blob-code blob-code-inner js-file-line"><span class="pl-c1">\end</span>{equation}</td>
      </tr>
      <tr>
        <td id="L267" class="blob-num js-line-number" data-line-number="267"></td>
        <td id="LC267" class="blob-code blob-code-inner js-file-line">The order of the resonance is <span class="pl-s"><span class="pl-pds">$</span>|j-k|<span class="pl-pds">$</span></span>, and the strength of the resonance depends on the planetary eccentricities to a power of the order minus 1.  Therefore, first order resonances affect planets with no initial eccentricity, but higher order resonances have a large effect only in the presence of some eccentricity. </td>
      </tr>
      <tr>
        <td id="L268" class="blob-num js-line-number" data-line-number="268"></td>
        <td id="LC268" class="blob-code blob-code-inner js-file-line">
</td>
      </tr>
      <tr>
        <td id="L269" class="blob-num js-line-number" data-line-number="269"></td>
        <td id="LC269" class="blob-code blob-code-inner js-file-line">Seeing two planets transit the star helps immensely to characterize a near-resonant system, because then the relative transit phase of the two planets can be compared with the phase of the TTV signals \citep{2012ApJ...761..122L}.  If the eccentricities are maximally damped out, then the resonant terms of the interaction continue forcing a small eccentricity that quickly precesses, causing the TTV.  In that case, the phase of the signal is predictable, and the two planets&#39; eccentricities are anti-aligned, so the TTV signals consist of anti-corrlated sinusoids.  Also useful in that case is that the amplitudes lead directly to the planetary masses.  If so-called ``free eccentricity&#39;&#39; remains, however, the phases would usually differ from that prediction, the TTV in the two planets may not be in perfect anti-phase, and only an approximate mass scale rather than a measurement is available, which is referred to as the mass-eccentricity degeneracy.  The first real system that showed this pattern convincingly was Kepler-18 \citep{2011ApJS..197....7C}. %[figure of that rather than the theory one given currently in figure 2?]. </td>
      </tr>
      <tr>
        <td id="L270" class="blob-num js-line-number" data-line-number="270"></td>
        <td id="LC270" class="blob-code blob-code-inner js-file-line">
</td>
      </tr>
      <tr>
        <td id="L271" class="blob-num js-line-number" data-line-number="271"></td>
        <td id="LC271" class="blob-code blob-code-inner js-file-line">The measurement of TTVs and TDVs has been used for confirmation, detection, and characterization of</td>
      </tr>
      <tr>
        <td id="L272" class="blob-num js-line-number" data-line-number="272"></td>
        <td id="LC272" class="blob-code blob-code-inner js-file-line">transiting exoplanets and their companions.  The \emph{Kepler} spacecraft discovered thousands of transiting</td>
      </tr>
      <tr>
        <td id="L273" class="blob-num js-line-number" data-line-number="273"></td>
        <td id="LC273" class="blob-code blob-code-inner js-file-line">exoplanet candidates;  the classification as `candidate&#39; was cautiously used to allow for other</td>
      </tr>
      <tr>
        <td id="L274" class="blob-num js-line-number" data-line-number="274"></td>
        <td id="LC274" class="blob-code blob-code-inner js-file-line">possible explanations, such as a blend of a foreground star and a background eclipsing binary causing</td>
      </tr>
      <tr>
        <td id="L275" class="blob-num js-line-number" data-line-number="275"></td>
        <td id="LC275" class="blob-code blob-code-inner js-file-line">an apparent transit-like signal.  The presence of multiple transiting planets around the same star</td>
      </tr>
      <tr>
        <td id="L276" class="blob-num js-line-number" data-line-number="276"></td>
        <td id="LC276" class="blob-code blob-code-inner js-file-line">gave a means of confirming two planets that display {<span class="pl-c1">\em</span> anti-correlated} TTVs: due to energy conservation (equation~<span class="pl-c1">\ref</span>{eqn:deltat}), the anti-correlation indicates dynamical interactions between the</td>
      </tr>
      <tr>
        <td id="L277" class="blob-num js-line-number" data-line-number="277"></td>
        <td id="LC277" class="blob-code blob-code-inner js-file-line">two planets, while such a configuration would not be stable for a triple star system.  Many papers </td>
      </tr>
      <tr>
        <td id="L278" class="blob-num js-line-number" data-line-number="278"></td>
        <td id="LC278" class="blob-code blob-code-inner js-file-line">used this technique to confirm that \emph{Kepler} planet candidates were bonafide exoplanets, starting with a series of three using different techniques to identify the anticorrelation in data <span class="pl-c1">\citet</span>{2011ApJS..197....2F, 2012ApJ...750..113F, 2012ApJ...750..114F}.</td>
      </tr>
      <tr>
        <td id="L279" class="blob-num js-line-number" data-line-number="279"></td>
        <td id="LC279" class="blob-code blob-code-inner js-file-line">
</td>
      </tr>
      <tr>
        <td id="L280" class="blob-num js-line-number" data-line-number="280"></td>
        <td id="LC280" class="blob-code blob-code-inner js-file-line">The characterization of exoplanets with TTVs also began in earnest with the \emph{Kepler} spacecraft.</td>
      </tr>
      <tr>
        <td id="L281" class="blob-num js-line-number" data-line-number="281"></td>
        <td id="LC281" class="blob-code blob-code-inner js-file-line">In addition to Kepler-9, the Kepler-18 system was characterized by a combination of TTVs and</td>
      </tr>
      <tr>
        <td id="L282" class="blob-num js-line-number" data-line-number="282"></td>
        <td id="LC282" class="blob-code blob-code-inner js-file-line">RVs, giving density estimates for the three transiting planets <span class="pl-c1">\citep</span>{2011ApJS..197....7C} and assuring that the new method for mass characterization gave the same answers as the trusted, older method.</td>
      </tr>
      <tr>
        <td id="L283" class="blob-num js-line-number" data-line-number="283"></td>
        <td id="LC283" class="blob-code blob-code-inner js-file-line">
</td>
      </tr>
      <tr>
        <td id="L284" class="blob-num js-line-number" data-line-number="284"></td>
        <td id="LC284" class="blob-code blob-code-inner js-file-line">When only one planet transits in a near-resonant system, the measured TTVs may simply record a sinusoidal signal, which could result from the other planet being close to many different resonances with the transiting planet.  In Kepler-19, <span class="pl-c1">\cite</span>{2011ApJ...743..200B} were able to tell that a planetary companion was the only sensible cause of the TTV, but they were not able to break this finite set of degeneracies. </td>
      </tr>
      <tr>
        <td id="L285" class="blob-num js-line-number" data-line-number="285"></td>
        <td id="LC285" class="blob-code blob-code-inner js-file-line">
</td>
      </tr>
      <tr>
        <td id="L286" class="blob-num js-line-number" data-line-number="286"></td>
        <td id="LC286" class="blob-code blob-code-inner js-file-line">This degeneracy has made it extremely difficult to characterize non-transiting planets via TTV, and hence in many cases an additional planet is suspected due to TTV, but detailed work has not been pursued to determine its nature.  The first case of a non-transiting planet being discovered \emph{and} completely characterized was Kepler-47 (a.k.a. KOI-872; \citealt{2012Sci...336.1133N}).  The authors found that the TTVs of the transiting planet were far from a sinusoidal shape; in fact, they could be fourier-decomposed into at least four significant sinusoids.  Each of these sinusoids can be identified as the interaction with the non-transiting planet via a different resonance.  Even with all this extra information, TTVs could only narrow down the possible perturbing planets to a degenerate set of two.  The clever solution \citep{2012Sci...336.1133N} was to note that one of those solutions, to get the relative amplitudes of the component sinusoids correct,  requires the perturbing planet to be somewhat inclined with respect to the transiting planet.  As a consequence, a torque on that planet would drive TDV.  No such TDV were observed, so the unique solution was found. </td>
      </tr>
      <tr>
        <td id="L287" class="blob-num js-line-number" data-line-number="287"></td>
        <td id="LC287" class="blob-code blob-code-inner js-file-line">
</td>
      </tr>
      <tr>
        <td id="L288" class="blob-num js-line-number" data-line-number="288"></td>
        <td id="LC288" class="blob-code blob-code-inner js-file-line">Planets that are truly in resonance with each other have the largest TTV signals.  On a medium-baseline timescale like that of \emph{Kepler}, they can perturb each other&#39;s orbital periods.  The resonant interaction traps the planets at a specific period ratio, causing the periods to oscillate near that ratio.  The period of the full cycle of that oscillation depends on the ratio of the planet masses to the host star&#39;s mass, to the $-\nicefrac{2}{3}$ power \citep{2005MNRAS.359..567A,2016ApJ...823...72N}.  For instance, the touchstone system GJ876 has a 550 day libration cycle, about 10 times the outer planet&#39;s period, due to its relatively massive planets and low-mass star.  A system which was characterized by resonant interaction is KOI-142 \citep{2013ApJ...777....3N}, in which a non-transiting planet was discovered.   A system with two transiting planets in resonance with large TTVs is Kepler-30 \citep{2012ApJ...750..114F}.  A system with smaller libration amplitudes, but a surprising \emph{four} planets in resonance (forming a chain of resonances) is Kepler-223 \citep{2016Natur.533..509M}.</td>
      </tr>
      <tr>
        <td id="L289" class="blob-num js-line-number" data-line-number="289"></td>
        <td id="LC289" class="blob-code blob-code-inner js-file-line">
</td>
      </tr>
      <tr>
        <td id="L290" class="blob-num js-line-number" data-line-number="290"></td>
        <td id="LC290" class="blob-code blob-code-inner js-file-line">Several other TTV mechanisms have been detected which do not rely on resonances, but are relevant for more hierarchical situations (<span class="pl-s"><span class="pl-pds">$</span>P_<span class="pl-c1">2</span>/P_<span class="pl-c1">1</span> <span class="pl-c1">\gtrsim</span> <span class="pl-c1">4</span><span class="pl-pds">$</span></span>). </td>
      </tr>
      <tr>
        <td id="L291" class="blob-num js-line-number" data-line-number="291"></td>
        <td id="LC291" class="blob-code blob-code-inner js-file-line">
</td>
      </tr>
      <tr>
        <td id="L292" class="blob-num js-line-number" data-line-number="292"></td>
        <td id="LC292" class="blob-code blob-code-inner js-file-line">If the outer planet transits, and the inner orbiting body is very massive, the dominant effect can be the shifting of the primary star with respect to the barycenter.   Then, as the outer planet orbits the barycenter, it arrives at the moving target either early or late.  This effect was numbered (i) by <span class="pl-c1">\cite</span>{2005MNRAS.359..567A}, and it is seen clearly in circumbinary planet systems.  For instance, the secondary star of Kepler-16 <span class="pl-c1">\citep</span>{2011Sci...333.1602D} moves the primary by many times its own radius, resulting in an <span class="pl-s"><span class="pl-pds">$</span><span class="pl-c1">\sim</span> <span class="pl-c1">8</span><span class="pl-pds">$</span></span>~day TTV on top of a 225 day orbit. </td>
      </tr>
      <tr>
        <td id="L293" class="blob-num js-line-number" data-line-number="293"></td>
        <td id="LC293" class="blob-code blob-code-inner js-file-line">
</td>
      </tr>
      <tr>
        <td id="L294" class="blob-num js-line-number" data-line-number="294"></td>
        <td id="LC294" class="blob-code blob-code-inner js-file-line">A final mechanism of dynamical TTV is relevant for the inner orbit when a massive body orbits at large distance.  The tide that body exerts on the inner orbit causes its orbital period to differ slightly from what it would be in the absence of that outer body.  If the outer body is in the plane of the inner orbit, its tide slows down the inner orbit, lengthening its period.  If the outer body is far out of the plane of the inner orbit, its tide speeds up the inner orbit, shortening its period.  The tide also depends on the third power of the distance to that external body.  Hence, when the external body moves on an eccentric and/or inclined orbit, it induces a period variation in the inner orbit, which has the period of the outer orbit.  Also imporant for the timing is how the outer perturber instantaneously torques the inner orbit&#39;s eccentricity.  These effects were put together and analyzed by \cite{2003A&amp;A...398.1091B} in the context of triple star systems, and the in-plane physics was explained clearly as mechanism (ii) of \cite{2005MNRAS.359..567A}.  An example of these effects was provided by Kepler-419 \citep{2014ApJ...791...89D}, in which an eccentric massive planet accompanies an inner planet with a period ratio of 9.7.</td>
      </tr>
      <tr>
        <td id="L295" class="blob-num js-line-number" data-line-number="295"></td>
        <td id="LC295" class="blob-code blob-code-inner js-file-line">
</td>
      </tr>
      <tr>
        <td id="L296" class="blob-num js-line-number" data-line-number="296"></td>
        <td id="LC296" class="blob-code blob-code-inner js-file-line"><span class="pl-c1">\subsection</span>{Chopping}</td>
      </tr>
      <tr>
        <td id="L297" class="blob-num js-line-number" data-line-number="297"></td>
        <td id="LC297" class="blob-code blob-code-inner js-file-line">
</td>
      </tr>
      <tr>
        <td id="L298" class="blob-num js-line-number" data-line-number="298"></td>
        <td id="LC298" class="blob-code blob-code-inner js-file-line">When two planets are nearly resonant, the degeneracy between the mass ratios of the planets to the star</td>
      </tr>
      <tr>
        <td id="L299" class="blob-num js-line-number" data-line-number="299"></td>
        <td id="LC299" class="blob-code blob-code-inner js-file-line">and the eccentricity vector may be broken by examining additional harmonics present in the data <span class="pl-c1">\citep</span>{2015ApJ...802..116D}.</td>
      </tr>
      <tr>
        <td id="L300" class="blob-num js-line-number" data-line-number="300"></td>
        <td id="LC300" class="blob-code blob-code-inner js-file-line">These additional harmonics have a smaller amplitude due to the fact that they are not close to resonance,</td>
      </tr>
      <tr>
        <td id="L301" class="blob-num js-line-number" data-line-number="301"></td>
        <td id="LC301" class="blob-code blob-code-inner js-file-line">and thus require higher signal-to-noise to break the degeneracy between the mass and eccentricity.</td>
      </tr>
      <tr>
        <td id="L302" class="blob-num js-line-number" data-line-number="302"></td>
        <td id="LC302" class="blob-code blob-code-inner js-file-line">Nevertheless, the chopping component can be detected in many cases, and it leads to a unique measurement</td>
      </tr>
      <tr>
        <td id="L303" class="blob-num js-line-number" data-line-number="303"></td>
        <td id="LC303" class="blob-code blob-code-inner js-file-line">of the masses of the exoplanets <span class="pl-c1">\citep</span>{2014ApJ...790...58N,2014ApJ...795..167S,2015ApJ...802..116D}.</td>
      </tr>
      <tr>
        <td id="L304" class="blob-num js-line-number" data-line-number="304"></td>
        <td id="LC304" class="blob-code blob-code-inner js-file-line">
</td>
      </tr>
      <tr>
        <td id="L305" class="blob-num js-line-number" data-line-number="305"></td>
        <td id="LC305" class="blob-code blob-code-inner js-file-line">As an example, consider a pair of planets with period ratio of <span class="pl-s"><span class="pl-pds">$</span>P_<span class="pl-c1">2</span>/P_<span class="pl-c1">1</span> = <span class="pl-c1">1.52</span><span class="pl-pds">$</span></span>.  This period ratio</td>
      </tr>
      <tr>
        <td id="L306" class="blob-num js-line-number" data-line-number="306"></td>
        <td id="LC306" class="blob-code blob-code-inner js-file-line">is close to 3:2, and thus is affected by this resonant term, giving a TTV period of <span class="pl-s"><span class="pl-pds">$</span><span class="pl-c1">38</span> P_<span class="pl-c1">1</span><span class="pl-pds">$</span></span> by equation~<span class="pl-c1">\ref</span>{eqn:pttv}.</td>
      </tr>
      <tr>
        <td id="L307" class="blob-num js-line-number" data-line-number="307"></td>
        <td id="LC307" class="blob-code blob-code-inner js-file-line">Figure <span class="pl-c1">\ref</span>{ttv_chopping} compares two planets with this period ratio with zero eccentricity</td>
      </tr>
      <tr>
        <td id="L308" class="blob-num js-line-number" data-line-number="308"></td>
        <td id="LC308" class="blob-code blob-code-inner js-file-line">and mass-ratios of <span class="pl-s"><span class="pl-pds">$</span><span class="pl-c1">10</span>^{-6}<span class="pl-pds">$</span></span> to a pair of planets with eccentricites of <span class="pl-s"><span class="pl-pds">$</span>e_<span class="pl-c1">1</span>=e_<span class="pl-c1">2</span>=<span class="pl-c1">0.04</span><span class="pl-pds">$</span></span></td>
      </tr>
      <tr>
        <td id="L309" class="blob-num js-line-number" data-line-number="309"></td>
        <td id="LC309" class="blob-code blob-code-inner js-file-line">and mass-ratios near <span class="pl-s"><span class="pl-pds">$</span><span class="pl-c1">10</span>^{-7}<span class="pl-pds">$</span></span>.  Both pairs of planets give nearly identical amplitudes</td>
      </tr>
      <tr>
        <td id="L310" class="blob-num js-line-number" data-line-number="310"></td>
        <td id="LC310" class="blob-code blob-code-inner js-file-line">for the large resonant term due to the mass-eccentricity discussed above, while the larger mass ratio planets show a </td>
      </tr>
      <tr>
        <td id="L311" class="blob-num js-line-number" data-line-number="311"></td>
        <td id="LC311" class="blob-code blob-code-inner js-file-line">much stronger chopping variation.  In this case there is a clear difference between the TTVs of the two simulated</td>
      </tr>
      <tr>
        <td id="L312" class="blob-num js-line-number" data-line-number="312"></td>
        <td id="LC312" class="blob-code blob-code-inner js-file-line">systems:  the inner planet shows a drift over three orbital periods, and a sudden jump</td>
      </tr>
      <tr>
        <td id="L313" class="blob-num js-line-number" data-line-number="313"></td>
        <td id="LC313" class="blob-code blob-code-inner js-file-line">every third orbital period, while the outer one shows a similar pattern over two orbital</td>
      </tr>
      <tr>
        <td id="L314" class="blob-num js-line-number" data-line-number="314"></td>
        <td id="LC314" class="blob-code blob-code-inner js-file-line">periods.  These variations are due to perturbations at integer multiples of the synodic</td>
      </tr>
      <tr>
        <td id="L315" class="blob-num js-line-number" data-line-number="315"></td>
        <td id="LC315" class="blob-code blob-code-inner js-file-line">frequency, which has a period <span class="pl-s"><span class="pl-pds">$</span>P_{<span class="pl-c1">\rm</span> syn}= (<span class="pl-c1">1</span>/P_<span class="pl-c1">1</span>-<span class="pl-c1">1</span>/P_<span class="pl-c1">2</span>)^{-1}<span class="pl-pds">$</span></span>; this is the frequency between conjunctions of</td>
      </tr>
      <tr>
        <td id="L316" class="blob-num js-line-number" data-line-number="316"></td>
        <td id="LC316" class="blob-code blob-code-inner js-file-line">the planets when they perturb one another most strongly.  In this example the phase of the orbital parameters are set such that the TTV amplitudes match;  change</td>
      </tr>
      <tr>
        <td id="L317" class="blob-num js-line-number" data-line-number="317"></td>
        <td id="LC317" class="blob-code blob-code-inner js-file-line">in the phase can also be indicative of a non-zero eccentricity contributing to the TTVs,</td>
      </tr>
      <tr>
        <td id="L318" class="blob-num js-line-number" data-line-number="318"></td>
        <td id="LC318" class="blob-code blob-code-inner js-file-line">and with an ensemble of planets which are believed to have a similar eccentricity</td>
      </tr>
      <tr>
        <td id="L319" class="blob-num js-line-number" data-line-number="319"></td>
        <td id="LC319" class="blob-code blob-code-inner js-file-line">distribution, the mass-eccentricity degeneracy may be broken statistically <span class="pl-c1">\citep</span>{2012ApJ...761..122L,</td>
      </tr>
      <tr>
        <td id="L320" class="blob-num js-line-number" data-line-number="320"></td>
        <td id="LC320" class="blob-code blob-code-inner js-file-line">2014ApJ...787...80H}.</td>
      </tr>
      <tr>
        <td id="L321" class="blob-num js-line-number" data-line-number="321"></td>
        <td id="LC321" class="blob-code blob-code-inner js-file-line">
</td>
      </tr>
      <tr>
        <td id="L322" class="blob-num js-line-number" data-line-number="322"></td>
        <td id="LC322" class="blob-code blob-code-inner js-file-line"><span class="pl-c"><span class="pl-c">%</span> For figures use</span></td>
      </tr>
      <tr>
        <td id="L323" class="blob-num js-line-number" data-line-number="323"></td>
        <td id="LC323" class="blob-code blob-code-inner js-file-line"><span class="pl-c1">\begin</span>{figure}</td>
      </tr>
      <tr>
        <td id="L324" class="blob-num js-line-number" data-line-number="324"></td>
        <td id="LC324" class="blob-code blob-code-inner js-file-line"><span class="pl-c1">\centerline</span>{</td>
      </tr>
      <tr>
        <td id="L325" class="blob-num js-line-number" data-line-number="325"></td>
        <td id="LC325" class="blob-code blob-code-inner js-file-line"><span class="pl-c1">\includegraphics</span>[width=0.9<span class="pl-c1">\textwidth</span>]{ttv_chopping.pdf}}</td>
      </tr>
      <tr>
        <td id="L326" class="blob-num js-line-number" data-line-number="326"></td>
        <td id="LC326" class="blob-code blob-code-inner js-file-line"><span class="pl-c"><span class="pl-c">%</span></span></td>
      </tr>
      <tr>
        <td id="L327" class="blob-num js-line-number" data-line-number="327"></td>
        <td id="LC327" class="blob-code blob-code-inner js-file-line"><span class="pl-c1">\caption</span>{Transit-timing variations of two low-eccentricity planets with larger</td>
      </tr>
      <tr>
        <td id="L328" class="blob-num js-line-number" data-line-number="328"></td>
        <td id="LC328" class="blob-code blob-code-inner js-file-line">mass ratios, <span class="pl-s"><span class="pl-pds">$</span>m_<span class="pl-c1">1</span> = m_<span class="pl-c1">2</span> = <span class="pl-c1">10</span>^{-6} m_*<span class="pl-pds">$</span></span> (green) compared with two higher eccentricity planets (<span class="pl-s"><span class="pl-pds">$</span>e_<span class="pl-c1">1</span>=e_<span class="pl-c1">2</span>=<span class="pl-c1">0.04</span><span class="pl-pds">$</span></span>)</td>
      </tr>
      <tr>
        <td id="L329" class="blob-num js-line-number" data-line-number="329"></td>
        <td id="LC329" class="blob-code blob-code-inner js-file-line">with smaller mass ratios <span class="pl-s"><span class="pl-pds">$</span>m_<span class="pl-c1">1</span> = m_<span class="pl-c1">2</span> = <span class="pl-c1">10</span>^{-7} m_*<span class="pl-pds">$</span></span>.  The zig-zag chopping component</td>
      </tr>
      <tr>
        <td id="L330" class="blob-num js-line-number" data-line-number="330"></td>
        <td id="LC330" class="blob-code blob-code-inner js-file-line">is apparent in the high-mass/low-eccentricity case, while less apparent in the low-mass/</td>
      </tr>
      <tr>
        <td id="L331" class="blob-num js-line-number" data-line-number="331"></td>
        <td id="LC331" class="blob-code blob-code-inner js-file-line">high-eccentricity case.}</td>
      </tr>
      <tr>
        <td id="L332" class="blob-num js-line-number" data-line-number="332"></td>
        <td id="LC332" class="blob-code blob-code-inner js-file-line"><span class="pl-c1">\label</span>{ttv_chopping}       <span class="pl-c"><span class="pl-c">%</span> Give a unique label</span></td>
      </tr>
      <tr>
        <td id="L333" class="blob-num js-line-number" data-line-number="333"></td>
        <td id="LC333" class="blob-code blob-code-inner js-file-line"><span class="pl-c1">\end</span>{figure}</td>
      </tr>
      <tr>
        <td id="L334" class="blob-num js-line-number" data-line-number="334"></td>
        <td id="LC334" class="blob-code blob-code-inner js-file-line">
</td>
      </tr>
      <tr>
        <td id="L335" class="blob-num js-line-number" data-line-number="335"></td>
        <td id="LC335" class="blob-code blob-code-inner js-file-line">
</td>
      </tr>
      <tr>
        <td id="L336" class="blob-num js-line-number" data-line-number="336"></td>
        <td id="LC336" class="blob-code blob-code-inner js-file-line">
</td>
      </tr>
      <tr>
        <td id="L337" class="blob-num js-line-number" data-line-number="337"></td>
        <td id="LC337" class="blob-code blob-code-inner js-file-line"><span class="pl-c1">\subsection</span>{Transit Duration Variations}</td>
      </tr>
      <tr>
        <td id="L338" class="blob-num js-line-number" data-line-number="338"></td>
        <td id="LC338" class="blob-code blob-code-inner js-file-line">
</td>
      </tr>
      <tr>
        <td id="L339" class="blob-num js-line-number" data-line-number="339"></td>
        <td id="LC339" class="blob-code blob-code-inner js-file-line">TDVs have given useful results for characterization of individual systems, though fewer in number than TTVs.</td>
      </tr>
      <tr>
        <td id="L340" class="blob-num js-line-number" data-line-number="340"></td>
        <td id="LC340" class="blob-code blob-code-inner js-file-line">
</td>
      </tr>
      <tr>
        <td id="L341" class="blob-num js-line-number" data-line-number="341"></td>
        <td id="LC341" class="blob-code blob-code-inner js-file-line">The most dramatic TDVs are due to the moving-target effect described for TTVs in the context of CBPs.  If the transit occurs while the star is moving in the same direction as the planet, the transit duration is longer, if in opposite directions, the transit duration is shorter.  Matching the prediction from the phase of the binary completely secures the interpretation of the signal that an object is in a circumbinary orbit, as discussed extensively by <span class="pl-c1">\cite</span>{2013ApJ...770...52K} for the cases of Kepler-47 and Kepler-64 (a.k.a. PH-1, KIC 4862625b).</td>
      </tr>
      <tr>
        <td id="L342" class="blob-num js-line-number" data-line-number="342"></td>
        <td id="LC342" class="blob-code blob-code-inner js-file-line">
</td>
      </tr>
      <tr>
        <td id="L343" class="blob-num js-line-number" data-line-number="343"></td>
        <td id="LC343" class="blob-code blob-code-inner js-file-line">Just as in several known cases of stellar triples, the precession of CBPs has caused transits to turn on and off <span class="pl-c1">\citep</span>{2017MNRAS.465.3235M}.  The first case of that phenomenon was Kepler-35 <span class="pl-c1">\citep</span>{2012Natur.481..475W}, and the most spectacular so far observed was Kepler-413 <span class="pl-c1">\citep</span>{2014ApJ...784...14K}, in which a <span class="pl-s"><span class="pl-pds">$</span><span class="pl-c1">4</span>^<span class="pl-c1">\circ</span><span class="pl-pds">$</span></span> mutual inclination caused transits to stop and then start again nearly half a precession cycle later.  Planetary torques also cause TDV, albeit on a longer timescale through changing the inclination.  This was observed in Kepler-117 <span class="pl-c1">\citep</span>{2015MNRAS.453.2644A} as well as Kepler-108 <span class="pl-c1">\citep</span>{1538-3881-153-1-45}, the latter indicating mutual inclination of <span class="pl-s"><span class="pl-pds">$</span><span class="pl-c1">\sim</span> <span class="pl-c1">15</span>^<span class="pl-c1">\circ</span><span class="pl-pds">$</span></span> in a rather hierarchical pair of planets. </td>
      </tr>
      <tr>
        <td id="L344" class="blob-num js-line-number" data-line-number="344"></td>
        <td id="LC344" class="blob-code blob-code-inner js-file-line">
</td>
      </tr>
      <tr>
        <td id="L345" class="blob-num js-line-number" data-line-number="345"></td>
        <td id="LC345" class="blob-code blob-code-inner js-file-line">Two other mechanisms for TDV have been observed in planetary system orbiting a single primary star.</td>
      </tr>
      <tr>
        <td id="L346" class="blob-num js-line-number" data-line-number="346"></td>
        <td id="LC346" class="blob-code blob-code-inner js-file-line">
</td>
      </tr>
      <tr>
        <td id="L347" class="blob-num js-line-number" data-line-number="347"></td>
        <td id="LC347" class="blob-code blob-code-inner js-file-line">The first is torque due to the rotational oblateness of the star.  It is a convincing model for the duration changes in Kepler-13 b <span class="pl-c1">\citep</span>[KOI 13.01][]{Szab2012} and a controversial explanation for transit shape anomalies in PTFO 8-8695 <span class="pl-c1">\citep</span>{2013ApJ...774...53B}. </td>
      </tr>
      <tr>
        <td id="L348" class="blob-num js-line-number" data-line-number="348"></td>
        <td id="LC348" class="blob-code blob-code-inner js-file-line">
</td>
      </tr>
      <tr>
        <td id="L349" class="blob-num js-line-number" data-line-number="349"></td>
        <td id="LC349" class="blob-code blob-code-inner js-file-line">The last planetary cause of TDVs is dramatic eccentricity variations due to a resonant interaction.  The length of the chord across the star, as well as the speed at which the planet moves along that chord, are changed during the planetary interaction.  This effect has been observed in KOI-142 <span class="pl-c1">\citep</span>{2013ApJ...777....3N}.</td>
      </tr>
      <tr>
        <td id="L350" class="blob-num js-line-number" data-line-number="350"></td>
        <td id="LC350" class="blob-code blob-code-inner js-file-line">
</td>
      </tr>
      <tr>
        <td id="L351" class="blob-num js-line-number" data-line-number="351"></td>
        <td id="LC351" class="blob-code blob-code-inner js-file-line">Finally, it has not been measured yet, but slow secular precession of the eccentricity is expected by general relativity <span class="pl-c1">\citep</span>{2008MNRAS.389..191P}, by stellar oblateness <span class="pl-c1">\citep</span>{2007MNRAS.377.1511H}, and by tidal distortion <span class="pl-c1">\citep</span>{2009ApJ...698.1778R}.  It is likely that very long time-baseline measurements, or comparing the measurements of two time-separated space missions like \emph{Kepler} and \emph{Plato}, will be able to detect this effect.</td>
      </tr>
      <tr>
        <td id="L352" class="blob-num js-line-number" data-line-number="352"></td>
        <td id="LC352" class="blob-code blob-code-inner js-file-line">
</td>
      </tr>
      <tr>
        <td id="L353" class="blob-num js-line-number" data-line-number="353"></td>
        <td id="LC353" class="blob-code blob-code-inner js-file-line"><span class="pl-c1">\section</span>{Observational considerations: timing precision} <span class="pl-c"><span class="pl-c">%</span> EA</span></td>
      </tr>
      <tr>
        <td id="L354" class="blob-num js-line-number" data-line-number="354"></td>
        <td id="LC354" class="blob-code blob-code-inner js-file-line">
</td>
      </tr>
      <tr>
        <td id="L355" class="blob-num js-line-number" data-line-number="355"></td>
        <td id="LC355" class="blob-code blob-code-inner js-file-line">The steepest portions of a transit are the ingress and egress when the planet crosses onto and</td>
      </tr>
      <tr>
        <td id="L356" class="blob-num js-line-number" data-line-number="356"></td>
        <td id="LC356" class="blob-code blob-code-inner js-file-line">off of the disk of the star, causing a dip of depth <span class="pl-s"><span class="pl-pds">$</span><span class="pl-c1">\delta</span> = (R_p/R_*)^<span class="pl-c1">2</span><span class="pl-pds">$</span></span> if limb-darkening is ignored.   </td>
      </tr>
      <tr>
        <td id="L357" class="blob-num js-line-number" data-line-number="357"></td>
        <td id="LC357" class="blob-code blob-code-inner js-file-line">Suppose for the moment that the only source of noise is Poisson noise due to the count rate of </td>
      </tr>
      <tr>
        <td id="L358" class="blob-num js-line-number" data-line-number="358"></td>
        <td id="LC358" class="blob-code blob-code-inner js-file-line">the star, <span class="pl-s"><span class="pl-pds">$</span><span class="pl-c1">\dot</span> N<span class="pl-pds">$</span></span>.  The photometric precision over the duration of ingress scales as</td>
      </tr>
      <tr>
        <td id="L359" class="blob-num js-line-number" data-line-number="359"></td>
        <td id="LC359" class="blob-code blob-code-inner js-file-line"><span class="pl-s"><span class="pl-pds">$</span>(<span class="pl-c1">\dot</span> N <span class="pl-c1">\tau</span>)^{1/2}<span class="pl-pds">$</span></span>, where <span class="pl-s"><span class="pl-pds">$</span><span class="pl-c1">\tau</span><span class="pl-pds">$</span></span> is the ingress duration (eqn.<span class="pl-cce">\ </span><span class="pl-c1">\ref</span>{ingress}).  If the time of </td>
      </tr>
      <tr>
        <td id="L360" class="blob-num js-line-number" data-line-number="360"></td>
        <td id="LC360" class="blob-code blob-code-inner js-file-line">ingress fit from a model is offset by <span class="pl-s"><span class="pl-pds">$</span><span class="pl-c1">\sigma</span>_g<span class="pl-pds">$</span></span>, then the difference in counts observed</td>
      </tr>
      <tr>
        <td id="L361" class="blob-num js-line-number" data-line-number="361"></td>
        <td id="LC361" class="blob-code blob-code-inner js-file-line">versus the model is <span class="pl-s"><span class="pl-pds">$</span><span class="pl-c1">\sigma</span>_g <span class="pl-c1">\delta</span> <span class="pl-c1">\dot</span> N<span class="pl-pds">$</span></span> (the pink region in Fig.<span class="pl-cce">\ </span><span class="pl-c1">\ref</span>{fig:ingress}).  Equating </td>
      </tr>
      <tr>
        <td id="L362" class="blob-num js-line-number" data-line-number="362"></td>
        <td id="LC362" class="blob-code blob-code-inner js-file-line">this to photometric uncertainty gives <span class="pl-s"><span class="pl-pds">$</span><span class="pl-c1">\sigma</span>_g = <span class="pl-c1">\tau</span>^{1/2} <span class="pl-c1">\dot</span> N^{-1/2} <span class="pl-c1">\delta</span>^{-1}<span class="pl-pds">$</span></span>, which</td>
      </tr>
      <tr>
        <td id="L363" class="blob-num js-line-number" data-line-number="363"></td>
        <td id="LC363" class="blob-code blob-code-inner js-file-line">is the 68.3<span class="pl-cce">\%</span> confidence timing precision assuming that the exposure time is much shorter than the</td>
      </tr>
      <tr>
        <td id="L364" class="blob-num js-line-number" data-line-number="364"></td>
        <td id="LC364" class="blob-code blob-code-inner js-file-line">ingress duration and that <span class="pl-s"><span class="pl-pds">$</span><span class="pl-c1">\sigma</span>_g <span class="pl-c1">\ll</span> <span class="pl-c1">\tau</span><span class="pl-pds">$</span></span>.  The same formula applies to egress. A longer transit </td>
      </tr>
      <tr>
        <td id="L365" class="blob-num js-line-number" data-line-number="365"></td>
        <td id="LC365" class="blob-code blob-code-inner js-file-line">ingress duration leads to a shallower slope in ingress, which makes it more difficult to measure an </td>
      </tr>
      <tr>
        <td id="L366" class="blob-num js-line-number" data-line-number="366"></td>
        <td id="LC366" class="blob-code blob-code-inner js-file-line">offset in time of the model.  Higher count rates and deeper transits improve the precision, as </td>
      </tr>
      <tr>
        <td id="L367" class="blob-num js-line-number" data-line-number="367"></td>
        <td id="LC367" class="blob-code blob-code-inner js-file-line">expected.  Note that we&#39;ve assumed that the duration of the transit is sufficiently long that the </td>
      </tr>
      <tr>
        <td id="L368" class="blob-num js-line-number" data-line-number="368"></td>
        <td id="LC368" class="blob-code blob-code-inner js-file-line">error on <span class="pl-s"><span class="pl-pds">$</span><span class="pl-c1">\delta</span><span class="pl-pds">$</span></span> is small.</td>
      </tr>
      <tr>
        <td id="L369" class="blob-num js-line-number" data-line-number="369"></td>
        <td id="LC369" class="blob-code blob-code-inner js-file-line">
</td>
      </tr>
      <tr>
        <td id="L370" class="blob-num js-line-number" data-line-number="370"></td>
        <td id="LC370" class="blob-code blob-code-inner js-file-line">Suppose the transit duration is <span class="pl-s"><span class="pl-pds">$</span>T<span class="pl-pds">$</span></span>.  Then, the uncertainty on the duration is given by</td>
      </tr>
      <tr>
        <td id="L371" class="blob-num js-line-number" data-line-number="371"></td>
        <td id="LC371" class="blob-code blob-code-inner js-file-line">the sum of the uncertainties on the ingress and egress, added in quadrature:</td>
      </tr>
      <tr>
        <td id="L372" class="blob-num js-line-number" data-line-number="372"></td>
        <td id="LC372" class="blob-code blob-code-inner js-file-line"><span class="pl-s"><span class="pl-pds">$</span><span class="pl-c1">\sigma</span>_T = <span class="pl-c1">\sqrt</span>{2} <span class="pl-c1">\sigma</span>_g<span class="pl-pds">$</span></span>.  The timing precision, <span class="pl-s"><span class="pl-pds">$</span><span class="pl-c1">\sigma</span>_t<span class="pl-pds">$</span></span>, is set by the mean of the ingress</td>
      </tr>
      <tr>
        <td id="L373" class="blob-num js-line-number" data-line-number="373"></td>
        <td id="LC373" class="blob-code blob-code-inner js-file-line">and egress, giving <span class="pl-s"><span class="pl-pds">$</span><span class="pl-c1">\sigma</span>_t = <span class="pl-c1">\frac</span>{1}{<span class="pl-c1">\sqrt</span>{2}} <span class="pl-c1">\sigma</span>_g<span class="pl-pds">$</span></span>.</td>
      </tr>
      <tr>
        <td id="L374" class="blob-num js-line-number" data-line-number="374"></td>
        <td id="LC374" class="blob-code blob-code-inner js-file-line">
</td>
      </tr>
      <tr>
        <td id="L375" class="blob-num js-line-number" data-line-number="375"></td>
        <td id="LC375" class="blob-code blob-code-inner js-file-line">A more complete derivation of these expressions is given by <span class="pl-c1">\citet</span>{2008ApJ...689..499C}, while an </td>
      </tr>
      <tr>
        <td id="L376" class="blob-num js-line-number" data-line-number="376"></td>
        <td id="LC376" class="blob-code blob-code-inner js-file-line">expression which includes the effects of a finite integration time is given by <span class="pl-c1">\citet</span>{2014ApJ...794...92P}.</td>
      </tr>
      <tr>
        <td id="L377" class="blob-num js-line-number" data-line-number="377"></td>
        <td id="LC377" class="blob-code blob-code-inner js-file-line">The assumptions of no limb-darkening and Poisson noise are generally broken by stars;  in addition,</td>
      </tr>
      <tr>
        <td id="L378" class="blob-num js-line-number" data-line-number="378"></td>
        <td id="LC378" class="blob-code blob-code-inner js-file-line">stellar variability contributes to timing uncertainty, for which there is yet to be a general</td>
      </tr>
      <tr>
        <td id="L379" class="blob-num js-line-number" data-line-number="379"></td>
        <td id="LC379" class="blob-code blob-code-inner js-file-line">expression.  These effects generally increase the uncertainty on the measurement of transit times</td>
      </tr>
      <tr>
        <td id="L380" class="blob-num js-line-number" data-line-number="380"></td>
        <td id="LC380" class="blob-code blob-code-inner js-file-line">and durations, and so the best practice would be to estimate the timing uncertainties from the</td>
      </tr>
      <tr>
        <td id="L381" class="blob-num js-line-number" data-line-number="381"></td>
        <td id="LC381" class="blob-code blob-code-inner js-file-line">data, accounting for effects of correlated stellar variability by including the full covariance</td>
      </tr>
      <tr>
        <td id="L382" class="blob-num js-line-number" data-line-number="382"></td>
        <td id="LC382" class="blob-code blob-code-inner js-file-line">matrix of the timing uncertainty <span class="pl-c1">\citep</span>{2009ApJ...704...51C,2012MNRAS.419.2683G}.  Crossing of</td>
      </tr>
      <tr>
        <td id="L383" class="blob-num js-line-number" data-line-number="383"></td>
        <td id="LC383" class="blob-code blob-code-inner js-file-line">the path of the planet across star spots may also cause some uncertainty on the timing precision <span class="pl-c1">\citep</span>{2013A&amp;A...556A..19O,2013MNRAS.430.3032B};</td>
      </tr>
      <tr>
        <td id="L384" class="blob-num js-line-number" data-line-number="384"></td>
        <td id="LC384" class="blob-code blob-code-inner js-file-line">this can be diagnosed by a larger scatter within transit than outside transit or other signs of</td>
      </tr>
      <tr>
        <td id="L385" class="blob-num js-line-number" data-line-number="385"></td>
        <td id="LC385" class="blob-code blob-code-inner js-file-line">significant stellar activity, and can be handled best by including the spots in the transit model </td>
      </tr>
      <tr>
        <td id="L386" class="blob-num js-line-number" data-line-number="386"></td>
        <td id="LC386" class="blob-code blob-code-inner js-file-line"><span class="pl-c1">\citep</span>{2016A&amp;A...585A..72I}.</td>
      </tr>
      <tr>
        <td id="L387" class="blob-num js-line-number" data-line-number="387"></td>
        <td id="LC387" class="blob-code blob-code-inner js-file-line">
</td>
      </tr>
      <tr>
        <td id="L388" class="blob-num js-line-number" data-line-number="388"></td>
        <td id="LC388" class="blob-code blob-code-inner js-file-line">Note that the barycentric light-travel time offset must be corrected for carefully for high-precision</td>
      </tr>
      <tr>
        <td id="L389" class="blob-num js-line-number" data-line-number="389"></td>
        <td id="LC389" class="blob-code blob-code-inner js-file-line">TTV <span class="pl-c1">\citep</span>{2010PASP..122..935E}.</td>
      </tr>
      <tr>
        <td id="L390" class="blob-num js-line-number" data-line-number="390"></td>
        <td id="LC390" class="blob-code blob-code-inner js-file-line">
</td>
      </tr>
      <tr>
        <td id="L391" class="blob-num js-line-number" data-line-number="391"></td>
        <td id="LC391" class="blob-code blob-code-inner js-file-line"><span class="pl-c"><span class="pl-c">%</span> For figures use</span></td>
      </tr>
      <tr>
        <td id="L392" class="blob-num js-line-number" data-line-number="392"></td>
        <td id="LC392" class="blob-code blob-code-inner js-file-line"><span class="pl-c1">\begin</span>{figure}</td>
      </tr>
      <tr>
        <td id="L393" class="blob-num js-line-number" data-line-number="393"></td>
        <td id="LC393" class="blob-code blob-code-inner js-file-line"><span class="pl-c"><span class="pl-c">%</span> Use the relevant command for your figure-insertion program</span></td>
      </tr>
      <tr>
        <td id="L394" class="blob-num js-line-number" data-line-number="394"></td>
        <td id="LC394" class="blob-code blob-code-inner js-file-line"><span class="pl-c"><span class="pl-c">%</span> to insert the figure file.</span></td>
      </tr>
      <tr>
        <td id="L395" class="blob-num js-line-number" data-line-number="395"></td>
        <td id="LC395" class="blob-code blob-code-inner js-file-line"><span class="pl-c"><span class="pl-c">%</span> For example, with the graphicx style use</span></td>
      </tr>
      <tr>
        <td id="L396" class="blob-num js-line-number" data-line-number="396"></td>
        <td id="LC396" class="blob-code blob-code-inner js-file-line"><span class="pl-c1">\centerline</span>{</td>
      </tr>
      <tr>
        <td id="L397" class="blob-num js-line-number" data-line-number="397"></td>
        <td id="LC397" class="blob-code blob-code-inner js-file-line"><span class="pl-c1">\includegraphics</span>[width=0.5<span class="pl-c1">\textwidth</span>]{ingress.pdf}}</td>
      </tr>
      <tr>
        <td id="L398" class="blob-num js-line-number" data-line-number="398"></td>
        <td id="LC398" class="blob-code blob-code-inner js-file-line"><span class="pl-c"><span class="pl-c">%</span></span></td>
      </tr>
      <tr>
        <td id="L399" class="blob-num js-line-number" data-line-number="399"></td>
        <td id="LC399" class="blob-code blob-code-inner js-file-line"><span class="pl-c1">\caption</span>{Diagram of the transit ingress of a planet, flux versus time.  The precision of the timing of ingress, <span class="pl-s"><span class="pl-pds">$</span><span class="pl-c1">\sigma</span>_g<span class="pl-pds">$</span></span>, is set by</td>
      </tr>
      <tr>
        <td id="L400" class="blob-num js-line-number" data-line-number="400"></td>
        <td id="LC400" class="blob-code blob-code-inner js-file-line">when the area of the ingress (pink) equals the timing precision over the duration of ingress. The same applies to egress, albeit</td>
      </tr>
      <tr>
        <td id="L401" class="blob-num js-line-number" data-line-number="401"></td>
        <td id="LC401" class="blob-code blob-code-inner js-file-line">with the time flipped in this plot.}</td>
      </tr>
      <tr>
        <td id="L402" class="blob-num js-line-number" data-line-number="402"></td>
        <td id="LC402" class="blob-code blob-code-inner js-file-line"><span class="pl-c1">\label</span>{fig:ingress}       <span class="pl-c"><span class="pl-c">%</span> Give a unique label</span></td>
      </tr>
      <tr>
        <td id="L403" class="blob-num js-line-number" data-line-number="403"></td>
        <td id="LC403" class="blob-code blob-code-inner js-file-line"><span class="pl-c1">\end</span>{figure}</td>
      </tr>
      <tr>
        <td id="L404" class="blob-num js-line-number" data-line-number="404"></td>
        <td id="LC404" class="blob-code blob-code-inner js-file-line">
</td>
      </tr>
      <tr>
        <td id="L405" class="blob-num js-line-number" data-line-number="405"></td>
        <td id="LC405" class="blob-code blob-code-inner js-file-line"><span class="pl-c1">\section</span>{Science Results}</td>
      </tr>
      <tr>
        <td id="L406" class="blob-num js-line-number" data-line-number="406"></td>
        <td id="LC406" class="blob-code blob-code-inner js-file-line">
</td>
      </tr>
      <tr>
        <td id="L407" class="blob-num js-line-number" data-line-number="407"></td>
        <td id="LC407" class="blob-code blob-code-inner js-file-line">The best characterized pair of small planets to date using TTV reside in the Kepler-36 system \citep{2012Sci...337..556C}.  As this planet pair is in close proximity, the conjunctions cause a significant kick to each planet resulting a TTV amplitude that is $\approx 1$\% of the orbital periods of the planets.  Figure \ref{fig:kep36} shows a `river-plot&#39; for all seventeen quarters of long-cadence \emph{Kepler} data for this pair of planets.  After each 7(6) orbits of the inner(outer) planet (or so), there is a conjunction which causes a change in the eccentricity vector and period of each planet.  The change in the eccentricity vector causes a sudden change in the subsequent transit time, while the change in period causes a change in slope;  these are apparent for Kepler-36c in Figure \ref{fig:kep36}.  The large TTVs enable a precise measurement of the planet-star mass ratios for both planets (using the TTVs of the companion planet), while the star shows asteroseismic variability which gives a precise estimate of the stellar mass.   The net result are masses with uncertainties of $&lt;8$\%, which is the most precise to date for planets of approximately this mass or lower, $4.5$ and $8.0 M_\oplus$.  The inner planet shows a density which is consistent with scaling up in mass a planet of the composition of Earth, while the outer planet requires a significant H/He envelope to explain its size which is comparable to Neptune \citep{2012Sci...337..556C}.</td>
      </tr>
      <tr>
        <td id="L408" class="blob-num js-line-number" data-line-number="408"></td>
        <td id="LC408" class="blob-code blob-code-inner js-file-line">
</td>
      </tr>
      <tr>
        <td id="L409" class="blob-num js-line-number" data-line-number="409"></td>
        <td id="LC409" class="blob-code blob-code-inner js-file-line"><span class="pl-c1">\begin</span>{figure}</td>
      </tr>
      <tr>
        <td id="L410" class="blob-num js-line-number" data-line-number="410"></td>
        <td id="LC410" class="blob-code blob-code-inner js-file-line"><span class="pl-c1">\centerline</span>{</td>
      </tr>
      <tr>
        <td id="L411" class="blob-num js-line-number" data-line-number="411"></td>
        <td id="LC411" class="blob-code blob-code-inner js-file-line"><span class="pl-c1">\includegraphics</span>[width=0.9<span class="pl-c1">\textwidth</span>]{kepler36.pdf}}</td>
      </tr>
      <tr>
        <td id="L412" class="blob-num js-line-number" data-line-number="412"></td>
        <td id="LC412" class="blob-code blob-code-inner js-file-line"><span class="pl-c1">\caption</span>{River plot of Kepler-36b (left) and Kepler-36c (right).}</td>
      </tr>
      <tr>
        <td id="L413" class="blob-num js-line-number" data-line-number="413"></td>
        <td id="LC413" class="blob-code blob-code-inner js-file-line"><span class="pl-c1">\label</span>{fig:kep36}</td>
      </tr>
      <tr>
        <td id="L414" class="blob-num js-line-number" data-line-number="414"></td>
        <td id="LC414" class="blob-code blob-code-inner js-file-line"><span class="pl-c1">\end</span>{figure}</td>
      </tr>
      <tr>
        <td id="L415" class="blob-num js-line-number" data-line-number="415"></td>
        <td id="LC415" class="blob-code blob-code-inner js-file-line">
</td>
      </tr>
      <tr>
        <td id="L416" class="blob-num js-line-number" data-line-number="416"></td>
        <td id="LC416" class="blob-code blob-code-inner js-file-line">
</td>
      </tr>
      <tr>
        <td id="L417" class="blob-num js-line-number" data-line-number="417"></td>
        <td id="LC417" class="blob-code blob-code-inner js-file-line">A wide-spread phenomenon was detected by TTV characterization of masses: the existence of puffy sub-Neptune planets.  In the first such case, Kepler-11 e <span class="pl-c1">\citep</span>{2011Natur.470...53L}, a planet with a mass half of Neptune&#39;s has a size slightly bigger than Neptune.  Even more extreme cases of this class have been found, the most extreme which we consider secure is a <span class="pl-s"><span class="pl-pds">$</span><span class="pl-c1">2.1</span>^{+1.5}_{-0.8} M_<span class="pl-c1">\oplus</span><span class="pl-pds">$</span></span> planet with a radius of <span class="pl-s"><span class="pl-pds">$</span><span class="pl-c1">7</span> R_<span class="pl-c1">\oplus</span><span class="pl-pds">$</span></span> in Kepler-51 <span class="pl-c1">\citep</span>{2014ApJ...783...53M}.  These massive envelopes mean these low-mass planets formed while gas was still present in the protoplanetary disk, and that they were able to capture that gas, a surprising result <span class="pl-c1">\citep</span>[e.g.][]{2016ApJ...817...90L,2016ApJ...825...29G}. </td>
      </tr>
      <tr>
        <td id="L418" class="blob-num js-line-number" data-line-number="418"></td>
        <td id="LC418" class="blob-code blob-code-inner js-file-line"> </td>
      </tr>
      <tr>
        <td id="L419" class="blob-num js-line-number" data-line-number="419"></td>
        <td id="LC419" class="blob-code blob-code-inner js-file-line">
</td>
      </tr>
      <tr>
        <td id="L420" class="blob-num js-line-number" data-line-number="420"></td>
        <td id="LC420" class="blob-code blob-code-inner js-file-line">Catalogs of transit times have been produced for the multi-planet \emph{Kepler} systems \citep{2013ApJS..208...16M,Rowe2015,2016ApJS..225....9H}. Several analyses of an ensemble of TTV pairs of planets have recently been carried out \citep{2014ApJ...787...80H,2013ApJS..208...22X,2014ApJS..210...25X,2016ApJ...820...39J}, with the largest by Hadden \&amp; Lithwick (2016), yielding constraints on the RMS eccentricity of the population of planets.  A slightly smaller sample, selecting only planets with mass precisions of better than $3-\sigma$, yields Figure \ref{fig:density_period}.  There appears to be a trend of mean density decreasing with orbital period (one exception is K2-3d, although the authors warn its RV mass estimate may be affected by stellar variability).  At periods near $\approx 10$ days, the RV and TTV densities agree rather well.  At shorter period, most of the RV detections are single-planets, which in general appear to have lower density relative to their multi-planet counterparts \citep{Steffen2016}.  When radius is plotted versus mass, and color-coded as a function of flux, Fig.\ \ref{fig:density_period}, there is a general trend of radius increasing with mass, albeit with a large scatter in mass, while a handful of `puffy&#39; planets (with masses measured with TTV) show shockingly large radii given their small masses.  These mass measurements are surprising, but difficult to dispute as larger masses would have led to a larger, and hence easier-to-measure, TTV signal.</td>
      </tr>
      <tr>
        <td id="L421" class="blob-num js-line-number" data-line-number="421"></td>
        <td id="LC421" class="blob-code blob-code-inner js-file-line">
</td>
      </tr>
      <tr>
        <td id="L422" class="blob-num js-line-number" data-line-number="422"></td>
        <td id="LC422" class="blob-code blob-code-inner js-file-line"><span class="pl-c1">\begin</span>{figure}</td>
      </tr>
      <tr>
        <td id="L423" class="blob-num js-line-number" data-line-number="423"></td>
        <td id="LC423" class="blob-code blob-code-inner js-file-line"><span class="pl-c1">\centerline</span>{</td>
      </tr>
      <tr>
        <td id="L424" class="blob-num js-line-number" data-line-number="424"></td>
        <td id="LC424" class="blob-code blob-code-inner js-file-line"><span class="pl-c1">\includegraphics</span>[width=0.95<span class="pl-c1">\textwidth</span>]{density_vs_period_errors.png}}</td>
      </tr>
      <tr>
        <td id="L425" class="blob-num js-line-number" data-line-number="425"></td>
        <td id="LC425" class="blob-code blob-code-inner js-file-line"><span class="pl-c1">\centerline</span>{</td>
      </tr>
      <tr>
        <td id="L426" class="blob-num js-line-number" data-line-number="426"></td>
        <td id="LC426" class="blob-code blob-code-inner js-file-line"><span class="pl-c1">\includegraphics</span>[width=0.95<span class="pl-c1">\textwidth</span>]{mass_radius_flux.png}}</td>
      </tr>
      <tr>
        <td id="L427" class="blob-num js-line-number" data-line-number="427"></td>
        <td id="LC427" class="blob-code blob-code-inner js-file-line"><span class="pl-c1">\caption</span>{Density vs.<span class="pl-cce">\ </span>period for planets transiting planets with masses <span class="pl-s"><span class="pl-pds">$</span>&lt;<span class="pl-c1">25</span> M_<span class="pl-c1">\oplus</span><span class="pl-pds">$</span></span> (left).  Radius</td>
      </tr>
      <tr>
        <td id="L428" class="blob-num js-line-number" data-line-number="428"></td>
        <td id="LC428" class="blob-code blob-code-inner js-file-line">vs.<span class="pl-cce">\ </span>mass, with color indicating incident stellar flux (right).}</td>
      </tr>
      <tr>
        <td id="L429" class="blob-num js-line-number" data-line-number="429"></td>
        <td id="LC429" class="blob-code blob-code-inner js-file-line"><span class="pl-c1">\label</span>{fig:density_period}</td>
      </tr>
      <tr>
        <td id="L430" class="blob-num js-line-number" data-line-number="430"></td>
        <td id="LC430" class="blob-code blob-code-inner js-file-line"><span class="pl-c1">\end</span>{figure}</td>
      </tr>
      <tr>
        <td id="L431" class="blob-num js-line-number" data-line-number="431"></td>
        <td id="LC431" class="blob-code blob-code-inner js-file-line">
</td>
      </tr>
      <tr>
        <td id="L432" class="blob-num js-line-number" data-line-number="432"></td>
        <td id="LC432" class="blob-code blob-code-inner js-file-line">With the end of the primary \emph{Kepler} mission, the data volume of transit times has diminished.</td>
      </tr>
      <tr>
        <td id="L433" class="blob-num js-line-number" data-line-number="433"></td>
        <td id="LC433" class="blob-code blob-code-inner js-file-line">Nevertheless, the K2 mission has continued to provide TTV systems such as WASP-47, the first</td>
      </tr>
      <tr>
        <td id="L434" class="blob-num js-line-number" data-line-number="434"></td>
        <td id="LC434" class="blob-code blob-code-inner js-file-line">short-period hot Jupiter with nearby planet companions <span class="pl-c1">\citep</span>{2015Becker}.  With the launch</td>
      </tr>
      <tr>
        <td id="L435" class="blob-num js-line-number" data-line-number="435"></td>
        <td id="LC435" class="blob-code blob-code-inner js-file-line">of TESS in 2018 <span class="pl-c1">\citep</span>{2015JATIS...1a4003R}, transit timing will enhance the analysis </td>
      </tr>
      <tr>
        <td id="L436" class="blob-num js-line-number" data-line-number="436"></td>
        <td id="LC436" class="blob-code blob-code-inner js-file-line">of the multi-planet systems found, especially near the polar regions with longer term coverage, </td>
      </tr>
      <tr>
        <td id="L437" class="blob-num js-line-number" data-line-number="437"></td>
        <td id="LC437" class="blob-code blob-code-inner js-file-line">or when followed up with CHEOPS <span class="pl-c1">\citep</span>{2014PASP..126.1134B}. </td>
      </tr>
      <tr>
        <td id="L438" class="blob-num js-line-number" data-line-number="438"></td>
        <td id="LC438" class="blob-code blob-code-inner js-file-line">The PLATO mission next decade will cause another spike in TTV science <span class="pl-c1">\citep</span>{2014ExA....38..249R},</td>
      </tr>
      <tr>
        <td id="L439" class="blob-num js-line-number" data-line-number="439"></td>
        <td id="LC439" class="blob-code blob-code-inner js-file-line">as will possibly WFIRST <span class="pl-c1">\citep</span>{2017PASP..129d4401M}. The James Webb Space Telescope may allow the</td>
      </tr>
      <tr>
        <td id="L440" class="blob-num js-line-number" data-line-number="440"></td>
        <td id="LC440" class="blob-code blob-code-inner js-file-line">extension in time baseline and increase in precision for high-priority transit-timing targets </td>
      </tr>
      <tr>
        <td id="L441" class="blob-num js-line-number" data-line-number="441"></td>
        <td id="LC441" class="blob-code blob-code-inner js-file-line"><span class="pl-c1">\citep</span>{2014PASP..126.1134B}.  All to say, the future of characterizing multi-transiting planet</td>
      </tr>
      <tr>
        <td id="L442" class="blob-num js-line-number" data-line-number="442"></td>
        <td id="LC442" class="blob-code blob-code-inner js-file-line">systems with TTV (and TDVs) looks promising.</td>
      </tr>
      <tr>
        <td id="L443" class="blob-num js-line-number" data-line-number="443"></td>
        <td id="LC443" class="blob-code blob-code-inner js-file-line">
</td>
      </tr>
      <tr>
        <td id="L444" class="blob-num js-line-number" data-line-number="444"></td>
        <td id="LC444" class="blob-code blob-code-inner js-file-line">
</td>
      </tr>
      <tr>
        <td id="L445" class="blob-num js-line-number" data-line-number="445"></td>
        <td id="LC445" class="blob-code blob-code-inner js-file-line"><span class="pl-c1">\begin</span>{acknowledgement}</td>
      </tr>
      <tr>
        <td id="L446" class="blob-num js-line-number" data-line-number="446"></td>
        <td id="LC446" class="blob-code blob-code-inner js-file-line">EA acknowledges support from NASA Grants NNX13AF20G, NNX13A124G, NNX13AF62G, from National Science Foundation (NSF) grant AST-1615315, and from NASA Astrobiology Institute&#39;s Virtual Planetary</td>
      </tr>
      <tr>
        <td id="L447" class="blob-num js-line-number" data-line-number="447"></td>
        <td id="LC447" class="blob-code blob-code-inner js-file-line">Laboratory, supported by NASA under cooperative agreement NNH05ZDA001C.  DCF acknowledges support from NASA under Grant No. NNX14AB87G issued through the \emph{Kepler} Participating Scientist Program and from the Alfred P. Sloan Foundation.  We thank Jason Steffen, Sam Hadden, Jack Lissauer, Kento Masuda, Mahmoudreza Oshagh, </td>
      </tr>
      <tr>
        <td id="L448" class="blob-num js-line-number" data-line-number="448"></td>
        <td id="LC448" class="blob-code blob-code-inner js-file-line"><span class="pl-c1">\end</span>{acknowledgement}</td>
      </tr>
      <tr>
        <td id="L449" class="blob-num js-line-number" data-line-number="449"></td>
        <td id="LC449" class="blob-code blob-code-inner js-file-line">
</td>
      </tr>
      <tr>
        <td id="L450" class="blob-num js-line-number" data-line-number="450"></td>
        <td id="LC450" class="blob-code blob-code-inner js-file-line"><span class="pl-c"><span class="pl-c">%</span>  IF you do NOT use bibtex, put comments before the following 2 lines</span></td>
      </tr>
      <tr>
        <td id="L451" class="blob-num js-line-number" data-line-number="451"></td>
        <td id="LC451" class="blob-code blob-code-inner js-file-line"><span class="pl-c1">\bibliographystyle</span>{spbasicHBexo}  <span class="pl-c"><span class="pl-c">%</span>for bibtex</span></td>
      </tr>
      <tr>
        <td id="L452" class="blob-num js-line-number" data-line-number="452"></td>
        <td id="LC452" class="blob-code blob-code-inner js-file-line"><span class="pl-c1">\bibliography</span>{agol_fabrycky} <span class="pl-c"><span class="pl-c">%</span>for bibtex-example</span></td>
      </tr>
      <tr>
        <td id="L453" class="blob-num js-line-number" data-line-number="453"></td>
        <td id="LC453" class="blob-code blob-code-inner js-file-line">
</td>
      </tr>
      <tr>
        <td id="L454" class="blob-num js-line-number" data-line-number="454"></td>
        <td id="LC454" class="blob-code blob-code-inner js-file-line"><span class="pl-c1">\end</span>{document}</td>
      </tr>
</table>


  </div>

  </div>

  <button type="button" data-facebox="#jump-to-line" data-facebox-class="linejump" data-hotkey="l" class="d-none">Jump to Line</button>
  <div id="jump-to-line" style="display:none">
    <!-- '"` --><!-- </textarea></xmp> --></option></form><form accept-charset="UTF-8" action="" class="js-jump-to-line-form" method="get"><div style="margin:0;padding:0;display:inline"><input name="utf8" type="hidden" value="&#x2713;" /></div>
      <input class="form-control linejump-input js-jump-to-line-field" type="text" placeholder="Jump to line&hellip;" aria-label="Jump to line" autofocus>
      <button type="submit" class="btn">Go</button>
</form>  </div>

  </div>
  <div class="modal-backdrop js-touch-events"></div>
</div>

    </div>
  </div>

  </div>

      
<div class="container site-footer-container">
  <div class="site-footer " role="contentinfo">
    <ul class="site-footer-links float-right">
        <li><a href="https://github.com/contact" data-ga-click="Footer, go to contact, text:contact">Contact GitHub</a></li>
      <li><a href="https://developer.github.com" data-ga-click="Footer, go to api, text:api">API</a></li>
      <li><a href="https://training.github.com" data-ga-click="Footer, go to training, text:training">Training</a></li>
      <li><a href="https://shop.github.com" data-ga-click="Footer, go to shop, text:shop">Shop</a></li>
        <li><a href="https://github.com/blog" data-ga-click="Footer, go to blog, text:blog">Blog</a></li>
        <li><a href="https://github.com/about" data-ga-click="Footer, go to about, text:about">About</a></li>

    </ul>

    <a href="https://github.com" aria-label="Homepage" class="site-footer-mark" title="GitHub">
      <svg aria-hidden="true" class="octicon octicon-mark-github" height="24" version="1.1" viewBox="0 0 16 16" width="24"><path fill-rule="evenodd" d="M8 0C3.58 0 0 3.58 0 8c0 3.54 2.29 6.53 5.47 7.59.4.07.55-.17.55-.38 0-.19-.01-.82-.01-1.49-2.01.37-2.53-.49-2.69-.94-.09-.23-.48-.94-.82-1.13-.28-.15-.68-.52-.01-.53.63-.01 1.08.58 1.23.82.72 1.21 1.87.87 2.33.66.07-.52.28-.87.51-1.07-1.78-.2-3.64-.89-3.64-3.95 0-.87.31-1.59.82-2.15-.08-.2-.36-1.02.08-2.12 0 0 .67-.21 2.2.82.64-.18 1.32-.27 2-.27.68 0 1.36.09 2 .27 1.53-1.04 2.2-.82 2.2-.82.44 1.1.16 1.92.08 2.12.51.56.82 1.27.82 2.15 0 3.07-1.87 3.75-3.65 3.95.29.25.54.73.54 1.48 0 1.07-.01 1.93-.01 2.2 0 .21.15.46.55.38A8.013 8.013 0 0 0 16 8c0-4.42-3.58-8-8-8z"/></svg>
</a>
    <ul class="site-footer-links">
      <li>&copy; 2017 <span title="0.17639s from unicorn-3740486861-5w8sz">GitHub</span>, Inc.</li>
        <li><a href="https://github.com/site/terms" data-ga-click="Footer, go to terms, text:terms">Terms</a></li>
        <li><a href="https://github.com/site/privacy" data-ga-click="Footer, go to privacy, text:privacy">Privacy</a></li>
        <li><a href="https://github.com/security" data-ga-click="Footer, go to security, text:security">Security</a></li>
        <li><a href="https://status.github.com/" data-ga-click="Footer, go to status, text:status">Status</a></li>
        <li><a href="https://help.github.com" data-ga-click="Footer, go to help, text:help">Help</a></li>
    </ul>
  </div>
</div>



  <div id="ajax-error-message" class="ajax-error-message flash flash-error">
    <svg aria-hidden="true" class="octicon octicon-alert" height="16" version="1.1" viewBox="0 0 16 16" width="16"><path fill-rule="evenodd" d="M8.865 1.52c-.18-.31-.51-.5-.87-.5s-.69.19-.87.5L.275 13.5c-.18.31-.18.69 0 1 .19.31.52.5.87.5h13.7c.36 0 .69-.19.86-.5.17-.31.18-.69.01-1L8.865 1.52zM8.995 13h-2v-2h2v2zm0-3h-2V6h2v4z"/></svg>
    <button type="button" class="flash-close js-flash-close js-ajax-error-dismiss" aria-label="Dismiss error">
      <svg aria-hidden="true" class="octicon octicon-x" height="16" version="1.1" viewBox="0 0 12 16" width="12"><path fill-rule="evenodd" d="M7.48 8l3.75 3.75-1.48 1.48L6 9.48l-3.75 3.75-1.48-1.48L4.52 8 .77 4.25l1.48-1.48L6 6.52l3.75-3.75 1.48 1.48z"/></svg>
    </button>
    You can't perform that action at this time.
  </div>


    
    <script crossorigin="anonymous" integrity="sha256-+Eu4exSWhdHmxvBX7jJPLNSW5nf1o1motduFMxO7g+Y=" src="https://assets-cdn.github.com/assets/frameworks-f84bb87b149685d1e6c6f057ee324f2cd496e677f5a359a8b5db853313bb83e6.js"></script>
    
    <script async="async" crossorigin="anonymous" integrity="sha256-ShUbbPU8svmGOF51j4uI9EeR0yHsGSP0DwV2nhQZ7fQ=" src="https://assets-cdn.github.com/assets/github-4a151b6cf53cb2f986385e758f8b88f44791d321ec1923f40f05769e1419edf4.js"></script>
    
    
    
    
  <div class="js-stale-session-flash stale-session-flash flash flash-warn flash-banner d-none">
    <svg aria-hidden="true" class="octicon octicon-alert" height="16" version="1.1" viewBox="0 0 16 16" width="16"><path fill-rule="evenodd" d="M8.865 1.52c-.18-.31-.51-.5-.87-.5s-.69.19-.87.5L.275 13.5c-.18.31-.18.69 0 1 .19.31.52.5.87.5h13.7c.36 0 .69-.19.86-.5.17-.31.18-.69.01-1L8.865 1.52zM8.995 13h-2v-2h2v2zm0-3h-2V6h2v4z"/></svg>
    <span class="signed-in-tab-flash">You signed in with another tab or window. <a href="">Reload</a> to refresh your session.</span>
    <span class="signed-out-tab-flash">You signed out in another tab or window. <a href="">Reload</a> to refresh your session.</span>
  </div>
  <div class="facebox" id="facebox" style="display:none;">
  <div class="facebox-popup">
    <div class="facebox-content" role="dialog" aria-labelledby="facebox-header" aria-describedby="facebox-description">
    </div>
    <button type="button" class="facebox-close js-facebox-close" aria-label="Close modal">
      <svg aria-hidden="true" class="octicon octicon-x" height="16" version="1.1" viewBox="0 0 12 16" width="12"><path fill-rule="evenodd" d="M7.48 8l3.75 3.75-1.48 1.48L6 9.48l-3.75 3.75-1.48-1.48L4.52 8 .77 4.25l1.48-1.48L6 6.52l3.75-3.75 1.48 1.48z"/></svg>
    </button>
  </div>
</div>


  </body>
</html>

